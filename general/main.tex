\documentclass{report}

\usepackage{rubrics}

\begin{document}

\chap{Liturgical Action}{

\section{Uniformity of Action}

\rubric When two or more are preforming an action, e.g. genuflecting, bowing,
walking, reciting a prayer, they should act \textit{together} at the same time
and with the same speed. Nothing is more conducive to the smoothness of a
ceremony.

\section{Precedence}

\rubric In all ceremonies, strict precedence is observed:

\begin{enumerate}

    \item Superiors walk at the right of, or behind, inferiors; the place of
        dignity in a procession is at the end.\footnote{This refers to
        procession in which the celebrant is \textit{vested.} If he wears only
        cassock and surplice, however, he leads the procession, and the order
        of dignity is reversed.}

    \item When possible, inferiors cross behind, rather than in front of,
        superiors.

    \item Those of lower rank sit after those of higher tank, but rise before
        them.

    \item In the sanctuary those of higher tank sit on the Gospel side.

\end{enumerate}

\section{Standing, Sitting, Walking, Turning}

\rubric \textbf{Standing.} To stand for someone is a sign of reverence. Thus,
in the Mass it is proper to rise at the Gospel because the Gospel is the word
of God; similarly, the congregation rises for the celebrant who is about to
greet them with the words \textit{Dóminus vobíscum.}

\rubric \textbf{Sitting.} The position of sitting denotes authority to teach
and rule. For this reason it is customary for the bishop to sit during a
liturgical function.

\rubric \textbf{Walking.} Always walk erect, at a moderate pace, and without
gazing about. Never walk backwards or sideways. When encountering an obstacle,
step off to the right so that you will pass it at your left; e.g., when the
sacristy is behind tha ltar, approach the altar from the Gospel side and leave
it from the Epistle side. When mounting steps, always place your right foot
first.

In passing an altar at which Mass is being celebrated:

\begin{enumerate}

    \item If the Consecration is just taking place, kneel and bow; remain
        kneeling until the chalice has been replaced on the altar. Then bow,
        rise, and continue on your way.

    \item After the Consecration, but before the Communion, genuflect on one
        knee.

    \item If Communion is being distributed, make a double genuflection.

\end{enumerate}

\rubric \textbf{Turning.} When walking alone:

\begin{enumerate}

    \item Always turn toward the person or thing of greater dignity, e.g., the
        Blessed Sacrament, altar, bishop, celebrant. In cases of conflict of
        dignity, decide in favor of the nearer, e.g., at the atlar turn toward
        it and away from the bishop or celebrant.

    \item When you are free to turn either way, turn toward the right.

    \item When walking with another, ordinarily, turn toward the person with
        whom you are walking. Under no circumstances should you turn your back
        to the Blessed Sacrament if It is exposed.

\end{enumerate}

}

\chap{Low Mass}{
    \section{Preparation}

    \rubric The celebrant and servers bow to the sacristy cross. Acolyte two
    leads the way to the chapel and rings the bell. Acolyte one presents holy
    water to the celebrant.

    \rubric Upon arriving at his place at the Gospel side, acolyte two stands
    in his place, acolyte one walks behind acolyte two to his place on the
    Epistle side, and the celebrant stands between them. Acolyte one takes the
    biretta from the celebrant, kissing first his hand, then the biretta. All
    three genuflect to the cross, the celebrant bowing if the Blessed Sacrament
    is not reserved.

    \rubric The celebrant ascends the altar. Acolyte one places the biretta on
    the sedillia. Acolyte two returns the way the procession came and closes
    the door to the sacristy. Both acolytes return to their places, one not
    waiting for the other.

    \section{Mass of the Catechumens}
    \section{Offertory}
    \section{Canon}
    \section{Communion}
    \section{After Communion}
    }

\end{document}
