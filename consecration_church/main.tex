% !TEX program = lualatex
\documentclass[letterpaper]{report}
\title{Consecration of a Church}

% PACKAGES

\usepackage{enumitem}
\usepackage{rubrics}
\usepackage{microtype}
\usepackage{booktabs}
\usepackage{csquotes}
\usepackage{relsize,etoolbox}
\AtBeginEnvironment{quote}{\smaller}% Step font down one size relative to current font.
% \usepackage[backend=biber,style=reading,style=verbose-ibid]{biblatex}
\usepackage[backend=biber,style=reading,style=apa]{biblatex}
\addbibresource{~/Library/texmf/bibtex/bib/rubrics.bib}

\newcommand\blessincense{
\begin{quote}
   Ab illo benedicáris, in cujus honóre cremáberis. In nómine Pa\cross
    tris, et Fí\cross lii, et Spíritus \cross\ sancti. \rbar Amen.
\end{quote}
}
\newcommand\crossplan{
\begin{center}
    \begin{tabular}{ | l c r | }
       \hline
        \cross\ {\tiny 2} &         & {\tiny 5} \cross\ \\
                           & \cross &           \\
        \cross\ {\tiny 4} &         & {\tiny 3} \cross\ \\
       \hline
   \end{tabular} 
\end{center}
}

\newcommand\src{\textsc{S.R.C.}}

\begin{document}
\maketitle

\chap{Preparations}{
    
    \section{Attestation.}

    The following attestation is written or printed on a small piece of
    parchment. The parchemnt is placed in the case containing the relics the
    day before the ceremony, but the Bishop.

    \begin{quote} 
        [Year in Roman numerals]., die [date] mensis [name of month].
        Ego [consecrator's Christian name] Episcopus consecravi Ecclesiam et
        alt\'are hoc, in honorem Sancti(\textit{orum}) [name of
        saint(\textit{s})]\footnote{Name of the saint(s) in whose honor the
        altar is to be consecrated. If the church and altar have different
        Titulars, after \textit{consecravi Ecclesiam} insert \textit{hanc in
        honorem Sancti \dots N\dots} and then continue, \textit{et alt\'are hoc
        in honorem Sancti \dots N\dots}} et Reliquias Sanctorum Martyrum [names
        of the martyrs] in eo inclusi.
    \end{quote}

    \section{Charts.}

    The Rubrics prescribe that at times during the fuction the Subdeacon should
    hold before the Bishop charts containing certain formulas to be used. These
    should be printed on stiff paper and numbered according to the order in
    which they are used in the ceremony.

    \subsection*{Chart 1}

    \vbar Ut locum istum visit\'are dign\'eris.\\
    \rbar Te rog\'amus, audi nos.\\
    \vbar Ut in eo Angel\'orum cust\'odiam deput\'are dign\'eris.\\
    \rbar Te rog\'amus, audi nos.\\
    \vbar Ut Eccl\'esiam, et alt\'are hoc (\textit{alt\'aria h\ae c}), ad
    hon\'orem tuum, et nomen sancti \dots N\dots (\textit{sanct\'orum N.N.})
    consecr\'anda bene\cross d\'icere dign\'eris.\\
    \rbar Te rog\'amus, audi nos.\\
    \vbar Ut Eccl\'esiam, et alt\'are hoc (\textit{alt\'aria h\ae c}), ad
    hon\'orem tuum, et nomen sancti \dots N\dots (\textit{sanct\'orum N.N.})
    consecr\'anda bene\cross d\'icere et sancti\cross fic\'are dign\'eris.\\
    \rbar Te rog\'amus, audi nos.\\
    \vbar Ut Eccl\'esiam, et alt\'are hoc (\textit{alt\'aria h\ae c}), ad
    hon\'orem tuum, et nomen sancti \dots N\dots (\textit{sanct\'orum N.N.})
    consecr\'anda bene\cross d\'icere, sancti\cross fic\'are, et conse\cross
    cr\'are dign\'eris.\\
    \rbar Te rog\'amus, audi nos.

    If the church and the altar have different Titulars, the beginning of the
    last three petitions must be changed in this manner:

    Ut Eccl\'esiam hanc ad hon\'orem tuum et nomen sancti N.
    (\textit{sanct\'orum N.N.}), et alt\'are hoc (\textit{lat\'aria h\ae c}),
    ad hon\'orem tuum et nomen sancti N. (\textit{sanct\'orum N.N.})
    consecr\'anda, etc.

    \subsection*{Chart 2}

    Sancti\cross fic\'etur hoc alt\'are, in hon\'orem Dei omnipot\'entix, et
    glori\'os\ae\ V\'irginis Mar\'i\ae, atque \'omnium Sanct\'orum, et ad nomen
    ac mem\'oriam Sancti N. (\textit{Sanct\'orum N.N.}).\footnote{Here insert
    the name of the Titular of the altar.} In n\'omine Pa\cross tris, et
    F\'i\cross lii, et Sp\'iritus \cross\ sancti. Pax tibi.

    \subsection*{Chart 3}

    Conse\cross cr\'etur, et sancti\cross fic\'etur hoc sep\'ulchrum. In
    n\'omine Pa\cross tris, et F\'i\cross lii, et Sp\'iritus \cross\ sancti.
    Pax huic d\'omui.

    Conse\cross cr\'etur, et sancti\cross fic\'etur h\ae c t\'abula
    (\textit{vel} hic lapis) per istam uncti\'onem, et Dei bened\'ictionem. In
    n\'omine Pa\cross tris, et F\'i\cross lii, et Sp\'iritus \cross\ sancti.
    Pax tibi.

    Sign\'e\cross tur, et sancti\cross fic\'etur hoc alt\'are. In n\'omine
    Pa\cross tris, et F\'i\cross lii, et Sp\'iritus \cross\ sancti. Pax tibi.

    \subsection*{Chart 4}

    Sancti\cross fic\'etur, et conse\cross cr\'etur lapis iste. In n\'omine
    Pa\cross tris, et F\'i\cross lii, et Sp\'iritus \cross\ sancti, in
    hon\'orem Dei, et glori\'os\ae\ V\'irginis Mar\'i\ae, atque \'omnium
    Sanct\'orum, ad nomen, et mem\'oriam Sancti N. (\textit{Sanct\'orum N.
    N.}).\footnote{Here insert the name of the Titular of the \textit{altar.}}
    Pax tibi.

    \subsection*{Chart 5}

    Sancti\cross fic\'etur, et conse\cross cr\'etur hoc templum. In n\'omine
    Pa\cross tris, et F\'i\cross lii, et Sp\'iritus \cross\ sancti, in
    hon\'orem Dei, et glori\'os\ae\ V\'irginis Mar\'i\ae, atque \'omnium
    Sanct\'orum, ad nomen, et mem\'oriam Sancti N. (\textit{Sanct\'orum N.
    N.}).\footnote{Here insert the name of the Titular of the \textit{church.}}
    Pax tibi.

    \section{Articles to Be Prepared.}

    \begin{enumerate}[label=\Roman*.]

        \item On or by the altar for the Holy Relics:

            \begin{enumerate}[label=\arabic*.]

                \item A small urn or tabernacle, ornamented with red silk, in
                    which the case of the relics will be placed. A silver
                    salver with a small red cloth may substitute;
                
                \item Near the altar a bier for carrying the relics;

                \item Two candlesticks.

            \end{enumerate}

        \item On a table on the Epistle side of the altar for the Holy Relics:

            \begin{enumerate}[label=\arabic*.]

                \item A red stole for the Bishop;
                    
                \item The case for the relics;

                \item The attestation of the consecration;

                \item Three grains of incense.\footnote{The size of these
                    grains should be quite small, and it should be verified
                    beforehand that these grains and the case of the relics can
                    fit in the sepulcher.};

                \item A piece of red silk ribbon, about 18 inches long, to be
                    tied around the case of the Holy Relics;

                \item A pair of scissors;

                \item A piece of sealing-wax;

                \item The consecrator's signet;

                \item A silver salver on which the relics are placed.

            \end{enumerate}

        \item On or by a table on the Gospel side of the altar for the Holy
            Relics:

            \begin{enumerate}[label=\arabic*.]

                \item \textit{Pontificale Romanum, Pars II;}

                \item Bugia;

                \item Amice, alb, cincture, white stole and cope,
                    \textit{auriphrygiata} miter for the Bishop;

                \item White silk \textit{vimpa} for the miter-bearer;

                \item Ewer, basin and two large towels;

                \item Six or eight large candles or torches to be carried at
                    the sides of the bier during the procession;

                \item Four amices, albs, cinctures and red chasbules for the
                    priests who carry the Bier;

                \item Crosier;

                \item A faldstool covered in white;

                \item A cushion for the Bishop;

                \item Stools for the Deacon and Subdeacon;

                \item Chairs for the clergy who are to recite Matins and
                    Laudes.

            \end{enumerate}

        \item Outside the main door of the church:

            \begin{enumerate}[label=\arabic*.]
                
                \item A faldstool covered in white and a cusion on a piece of
                    carpet;

                \item A tabled covered in white, to the right side of the door

            \end{enumerate}

        \item On the table outside the main door:

            \begin{enumerate}[label=\arabic*.]
                
                \item A large vessel containing water to be blessed;

                \item A smaller vessel to hold the Gregorian water set aside
                    for the cement;

                \item A little dish with salt;

                \item An ordinary empty aspersorium with a sprinkler
                    ``\textit{ex hyssopo}'';

                \item A pitcher or ladle for conveying the holyt water from the
                    large vessel to the aspersorium;

                \item A large towel for the use of the Bishop;

                \item Four candlesticks with candles, unlit.

            \end{enumerate}

        \item Inside the church:

            \begin{enumerate}[label=\arabic*.]

                \item The church should be unfurnished; the altar(s) bare and
                    the holy-water stoups at the entrance of the  church empty;

                \item On the pavement of the church two parallel lines are
                    maked with chalk extending from the left corner of the
                    front of the church to the Epistle corner of the church
                    neat the communion-rail, and two others extending from the
                    right corner of the front of the church to the Gospel
                    corner of the church near the communion rail. The parallel
                    lines should be about 9 in. apart. Instead of these lines
                    48 chalk-marks about 8 in. square may be made on the floor,
                    i.e., 24 from the left of the entrance to the Epistle
                    corner of the church, and 24 in the other direction,
                    corresponding to the number of the letters of the Greek and
                    Latin alphabets;\footnote{It would facilitate the work of
                    the consecrator if the forms of the letters of both
                    alphabets were drawn with chalk outside the parallel lines
                    or marks.}

                \item In the middle of the church a faldstool and a cushion of
                    a piece of carpet;

                \item On the predella on the Gospel side of the altar, a
                    lectern for the sermon.

                \item A candle-lighter and matches.

            \end{enumerate}

        \item On a table covered with a white cloth, near the altar which is to
            be consecrated:

            \begin{enumerate}[label=\arabic*.]

                \item A large vessel containing water to be blessed;

                \item Empty aspersorium and an aspergillum ``\textit{ex
                    hyssopo}'';

                \item A small dish with salt;

                \item A small vessel containing finely sifted ashes;

                \item A cruet of wine on a plate;

                \item A pitcher or ladle for conveying the blessed water from
                    the large vessel to the aspersorium.

            \end{enumerate}

        \item On the credence, covered in a white cloth:

            \begin{enumerate}[label=\arabic*.]

                \item Two small oil stocks, filled with cotton, one containing
                    Oil of Catechumens, the other Holy Chrism;

                \item A cruet filled with Oil of Catechumens and a cruet filled
                    with Holy Chrism on a salver;

                \item Two thuribles and a boat filled with incense;

                \item A quantity of incense to refill the boat;

                \item A little basin containing a small quantity of cement;

                \item A small trowel;

                \item Ewer, basin, towels and a salver with cotton and alcohol
                    swabs;

                \item Five small crosses made of wax tapers (\textit{about 6
                    inches long}), with five large grains of incense at each of 
                    the ends and at the center of each of the crosses.

            \end{enumerate}

        \item Near the communion rail on the Gospel side there should be a
            table covered with a white cloth for the the relics.
            On this table there should be four candlesticks and a silver salver.


        \item In the sacristy behind the altar:

            \begin{enumerate}[label=\arabic*.]

                \item A brazier with live coals and tongs;

                \item A large porcelain or metal vessel for receiving the
                    soiled sponges, towels, cotton, etc.;

                \item Candle-lighters and matches.

            % \end{enumerate}

        % \item In the sacristy:

            % \begin{enumerate}[label=\arabic*.]

                \item All the vestments necessary for the Mass to be celebrated
                    after the consecration; the chalice;

                \item Amice, alb, cincture, white stole and small pontifical
                    for the deacon who will act as the \textit{guard} of the
                    church;

                \item Two amices, albs, and cinctures for the assistant Deacon
                    and Subdeacon and a white stole for the
                    deacon;\footnote{They do not use the tunic and dalmatic.
                    \src, May 17, 1890, n. 3729 ad III.}

                \item Surplices for the servers, assisting and visiting clergy;

                \item Ornaments of the altar: Crucifix, candelabra,
                    altar-cards, little altar bell, missal and stand, cruets
                    containing wine and water, basin and finger towel,
                    reliquaries, statues, ablution cup, key of the tabernacle
                    (if the Blessed Sacrament is to be preserved), flowers,
                    carpets, cerecloth (if necessary), three altar-cloths;

                \item Processional cross and two candlesticks for the acolytes.

            \end{enumerate}

    \end{enumerate}

    \section{Ministers Necessary for the Function.}

    \begin{enumerate}[label=\Roman*.]

        \item Two deacons and one subdeacon;

        \item Two masters of ceremonies;

        \item Cross-bearer;

        \item Two acolyes;

        \item Thurifer;

        \item Book, candle, miter and cross-bearers;

        \item Four servers for various actions during the ceremony;

        \item Eight torch-bearers for carrying the torches during the
            procession with the Holy Relics and during Mass;

        \item Four priests to carry the bier on which is placed the the urn
            containing the Holy Relics;

        \item A mason to assit with sealing the relics in the sepulcher.

    \end{enumerate}

}

\chap{Exposition of the Relics}{

    \section{The Bishop Seals the Case of the Relics.}

    \rubric This part of the ceremony takes place the evening before the
    consecration. The candles on the table in the oratory are lit.

    \rubric Assisting clergy with surplices, preceded by a server with
    holy-water, go to the entrance of the oratory to receive the Bishop, vested
    in rochet and mantelletta, with biretta. The Bishop removes his biretta at
    the entrace and receives holy water from the rector.\footcite[The bishop
    does not sprinkle the clergy, as the rubrics prescribe, if he be not the
    Ordinary. See][note 1, p. 28.]{consecranda}

    \rubric If there is an altar in the oratory, the Bishop kneels on the
    lowest step before it for a short time, then rises and goes to the table,
    dons a red stole, and places the relics into the reliquary with the three grains of
    incense and the attest of the consecration.

    \rubric After putting the lid on the reliquary, Bishop folds around it a
    red silk ribbon that passes over the top, bottom and four sides of the
    reliquary and is tied at the top. Sealing wax is dropped over the knot and
    is sealed with the consecrator's signet.\footnote{Sealing the relics in the
    case may be done by the bishop privately beforehand.}

    \rubric The Bishop carries the reliquary with both hands and places it in
    the urn, locks the urn and removes his stole. The Bishop kneels for a short
    time in prayer before the reliquaries, rises and is accompanied by the
    clergy to the door, where he departs.\footnote{The bishop may remain to
    recite the office with the clergy, though the length of the ceremony of the
    next day may discourage the bishop from doing so.}

    \section{Matins and Lauds.}

    \rubric The clergy return to the relics and recite Matins and Lauds
    \textit{De communi plurimorum martyrum,}\footcite[If the relics are of
    martyrs who have a proper office in the Breviary, e.g., Ss. Vincent and
    Anastasius, Ss. Fabian and Sevastian, etc., the proper office of these
    martyrs may be recited.][n. 21, footnote 2, p. 25.]{consecranda:1956} (of
    double rite). The lessons of the first nocturn are \textit{Fratres,
    debit\'ores,} etc.; of the second nocturn \textit{Quotiesc\'umque,
    fratres,} etc.; of the third nocturn \textit{D\'ominus ac red\'emptor.} The
    oration is \textit{Deus, qui c\'onspicis,} the word \textit{\'annua} being
    omitted, without mentioning the name of the martyrs, found in the Breviary
    on October 14.\footnote{\src August 18, 1913.} No commemorations are made
    because it is a votive office\footnote{The office \textit{Dedicationis
    Ecclesi\ae} cannot be recited on this occasion --- \src, Dec. 7, 1844, n.
    2868. Neither does the recitation of the office \textit{Sanctorum Martyrum}
    on this occasion dispense from the recitation of the current office. The
    axiom \textit{``Officium pro officio valet''} cannot here be applied. ---
    \src, Sept. 16, 1881, n. 3532 ad II.}. The psalms with the antiphons and
    versicles of this office are taken from the same \textit{De communi
    plurimorum martyrum,}\footnote{\src 2886, 4306.} not from the
    Psalter.\footcite[][n. 738.]{ml:1947}

    \rubric During the night two candles should be kept burning before the
    relics.\footnote{It is not necessary to repeat Matins and Lauds so that the
    watch be kept throughout the night --- \src, Feb. 22, 1888, n. 3686 ad
    III.} Martinucci says that in Rome two, four or six lay persons continue
    the watch until the relics are carried to the church on the following
    morning.

}

\chap{Consecration of the Church}{

    \section{Notes.}

    \rubric The Consecrator may recite or sing the Orations\footnote{The
    rubrics use the terms \textit{dicit, dicens}.}; if he sings them he used
    the \textit{tonus ferialis.} The same rule applies for the Preface.

    \rubric The antiphons, responsories, psalms\footnote{Except the seven
    Penitential Psalms at the beginning of the function, which are recited
    \textit{recto tono}.} and Litany ought to be sung\footnote{The rubrics say
    \textit{cantat, cantant, schola seu ministris prosequentibus}.} unless
    \textit{ob defunctum cantorum} it is impssible to do so, in which case they
    should be recited \textit{recto tono.} 

    \rubric The cross-bearer and acolytes should stand opposite to the
    Consecrator, except when they lead the procession and when the Consecrator
    is engaged at the altar, in which case they stand \textit{in plano} on the
    Gospel side; they always accompany the Consecrator when moving from one
    altar to another.

    \rubric The book an candle bearers usually stand at the left of the
    consecrator, when the latter recites from the Pontifical. The candle-bearer
    should stand at the right of he book-bearer. When moving from place to plae
    the walk behind the miter and crosier bearers.

    \rubric The miter and crosier bearers always stand or move behind the
    Consecrator.

    \section{Lighting the Twelve Crosses.}

    \rubric The function ought to be begun at an early hour. All who are to
    take part in the ceremony enter the sacristy or church. Subdeacon and
    Deacon vest in amice, alb and cincture, the Deacon dons a white stole. The
    Deacon\footnote{Referred to in these notes as the ``Guard''.} who is to
    guard the church vests in the same manner as the officiating Deacon.

    \rubric The laity are not permitted into the church, and all doors but the
    main doors are locked. The Deacon (to the Gospel side) and the Subdeacon
    (to the Epistle side) flank the faldstool, which stands in the middle of
    the church, facing the main doors. The cross-bearer and acolytes stand to
    the right of the Deacon, facing the doors. Assisting clergy in surplice and
    the Guard go to the entrance and receive the Bishop, in the mantelletta,
    who is lead immediately to the faldstool. The Bishop sits and orders the
    twelve crosses to be lighted. The clergy\footnote{\textit{Clergy,} in these
    notes, refers not to the visiting clergy, but to the clergy of the ceremony
    and the altar-boys together.} arrange themselves in rows behind and on the
    sides of the faldstool.

    \rubric As soon as the candles have been lit all go to the place where the
    relics were kept overnight:

    \begin{enumerate}

        \item Cross-bearer and acolytes;

        \item Chanters\footnote{If the chanters are not vested in cassock and
            surplice they precede the cross-bearer. See \cite{consecranda} Note
            2, p. 31.} and clergy, two by two;

        \item Bishop between Deacon and Subdeacon;

        \item The Bishop's attendants

    \end{enumerate}

    As soon as the procession has left the church the main doors are locked and
    only the Gaurd is left in the church.

    \section{Vesting the Bishop.}

    \rubric Upon arrival, cross-bearer and acolytes stand to the Gospel side of
    the table, facing the Epistle side; the clerics and priests stand in rows
    before the table. The Bishop and his minsters
    go to the table and kneel before the relics a short time in prayer; all
    kneel with them.

    \rubric MC1 signals all to rise; Bishop goes to faldstool and sits down
    with biretta, facing the door of the chapel, Deacon and Subdeacon flanking.
    Once the Bishop is seated MC2 distributes vestments to the servers who will
    vest the Bishop; the servers approach the faldstool and stand a short
    distance from it. The book and candle bearers take Pontifical and candle
    and stand behind the Bishop and to his left. Servers stand by ready with
    ewer, basin and towel.

    \rubric MC1 signals the Bishop to remove his biretta, rise and turn towards
    the relics; Deacon and Subdeacon switch sides. The Bishop \textit{reads} in
    a loud tone \textit{Ne reminisc\'aris}, after which chanters and clergy
    slowly \textit{recite} the seven Pentitential Psalms.

    \rubric The Bishop sits immediately after reading \textit{Ne
    reminisc\'aris.} The Deacon removes the Bishop's pectoral cross and
    mantelletta. The Bishop dons biretta and his hands are washed, servers
    standing.\footcite[See][note 1, p. 39.]{consecranda}

    \rubric After the hand-washing, the Deacon takes the Bishop's biretta and
    the vesters approach the Bishop, who stands. The Deacon and the Subdeacon
    vest the Bishop in amice, alb, cincture, pectoral cross, white stole and
    cope. The Deacon imposes miter\footnote{It seems that Bishop should be
    sitting at this time, since he does not normally receive the miter at the
    faldstool while standing.} and the Bishop receives crosier. The Bishop sits
    with the Deacon and the Subdeacon and they read alternately\footcite[The
    bishop usually reads the first verse of each psalm and recites the verse
    \textit{Gl\'oria Patri.}][note 3, p. 39.]{consecranda} the seven
    penitential psalms. Once the chanters have finished the seventh psalm, the
    Bishop alone repeats \textit{Ne reminisc\'aris.}

    \rubric After the antiphon has been repeated, all go to the main doors of
    the church:

    \begin{enumerate}

        \item Cross-bearer and acolytes;

        \item Chanters, two by two;\footnote{If the chanters are not vested in
            cassock and surplice they precede the cross-bearer.}

        \item Clergy two by two;

        \item Bishop between Deacon and Subdeacon;

        \item Miter and crosier, book (with \textit{Pontificale}) and candle
            bearers.

    \end{enumerate}

    Cross-bearer and acolytes stand at the right side of the entrance so that
    the Bishop can see the cross\footnote{If the table for blessing the water
    is to the right side of the entrance, and slightly away from the front wall
    away fron the church (enough distance for the procession to go between the
    wall and the table) then the cross-bearer and acolytes can stand before the
    table and against the wall.}; clergy and clerics stand in semicircular
    rows; the Bishop stands with ministers in front of the door, facing it.

    \rubric Crosier and miter away. Bishop intones \textit{Ad\'esto Deus unus}
    (these three words only) which the chanters continue to the end. Bishop
    afterwards says the oration \textit{Acti\'ones nostras}. After the oration
    Bishop receives miter and kneels before the faldstool.\footnote{This
    faldstool is brought by the sacristan from the oratory after the bishop has
    vested. After placing the faldstool before the door, the sacristan can
    check that the table for the blessing of the holy water is prepared, and
    verify that any obstacles in the way of the procession around the church
    are removed.} All kneel at their places during the singing of the litany,
    which the chanters begin once Bishop kneels. The litany is not doubled and
    \textit{all present} sing the responses. All rise after the response
    \textit{Ex\'audi nos D\'omine} after the second \textit{Prop\'itius esto.}

    \section{Blessing of the Water.}

    \rubric The Bishop receives the crosier and goes to the large table
    accompanied by ministers, miter, crosier, book and candle bearers. Bishop
    stands in such a way that he faces the cross, to which he bows as often as
    he says \textit{Oremus} or the Holy Name.

    \rubric The Bishop exorcises the salt. The Bishop then gives crosier away,
    Deacon removes miter and blesses the salt with his hands joined.

    \rubric The Bishop receives miter and crosier and exorcises the water.

    \rubric Crosier away; the Deacon removes miter and Bishop blesses the water
    with joined hands.

    \rubric The Bishop takes a handful of the blessed salt and drops it three
    times in the form of a cross into the water saying \textit{Commíxtio salis
    et aqu\ae\ páriter fiat: In nómine Pa\cross tris et Fí\cross lii, et
    Spíritus \cross\ sancti. \rbar Amen.}

    \rubric The Deacon hands a towel to the Bishop; afterwards the Bishop says
    with folded hands the oration \textit{Deus, invíct\ae.}

    \section{Blessing of the Exterior of the Church.}

    \rubric The Bishop resumes the miter and returns with his ministers to the
    main entrance where he stands facing the door. The server in charge of the
    holy water transfers some of the water into the aspersorium and then stands
    to the right of the Deacon. The Deacon removes miter and hands aspergillum
    to the Bishop, who intones \textit{Asp\'erges me}, which is continued by
    the chanters. The Bishop sprinkles himself, the clergy and the bystanders.

    \rubric The cross-bearer and acolytes stand to the right of the Bishop, who
    resumes miter. The cross-bearer and acolytes precede the Bishop and
    ministers around the outside of the church, Epistle side first. The Bishop
    sprinkles in the form of a cross the \textit{upper part of the walls} of
    the church\footcite[The procession walks around and Bishop sprinkles the
    cemetery also if it be adjacent.][n. 41, p. 44.]{consecranda} saying
    continually: \textit{In nómine Pa\cross tris, et Fí\cross lii, et Spíritus
    \cross\ sancti,} not adding the word \textit{Amen}. The holy water bearer
    walks to the right of the Deacon and the people may follow the
    Bishop.\footnote{If it is impossible to go around the church, he sprinkles
    the wall at his right as far as he can, then, passing by the front of the
    church, he goes to the place at the other side of the church nearest the
    point at which he stoped on the right side and begins to sprinkle the walls
    from that point until he arrives at the main portal. \src 1321 ad 1.} The
    chanters finish the antiphon and sing the responsory \textit{Asp\'erges
    me}, which is sung while Bishop sprinkles the outside of the church.

    \rubric Having arrived back at the main entrance, the Bishop hands the
    aspergillum to the Deacon, who passes it to the water carrier.\footnote{The
    water carrier ought to refill at this time.} The Deacon removes the miter;
    the cross-bearer and the acolytes stand at the right of the Bishop. When
    the chanters have finished, the Bishop faces the door of the church says
    \textit{Orémus,} the Deacon adds \textit{Flectámus génua,} and all make a
    simple genuflection.\footnote{Except the cross-bearer, acolytes and book
    and candle bearers.} Then\footnote{The subdeacon does not say
    \textit{Leváte} until everyone (or at least the bishop and those close to
    him) achieved a genuflecting position.} the Subdeacon says \textit{Leváte}
    and all rise. The Bishop says the oration \textit{Omnípotens sempitérne
    Deus.}

    \rubric The Bishop resumes the miter and receives the crosier, strikes the
    door \textit{once} with the lower end of the crosier and says:

    \begin{quote}
        Attóllite portas príncipes vestras, et elevámini port\ae\ \ae ternáles: et
        introíbit Rex glóri\ae.
    \end{quote}

    The Guard from the inside of the church\footcite[The door may be slightly
    open, so that the bishop and the deacon of the church can hear each
    other.][note 1, p. 46.]{consecranda} says in a loud tone:

    \begin{quote}
       Quis est iste Rex Glóri\ae? 
    \end{quote}

    Bishop answers in the same tone:

    \begin{quote}
        Dóminus fortis, et potens: Dóminus potens in pr\'\ae lio.
    \end{quote}

    \rubric The Bishop gives away crosier, receives sprinkler from Deacon and
    goes around the church a second time, the same manner as before, but this
    time sprinkling the \textit{lower part of the walls} of the church. As soon
    as Bishop begins this second sprinkling chanters sing the responsory
    \textit{Benedic, Domine.} Having arrived back at the main entrance, the
    Bishop hands the aspergillum to the Deacon, who passes it to the water
    carrier. The Deacon removes the miter; the cross-bearer and the acolytes
    stand at the right of the Bishop. When the chanters have finished, the
    Bishop faces the door of the church says \textit{Orémus,} the Deacon adds
    \textit{Flectámus génua,} and all make a simple genuflection. Then the
    Subdeacon says \textit{Leváte} and all rise. The Bishop says the oration
    \textit{Omnípotens sempitérne Deus.}

    \rubric The Bishop resumes the miter and receives the crosier, strikes the
    door \textit{once} with the lower end of the crosier and says:

    \begin{quote}
        Attóllite portas príncipes vestras, et elevámini port\ae\ \ae ternáles:
        et introíbit Rex glóri\ae.
    \end{quote}

    The Guard from the inside of the church says in a loud tone:

    \begin{quote}
       Quis est iste Rex Glóri\ae? 
    \end{quote}

    Bishop answers in the same tone:

    \begin{quote}
        Dóminus fortis, et potens: Dóminus potens in pr\'\ae lio.
    \end{quote}

    \rubric The Bishop returns to the main entrance and repeats the ceremonies performed
    after his first circuit.

    \rubric The Bishop strikes the door as before and repeats the ceremonies as before.

    \rubric For a third time, Bishop walks around the church, this time
    starting on the \textit{left} side, sprinkling the wall at level of his
    face, chanters singing the responsory \textit{Tu Domine.} Having arrived
    back at the main entrance, the Bishop hands the aspergillum to the Deacon,
    who passes it to the water carrier. The Deacon removes the miter; the
    cross-bearer and the acolytes stand at the right of the Bishop. When the
    chanters have finished, the Bishop faces the door of the church says
    \textit{Orémus,} the Deacon adds \textit{Flectámus génua,} and all make a
    simple genuflection. Then the Subdeacon says \textit{Leváte} and all rise.
    The Bishop says the oration \textit{Omnípotens sempitérne Deus.}

    \rubric The Bishop resumes the miter and receives the crosier, strikes the
    door \textit{once} with the lower end of the crosier and says:

    \begin{quote}
        Attóllite portas príncipes vestras, et elevámini port\ae\ \ae ternáles: et
        introíbit Rex glóri\ae.
    \end{quote}

    The Guard from the inside of the church says in a loud tone:

    \begin{quote}
       Quis est iste Rex Glóri\ae? 
    \end{quote}

    The Bishop and all clergy answer:

    \begin{quote}
        Dóminus virtútum ipse est Rex glóri\ae. Aperíte. Aperíte. Aperíte.
    \end{quote}

    \rubric The Bishop makes with the lower part of the crosier the sign of the
    cross on the threshold, saying \textit{Ecce cru\cross is signum, f\'ugiant
    phant\'asmata cuncta.} The door is then opened by the Guard. The Bishop,
    Deacon and Subdeacon enter the church and Bishop says in a loud tone
    \textit{Pax huic d\'omui,} to which the Guard answers \textit{In introítu
    vestro,} and all the clergy answer \textit{Amen.}

    \rubric The Guard goes to the sacristy and devests. The Bishop moves to the
    right so that cross-bearer and acolytes, chanters, assistant clerics and
    mason (if required) enter the church, after which the church is closed and
    locked.\footnote{The visiting clergy and the people remain outside the
    church.} All in the church move in procession, cross-bearer and acolytes in
    the lead, Bishop and ministers in the rear, followed by the book, candle,
    miter and crosier bearers, to the faldstool which stands in the middle of
    the church. During the procession the chanters sing the antiphons
    \textit{Pax \ae t\'erna} and \textit{Zach\'\ae e fest\'inans.}

    \section{Blessing of the Interior of the Church.}

    \rubric The cross-bearer and acolytes stand at the Epistle side of the
    faldstool, the Bishop with his ministers stands
    before the faldstool, facing the high altar, and all the others stand near
    the Bishop. Upon the completion of the antiphons, the Bishop gives the
    crosier to crosier-bearer and Deacon removes miter and zucchetto. MC1
    signals all to kneel on both knees and Bishop intones the \textit{Veni
    Cre\'ator Sp\'iritus}, which is continued by the chanters. All rise at the
    end of the first strophe, the Deacon replaces the zucchetto, and all remain
    standing until the end of the hymn.

    \rubric At the beginning of the second strophe several assistants sprinkle
    the floor with ashes in which the letters of the Latin and Greek alphabets
    are to be delineated.\footnote{Beforehand forty-eight squares are marked
    with chalk on the floor. The ashes are sprinkled in these squares only,
    just enough to allow the letters to be distinctly formed.}

    \rubric After the singing of the \textit{Veni Cre\'ator Sp\'iritus,} the
    Bishop resumes miter and kneels before the faldstool. All also
    kneel\footnote{Except for, of course, the cross-bearer and acolytes, and
    the servers sprinkling the ash.} The chanters begin the litany, doubling
    the name (names) of the saint (saints) in whose honor the church and the
    altar (altars) is consecrated and the names of the martyrs whose relics are
    to be placed in the altar.\footcite[If the name of any of these saints is
    inscried in the litany, e.g., St. Peter, St. Stephen, it is invoked a
    second time immediately after the proper invocation in the Litany. If the
    name is not in the Litany, it is inserted after the \textit{individual}
    invocations of the saints of the same category, but before the general
    invocation, e.g., if the church is consecrated in honor of St. Francis de
    Sales, his name is inserted twice after \textit{Sancte Nicol\ae} and before
    \textit{Omnes sancti Pontifices et Confessores.} If the church or altar is
    consecrated in honor of the Blessed Virgin under any title whatever the
    invocation \textit{Sancta Maria} is mentioned twice; if in honor of a
    mystery of the Life and Passion of Our Lord, the petition \textit{Fili
    Redemptor mundi Deus} is repeated.][Note, p. 51.]{consecranda}

    \rubric The MC2 gets Chart 1. After the chanters have sung \textit{Ut
    \'omnibus fid\'elibus def\'unctis,} etc., and those present have answered
    \textit{Te rog\'amus audi nos} the Bishop, the Deacon, the Subdeacon and
    crosier-bearer rise. The Bishop receives the crosier and recites, using
    Chart 1, held by the Subdeacon, alternately with \textit{all present}:

    \begin{quote}
        \vbar Ut locum istum visitáre dignéris.\\
        \rbar Te rogámus audi nos.\\
        \vbar Ut in eo Angelórum custódiam deputáre dignéris.\\
        \rbar Te rogámus audi nos.\\
    \end{quote}

    The Bishop then raises his right hand and blesses the altar (altars) and
    church conjointly by making the sign of the cross towards the altar as
    often as indicated in the invocations that follow. The Deacon raises the
    border of the Bishop's cope whenever Bishop makes the sign of the cross.
    After the petitions, the Bishop gives away the crosier and he and all who
    rose with him kneel and the Litany is sung to the last \textit{Kyrie
    el\'eison.}

    \rubric After the Litany, all rise; the Deacon removes the Bishop's miter.
    \textit{Or\'emus. Flectámus génua. Leváte.} The Bishop facing the altar
    says the two orations \textit{Pr\ae véniat nos} and \textit{Magnificáre.}

    \rubric The Bishop resumes miter and receives the crosier. Cross-bearer and
    acolytes lead the Bishop and his ministers to the corner at the Gospel side
    of the main entrance, the Bishop facing the Gospel side wall. With the
    lower extremity of the crosier, which is held in both hands, the Bishop
    delineates the letters of the Greek alphabet in the ashes spread on the
    floor from this corner to the Epistle corner of the church near the altar.
    Then he goes to the corner at the Epistle side of the main entrance, again
    facing the Gospel side wall, and delineates in the same manner the letters
    of the Latin alphabet in the ashes spread on the floor from this corner to
    the Gospel corner of the church near the altar.

    \rubric The cross-bearer and acolytes stand opposite Bishop and move with him
    along the line. As soon as the Bishop begins the Greek alphabet the
    chanters sing the \textit{Canticle of Zachary,} the verses alternating with
    antiphons. The chanters take care that the antiphon after the \textit{Sicut
    erat} will be sung when the bishop is writing the last letters of the Latin
    alphabet. When the cantor notices that Bishop is approaching the last
    letters of the Latin alphabet, he instructs the chanters to break off
    singing the canticle and to begin the \textit{Gl\'oria
    Patri.}\footcite[][p. 56.]{consecranda}

    \rubric When the Bishop has finished writing the Latin alphabet he goes
    with his assistants, preceded by cross-bearer and acolytes, to the altar
    which is to be consecrated, and, standing a few paces from it, lays aside
    the crosier and miter. He then kneels, all others kneeling with him, facing
    the altar, and says:

    \begin{quote}
        \vbar Deus in adjutórium meum inténde.
    \end{quote}

    Bishop and all the others rise and the chanters say:

    \begin{quote}
        \rbar Dómine ad adjuvándum me festína.
    \end{quote}

    Bishop standing in the same place and without miter says:

    \begin{quote}
        \vbar Glória Patri, et Fílio, et Spirítui sancto.
    \end{quote}

    To which the chanters answer:

    \begin{quote}
        \rbar Sicut erat in princípio, et nunc, et semper, et in s\'\ae cula s\ae
        culórum. Amen.
    \end{quote}

    This ceremony is performed three times, the Bishop beginnig in a higher
    tone each time, the chanters answering each time in the same tone as the
    Bishop. All present kneel and rise with the Bishop. 

    \section{Blessing of the Gregorian Water.}

    \rubric After the \textit{Sicut erat} has been recited the third time the
    Bishop resumes the miter and receives the crosier, and goes with his
    assistants, preceded by cross-bearer and acolytes, to the table on which
    the water, ashes, salt and wine are kept. The cross-bearer and acolytes
    stand on the opposite side of the table as the Bishop, facing him. The
    Bishop retains the miter and crosier and exorcises the salt. Bishop then
    lays aide the crosier, miter is removed and he blesses the salt.

    \rubric The Bishop resumes miter and crosier and exorcises the water; then
    he lays aside the crosier and miter and blesses the water. 

    \rubric Without miter or crosier, the Bishop blesses the ashes.

    \rubric With his right hand he takes a handful of the salt and drops it on
    the ashes three times in the form of a cross, saying:

    \begin{quote}
       Commíxtio salis et cíneris páriter fiat. In nómine Pa\cross tris, et
        Fí\cross lii et Spíritus\cross sancti. \\
       \rbar Amen.
    \end{quote}

    The Bishop mixes the salt and ashes with his right hand. The Bishop then
    takes a handful of this mixture and drops it into the water \textit{one}
    time\footcite[The Pontifical seems to indicate that the bishop performs
    this ceremony \textit{three times,} but the authors imply that it is done
    only \textit{once.}][footnote 1, p. 60.]{consecranda} in the form of a
    cross saying:

    \begin{quote}
        Commíxtio salis, cíneris, et aqu\ae\ páriter fiat. In nómine Pa\cross
        tris, et Fí\cross lii et Spíritus\cross sancti. \\
        \rbar Amen.
    \end{quote}

    The Deacon hands the Bishop a towel with which he wipes his hand.

    \rubric The Bishop then without miter or crosier, blesses the wine, saying
    the oration \textit{D\'omine Jesu Christe.} He then takes the cruet of wine
    in his right hand and pours the wine into the water three times in the form
    of a cross saying:

    \begin{quote}
        Commíxtio vini, salis, cíneris, et aqu\ae\ páriter fiat. In nómine
        Pa\cross tris, et Fí\cross lii et Spíritus\cross sancti. \\
        \rbar Amen.
    \end{quote}

    Bishop then recites the oration \textit{Omn\'ipotens sempit\'erne Deus.}

    \rubric The Bishop resumes the miter and says the prayer
    \textit{Santcti\cross ficáre} over the water.

    \rubric The Bishop receives crosier and goes the the main door of the
    church accompanied by Deacon and Subdeacon and predeced by cross-bearer and
    acolytes. With his right hand,\footcite[][p. 291.]{moretti:4} using the
    lower extremity of the crosier, Bishop makes a cross on the inner side of
    the Gospel side door\footcite[][p. 291]{moretti:4} on the upper part and
    another cross on the lower part. He then gives away the crosier and,
    retaining the miter and facing the door, he says the prayer \textit{Sit
    pósitis crux.}

    \rubric The Bishop receives crosier and returns to the table on which the
    water was blessed. Bishop gives the crosier to the crosier-bearer and with
    the miter, facing the altar, says the invitatory \textit{Deum Patrem
    omnipoténtem.}

    While the Bishop recites this invitatory a server fills the aspersorium
    with the Gregorian water and goes to the Epistle corner of the altar to be
    consecrated. He takes a towel with him as well. MC2 takes Chart 2 to the
    Gospel corner of the altar.

}

\chap{Consecration of the Altar}{

    \section{Sprinkling of the Altar.}

    \rubric The Bishop takes crosier and goes to the altar which is to be
    consecrated. At the foot of the altar he gives away his crosier and ascends
    the predella where he intones the antiphon \textit{Intro\'ibo ad alt\'are
    Dei}(these four words only), which is continued by the chanters. After the
    antiphon the chanters sing the psalm \textit{Judica me Deus,} and
    \textit{if necessary} they repeat the antiphon \textit{Intro\'ibo} after
    each verse of this psalm.\footcite[The rubric after the first verse of this
    psalm says that the antiphon should be repeated after each verse \textit{si
    necesse fuerit.} The necessity would arise if \textit{several} altars were
    bering consecrated at the same time for the object of the repetition seems
    to be that the chant should continue as long as the function. If only
    \textit{one} altar is consecrated, the antiphon is repeated only at the end
    of the psalm instead of the \textit{Gl\'oria Patri,} etc., \textit{which is
    omitted.} The same is the case with the psalms that follow.][footnote 1, p.
    64.]{consecranda} The \textit{Gl\'oria Patri} is omitted and the antiphon
    \textit{Intro\'ibo} is repeated in its place.

    \rubric After intoning \textit{Intro\'ibo} the Bishop dips the thumb of his
    right hand into the Gregorian water and makes with it the sign of the cross on
    the table of the altar in the middle when he pronounces the word
    \textit{Sanctific\'etur}, and then raising his right hand he makes the sign of
    the cross over the pace which he signed with his thumb three times at the end:

    \begin{quote}
        Sancti\cross ficétur hoc altáre, in honórem Dei omnipoténtis, et
        gloriós\ae\ Vírginis Marí\ae, atque ómnium Sanctórum, et ad nomen ac
        memóriam Sancti N. (\textit{Santórum N.N.}). In nómine Pa\cross tris, et
        Fí\cross lii, et Spíritus \cross\ sancti. Pax tibi. 
    \end{quote}

    \rubric The Deacon receives the aspersorium from the server and holds it before
    the Bishop, and the Subdeacon holds before the Bishop Chart 2 which contains
    the above formula. Having performed this rite  on the center of the mensa, the
    Bishop repeats it with the same ceremonies and the same formula successively at
    the posterior corner of the Gospel side, the anterior corner of the Epistle
    side, the anterior corner of the Gospel side and the posterior corner of the
    Epistle side.

    \crossplan

    The Deacon receives the towel from the server and hands it to the Bishop who
    dries his thumb with it, after which the Bishop hands towel back to the Deacon
    who hands it and the aspersorium to the server. MC2 receives Chart 2 from
    the Subdeacon.

    \rubric The Bishop goes to the middle of the altar and remains standing on the
    predella. When the chanters have repeated the antiphons after the psalm, miter
    removed. \textit{Oremus; Flectámus genua; Leváte.} Afterwards Bishop says the
    prayer \textit{Singuláre.}

    During this prayer the server carries the towel to the table and refills the
    aspersorium with the Gregorian water and returns to the right of the Deacon.
    MC2 carries Cart 2 to the altar.

    \rubric The Bishop intones the antiphon \textit{Asp\'erges me} (these two
    words only) which the chanters continue to the end, after which they sing
    the first three verses of the psalm \textit{Miser\'ere.} After intoning,
    the Bishop resumes miter, receives the aspergillum from the Deacon and
    sprinkles the support and mensa conjointly. The Bishop begins in front at
    the middle of the altar, proceeds to the Epistle side, passes behind the
    altar and returns by the Gospel side to the middle at the front of the
    altar. The Bishop is accompanied by the Deacon, the Subdeacon and the
    server with aspersorium.\footcite[If the back part of the altar is attached
    to the wall, so that the Bishop cannot go around it, he sprinkles only the
    \textit{base} of the altar when passing from the middle to the Epistle
    corner, then the Epistle side of the altar, afterwards the table of the
    altar from the Epistle corner to the Gospel corner, then the Gospel side of
    the altar and finally the \textit{base} in front of the altar from the
    Gospel corner to the middle.][footnote, p. 68.]{consecranda}

    \textsc{Note. ---} If the support at any part of the altar consists only of
    columns, then only the columns are sprinkled.\footcite[][p. 68.]{consecranda}

    \rubric Having arrived at the center of the altar, the Bishop waits until
    the chanters have sung the third verse of the \textit{Miser\'ere,} after
    which he intones a second time the antiphon \textit{Asp\'erges me} and
    sprinkles the altar as before, while the chanters sing the antiphon and the
    next three verses of the \textit{Miser\'ere.}

    \rubric The altar is sprinkled seven times in this manner. Before each
    sprinkling the Bishop intones the \textit{Asp\'erges me,} which the
    chanters continue, after which they sing three verses of the psalm
    \textit{Miser\'ere,} except at the seventh sprinkling, when only two verses
    are sung.

    \begin{tabular}{l l}
        \toprule
        Sprinkling & Beginning of the three verses \\
        \midrule
        First   & \textit{Miserére mei Deus} \\
        Second  & \textit{Quóniam iniquitátem meam} \\
        Third   & \textit{Ecce enim vertátem} \\
        Fourth  & \textit{Avérte fáciem tuam} \\
        Fifth   & \textit{Redde mihi l\ae títiam} \\
        Sixth   & \textit{Dómine, lábia mea apéries} \\
        Seventh & \textit{Benígne fac Dómine} \\
        \bottomrule
    \end{tabular}

    The \textit{Gl\'oria Patri} is not sung.

    \section{Sprinkling of the Walls and Floor.}

    \rubric At the middle of the altar, after the seventh sprinkling, Bishop
    descends from the predella to the foot of the altar. Proceeded by
    cross-bearer and acolytes, accompanied by the Deacon, the Subdeacon and a
    server with the aspersorium, the Bishop goes behind the altar. Once the
    chanters begin the \textit{H\ae c est domus} Bishop begins to sprinkle the
    lower part of the wall, commencing at the middle behind the altar and then,
    preceded by the cross-bearer and acolytes, he passes down the
    \textit{Gospel} side and returns by the \textit{Epistle} side and finishes
    at the point behind the altar where he began. Chanters meanwhile sing
    \textit{H\ae c est domus} and the psalm \textit{L\ae tátus sum}; no
    \textit{Glória Patri} and the antiphon is not repeated.

    \rubric As soon as the chanters have finished \textit{L\ae t\'atus sum}
    they begin to sing the antiphon \textit{Exs\'urgat Deus} and the psalm
    \textit{In eccl\'esiis,} Bishop goes around the inside of the church a
    second time in the manner described above and sprinkles the midle
    part\footnote{\textit{circa altit\'udinem faci\'ei su\ae}} of the wall. No
    \textit{Glória Patri} after the psalm and the antiphon is not repeated.

    \rubric As soon as the chanters have finished \textit{In eccl\'esiis} they
    sing the antiphon \textit{Qui h\'abitat} and the psalm \textit{Dict
    D\'omino,} the Bishop goes around the inside of the church beginning in the
    middle behind the altar, passes this time down the \textit{Epistle} side
    and returns by the \textit{Gospel} side, and finishes at the point behind
    the altar where he began. He sprinkles the upper part of the wall. No
    \textit{Glória Patri} after the psalm and the antiphon is not repeated.

    \rubric The Bishop returns to the foot of the altar. When chanters intone
    the antiphon \textit{Domus mea} the Bishop, preceded by cross-bearer and
    acolytes and accompanied by the Deacon, the Subdeacon and the server with
    the Gregorian water, goes from the altar to the main door of the church
    sprinkling the floor; then he goes to the middle of the wall at the Gospel
    side and passes in a direct line to the middle of the wall on the Epistle
    side likewise sprinkling the floor; after which the Bishop goes to the
    center of the church.

    \rubric Having reached the center of the church Bishop stands facing the
    altar and when the chanters have finished the antiphon \textit{Non est
    hic,} the Bishop intones the antiphon \textit{Vidit Jacob scalam} (these
    three words only), which the chanters sing to the end. Having intoned this
    antiphon he sprinkles the floor towards the East, the West, the North and
    the South.\footcite[The Rubric in the \textit{Pontificale Romanum} supposes
    the altar to be towards the East; the bishop, therefore, sprinkles the
    floor before him, behind him, at his left and than at his right.][footnote
    1, p. 74.]{consecranda}

    \rubric After the antiphon the miter is removed; the Bishop turns towards
    the door of the church. \textit{Orémus; Flectámus génua; Leváte.} The
    Bishop says the prayer \textit{Deus, qui loca.}

    \rubric The Bishop still facing the door: \textit{Orémus; Flectámus génua;
    Leváte.} The Bishop says the prayer \textit{Deus sanctificatiónum},
    inserting the name of the saint (saints) whose name the church will bear.

    \rubric The Bishop, facing the door, recites with hands extended at his
    breast the Preface, in which after N. he inserts the name of the saint
    (saints) whose name the church will bear.

    \textsc{Note. ---} During the Preface the priests, who will carry the holy
    relics to the church, go to the place where the relics are exposed and put
    on amice, alb, cincture and red chasuble. Torches are prepared for the
    procession and the thurifer prepares the thurible.

    The Bishop adds the conclusion in a low tone of voice, loud enough to be
    heard by those near him.

    \rubric After the Preface the Bishop resumes the miter and, accompanied by
    his assistants, goes to the foot of the altar to bless the \textit{dry}
    cement\footcite[The Pontifical prescribes that he should first mix the
    cement with the water which he previously blessed and then bless the
    mixture. This is not feasible and consequently it is better to keep
    sufficient amount of water in a separate vessel to be poured over the dry
    cement shortly before it is to be used for closing the
    sepulcher.][footnote, p. 77.]{consecranda} contained in a dish, which a
    server holds before the Bishop. Miter removed, Bishop reads the oration
    \textit{Summe Deus.}

    \rubric After this the Gregorian water, except that which is reserved for
    preparing the cement, is poured by one of the servers around the base of
    the altar.\footcite[If much water remains only a small quantity of it is
    poured around the base of the altar, and what is left is afterwards poured
    into the \textit{secrarium.}][footnote 1, p. 78.]{consecranda}

    \section{Bringing of the Holy Relics to the Church.}

    \rubric The Bishop resumes miter and crosier and all go in procession to
    the place where the relics are exposed, in the following order:

    \begin{enumerate}
        \item Cross-bearer and acolytes;
        \item Chanters\footnote{If the chanters are not vested in cassock and
            suplice they precede the cross-bearer.}
        \item Clergy, two by two;
        \item Server carrying the vase of Chrism and some absorbent cotton on a
            salver;\footnote{Cotton sheets (woven and unwoven) are superior to
            cotton balls in this case.}
        \item Bishop, between Deacon and Subdeacon;
        \item Miter, crosier, book an candle-bearers.
    \end{enumerate}

    The procession passes through the main entrace of the church. The server
    carrying the Chrism remains at the door. A server lights the candles on the
    table outside the main entrance and on the table inside the church on which
    the relics are to be placed. All others go to the oratory where the relics
    are exposed. The Priests of the Bier repair to the sacristy and vest in
    amice, alb, cincture and chasuble.\footnote{\src\ 3365 ad V.}

    \rubric All remain standing outside the oratory. The Bishop gives crosier
    to the crosier-bearer and his miter is removed. \textit{Orémus; Flectámus
    génua; Leváte.} The Bishop recites \textit{Aufer a nobis.}

    \rubric The Bishop resumes miter and crosier. The procession moves into the
    oratory, while the chanters sing several antiphons and the psalm
    \textit{Venite exult\'emus;} the \textit{Glória Patri} is not
    sung.\footnote{Instead of the prescribed antiphons and the psalm
    \textit{Ven\'ite exult\'emus,} a Responsory proper of the martyrs whose
    relics are exposed, or one taken from the \textit{Commune Plurimorum SS.
    Martyrum} may be sung.}

    \rubric After the psalm or responsory the Bishop's miter is removed.
    \textit{Orémus; Flectámus génua; Leváte.} The Bishop recites the prayer
    \textit{Fac nos, qu\'\ae sumus.}

    \rubric The Bishop resumes miter, imposes incense with the usual ceremonies.

    \blessincense

    Meanwhile the torchbearers receive their torches. The Bishop intones the
    antiphon \textit{Cum jucundit\'ate} (these two words only) and the chanters
    continue this antiphon and sing the three others that follow, while the
    relics are carried in procession to the church in the following order:

    \begin{enumerate}
        \item Cross-bearer and acolytes;
        \item Chanters;\footnote{If the chanters are not vested in cassock and
            suplice they precede the cross-bearer.}
        \item Torchbearers carrying lighted torches;
        \item Thurifer swinging the censer before the relics;
        \item Two or Four Priests vested in red chasubles carrying the relics
            on a bier;\footcite[Instead of these priets the consecrator may
            carry the relics on a salver.][footnote 1, p. 81.]{consecranda}
        \item Bishop with miter and crosier between Deacon and Subdeacon;
        \item Miter, Crosier, Book and Candle bearers;
        \item Laity.
    \end{enumerate}

    When the procession arrives at the door of the church the chanters remain
    there and continue the chanting of the antiphons, whilse all the others, in
    the order above, go around the outside of the church beginning at the right
    (Epistle) side, passing around the rear, and returning to the door by the
    left (Gospel) side, saying continually \textit{Kyrie eléison\dots}

    \rubric When the procession returns to the front of the church the
    cross-bearer and the acolytes take their places at the left side of the
    door and the relics are placed on the table prepared for them. The priests
    who carried the relics in the procession, the thurifer and the
    torchbearers, arrange themselves around this table in such a manner that
    the relics may be seen by the Bishop when sitting on the faldstool placed
    at the right side of the door. When the Bishop arrives at the door he gives
    his crosier to the crosier-bearer and sits on the faldstool.


    \rubric The bishop remains sitting while the chanters sing the responsory
    \textit{Erit mihi.}

    \rubric When this responsory is finished the miter is removed and the
    Bishop rises and turns towards the door. \textit{Orémus; Flectámus génua;
    Leváte.} The Bishop reads the prayer \textit{Domum tuam.}

    \rubric The Bishop resumes miter. The Deacon and the Subdeacon holding the
    border of his cope, the Bishop goes to the right side of the door. He dips
    his thumb into the Chrism and anoints three times the stone pillar at the
    place where the cross is sculptured,\footcite[The Pontifical says
    ''\textit{signat ostium,''} which the \src, Aug. 7, 1875, n. 3364 ad VI,
    interprets to mean the two stone or brick pillars at the sides of the
    door.][footnote 1, p. 83.]{consecranda} saying:

    \begin{quote}
       In nomine Pa\cross tris, et Fí\cross lii, et Spirítus \cross Sancti.
    \end{quote}

    He then goes to the left side of the door and in the same manner anoints three
    times with Chrism the stone pillar at the place where the cross is sculptured.
    After wiping his thumb in cotton he goes to the middle of the door says the
    prayer \textit{Porta sis.}

    \rubric The Bishop goes to the faldstool and standing intones the antiphon
    \textit{Ingred\'imini sancti Dei} (these three words only) which the
    chanters continue to the end and then add the second antiphon
    \textit{Gaudent in c\oe lis.} After his intonation, the Bishop receives his
    crosier and all enter the church in the order given above and proceed to
    the altar which is to be consecrated. The people are now allowed to enter
    the church. Cross-bearer and acolytes stand at the Gospel side, and the
    chanters on the Gospel side, and the clergy carrying torches
    in a semicircle around the altar. 

    \rubric The priests who carry the relics bring the bier close to the table
    prepared for the relics and the Deacon removes the urn and places it on the
    silver salver on the table, surrounded by four candles. The priests then
    place the bier in the sacristy behind the altar.

    \rubric At this point a sermon may be given.\footcite[The exhortation
    here spoken of, the decrees of the Council of Trent which according to the
    Pontifical are now read by the archdeacon (any priest), the address of the
    Bishop to the founder of the church and the latter's reply may be omitted.
    --- \src, May 17, 1890, n. 3729 ad VIII.][footnote 2, p. 82.]{consecranda}

    \rubric The Bishop and his assitants stand before the altar in the middle
    of the sanctuary. The Bishop lays aside his crosier and intones the
    antiphon \textit{Exult\'abunt sancti} (these two words only) which the
    chanters continue, and to which they add the psalms \textit{Cant\'ate
    D\'omino} and \textit{Laud\'ate D\'ominum.} The \textit{Gl\'oria Patri} is
    not said and the antiphon is repated after the second psalm. Meanwhile a
    server places on the altar near the center on the Epistle side the small
    stock containing the Chrism and the slab of stone which is to be placed
    over the sepulcher after the case containing the relics has been placed in
    it, and MC2 places Chart 3 (formula of consecration) on the Gospel side.

    \section{Placing of the Holy Relics in the Sepulcher.}

    \rubric While the antiphon \textit{Exsult\'abunt} is being repeated, the
    Bishop and his assitants ascend to the predella, where the Deacon removes
    the miter. After the antiphon the Bishop facing the altar says the prayer
    \textit{Deus, qui in omni loco.}

    \rubric The Bishop resumes his miter and anoints each of the four corners
    of the sepulcher on the inside, reciting while anointing each corner:

    \begin{quote}
        Conse\cross crétur et sancti\cross ficétur hoc sepúlcrum. In nómine
        Pa\cross tris, et Fí\cross lii, et Spíritus \cross\ sancti. Pax huic
        dómui.
    \end{quote}

    \textsc{Note. ---} The Bishop anoints each corner \textit{twice,} i.e.,
    once at the word \textit{Consecr\'etur} and again at the word
    \textit{Sanctific\'etur.} He then reaises his right hand and makes the sign
    of the cross three times over the corner just anointed, i.e., at the words
    \textit{Patris, F\'ilii} and \textit{Spiritus sancti.}

    \rubric Afterwards the Bishop wipes his thumb with cotton, his miter is
    removed and he turns towards the people, receding a little to the Gospel
    side but remaining on the predella. The Deacon takes the salver with the
    case of the relics to the Bishop who takes with both hands the case
    containing the relics, and turning towards the altar, places the case into
    the sepulcher while intoning the antiphon \textit{Sub alt\'are Dei} (these
    three words only) which the chanters continue to the end.

    The candles on the table of the relics are extinguished by a server and the
    four priests go to the sacristy and divest.

    In the meantime the mason mixes in a dish the blessed cement with the
    Gregorian water set aside for this purpose.

    \rubric During the antiphon \textit{Sub alt\'are Dei} the Bishop imposes
    incense and incenses the relics with three swings, bowing profoundly before
    and afterwards. The Bishop resumes his miter, takes in his left hand the
    small slab that is to cover the sepulcher, dips his right thumb into the
    Chrism and anoints bottom of the slab in the same manner as he anointed the
    crosses, saying:

    \begin{quote}
        Conse\cross crétur et sancti\cross ficétur h\ae c tábula, per istam
        untiónem et Dei benedictiónem. In nómine Pa\cross tris, et Fí\cross lii, et
        Spíritus \cross\ sancti. Pax tibi.
    \end{quote}

    \rubric The Bishop then places the slap on the altar (not over the
    sepulcher) and wipes his thumb with cotton. The dish containing the
    prepared cement is now brought to the altar and with a small trowel the
    Bishop spreads the cement over the ledge on the inside of the sepulcher.
    The Bishop now takes the slab in his hand, intones the antiphon \textit{Sub
    alt\'are Dei} and places the slab over the opening of the sepulcher. The
    chanters continue the antiphon \textit{Sub alt\'are Dei aud\'ivi,} to which
    they add, if necessary, the antiphon \textit{C\'orpora sanct\'orum.}

    \rubric As soon as the chanters have finished the antiphons the Bishop
    removes his miter and says the oration \textit{Deus, qui ex ómnium.} Miter
    resumed, the Bishop \textit{begins} to fill with cement the crevices around
    the small slab (which the mason may continues and finishes). If any cement
    remain on the altar, it is removed with a sponge or rough
    trowel. The torchbearers' torches extinguished. The Bishop dips his
    thumb in the Chrism and anoints once the top of the slab and the
    altar\footcite[Martinucci, Hartmann, Moretti, etc., say that the cross
    should be made not only on the slab, but also on the portions of the table
    of the altar near the slab.][footnote 1, p. 90.]{consecranda} in the manner
    in which he anointed the four corners of the sepulcher, saying:

    \begin{quote}
        Signé\cross tur et sancti\cross ficétur hoc altáre. In nómine Pa\cross
        tris, et Fí\cross lii, et Spíritus \cross\ sancti. Pax tibi.
    \end{quote}

    The Bishop then wipes his thumb with cotton. The Deacon gives the Chrism to the
    server and the Subdeacon gives Chart 3 to MC2.

    The Chrism is carried to the credence. The cotton used by the Bishop and
    the sponge or towel used in removing the cement from the altar, as well as
    the scrapings of the cement, are thrown into the vessel prepared for this
    purpose behind the altar.

    \section{Incensation of the Altar.}

    \rubric Incense is imposed at the altar in the usual manner, the Bishop
    saying:

    \blessincense

    The miter is removed. The Deacon hands the thurible to the Bishop, who
    intones the antiphon \textit{Stetit Angelus} (these two words only). The
    miter is resumed and the Bishop incenses the altar to the right, to the
    left, in front and over the table until the chanters have finished the
    antiphon.

    \rubric The Bishop gives the thurible to the Deacon, and, miter removed,
    standing on the predella in the middle of the altar, says the prayer
    \textit{Dirigátur orátio nostra.} The Bishop resumes the miter and goes
    with his attendants to the faldstool, placed \textit{in plano} at the
    Epistle corner of the steps of the altar, and sits.

    \rubric Two priests, vested in surplice, wipe with sponges the altar and
    its base, and then dry these places with rough towels. The sponges and
    towels are afterwards placed in a large vase behind the altar. After the
    function they are washed and the water of this washing is poured into the
    sacrarium.

    \rubric A server carrying Chart 4 containg the formula used at the
    following unctions will stand \textit{in plano} at the Gospel side, and
    another server carrying a salver with the Oil of Catechumens and some
    cotton will stand \textit{in plano} at the Epistle side. The
    priest\footnote{From now on, referred to as the \textit{incensing
    priest.}}, vested in surplice, whose duty it will be to incense the altar
    during the consecration will stand near the server with the holy oil.

    \rubric When the altar has been wiped and dried, the Bishop, still sitting,
    imposes and blesses incense in the usual manner. 

    \blessincense

    He then rises, ascends the predella, receives the thurible from the Deacon
    and, saying nothing, incenses in the form of a cross the table of the altar
    \textit{once}: 

    \begin{enumerate}
        \item in the middle,
        \item at the posterior corner of the Gospel side,
        \item at the anterior corner of the Epistle side,
        \item at the anterior corner of the Gospel side,
        \item at the posterior corner of the Epistle side.
    \end{enumerate}

    \crossplan

    \rubric The Deacon takes the thurible from the Bishop and hands it to the
    thurifer. The Bishop imposes incense in the usual manner.

    \blessincense

    The Bishop again receives the thurible from the Deacon and intones the
    responsory \textit{Dirig\'atur} (this word only), which the chanters
    continue to the end. During the chanting of this responsory the Bishop goes
    around the altar \textit{three times} incensing continuously the support
    and the table together. He begins each time at the middle, proceeds to the
    Epistle corner, then behind the altar, and, passing by the Gospel corner,
    returns to the middle in front.\footcite[If the back of the altar is
    attached to the wall, he begins at the middle, incenses the support as fas
    as the Epistle corner, afterwards the table from the Epistle to the Gospel
    side, then the side of the altar at the Gospel side, then the side of the
    altar at the Gospel corner and finally the support from the Gospel corner
    to the middle.][footnote 1, p. 96.]{consecranda}

    \rubric After the third incensation, the Incensing Priest will go the the
    predella, receive the thurible from the Bishop with the usual kisses,
    descend \textit{in planum} at the middle and when the Bishop has intoned
    the antiphon \textit{Er\'exit Jacob} he will begin to incense the altar
    with \textit{single} swings. He begins at the middle, proceeds to the
    Epistle corner, goes behind the altar and returns to the front of the altar
    by the Gospel side, walking always \textit{in plano.} This he does
    continuously, except when the Bishop uses the thurible in the course of the
    consecration.

    \section{Anointing of the Altar.}

    \rubric The Bishop, having given the thurible to the Incensing Priest,
    intones the antiphon \textit{Er\'exit Jacob} (these two words only) which
    the chanters continue and to which they add the psalm \textit{Quam
    dil\'ecta.}\footcite[][the rubric appplied to psalm 42 above applies here
    as well.]{consecranda} The \textit{Gl\'oria Patri} is not added. The
    antiphon is repeated after the psalm.

    \rubric While the Bishop is intoning the antiphon \textit{Er\'exit Jacob}
    the server at the Gospel side gives Chart 4 to the Subdeacon who holds it
    before the Bishop, and the server at the Epistle side give the
    \textit{Oleum Catechumenorum} to the Deacon. A server holding a salver with
    cotton balls stands at the right of the deacon. After intoning the antiphon
    the Bishop anoints with the \textit{Oleum Catechumenorum} the table of the
    altar in the form of a cross in the middle and at the four corners in the
    following order:

    \crossplan

    pronouncing the following form at \textit{each} unction:

    \begin{quote}
        Sancti\cross ficétur et conse\cross crétur lapis iste. In nómine
        Pa\cross tris, et Fí\cross lii, et Spíritus \cross\ sancti: in honórem
        Dei, et gloriós\ae\ Vírginis Marí\ae, atque ómnium Sanctórum: ad nomen,
        et memóriam Sancti N. (\textit{Sanctórum N.N.}). Pax tibi.
    \end{quote}

    At each unction the Bishop makes the sign of the cross with his thumb
    twice, i.e., at the words \textit{Sanctific\'etur} and
    \textit{Consecr\'etur,} then raising his right hand he blesses it three
    times, i.e., at the words \textit{Patris, F\'ilii} and \textit{Sp\'iritus
    sancti.} At the letter \textit{N.} in the formula he inserts the name
    (\textit{names}) of the saint (\textit{saints}) to whom the altar is
    dedicated. During this ceremony the Deacon and Subdeacon hold the borders
    of the Bishop's cope. After the last unction the bishop wipes his thumb
    with cotton, and the Deacon and Subdeacon give the \textit{Oleum
    Catechumenorum} and Chart 4 to the servers.

    \textsc{Note. ---} Another thurible may be prepared for the incensation,
    which is to follow, and given to the priest who incenses the altar.

    \rubric Towards the end of the psalm \textit{Quam dil\'ecta} the Bishop
    imposes incense into the thurible presented to him by the Incensing Priest,
    blessing the incense in the usual manner. As soon as the chanters have
    repeated the antiphon \textit{Er\'exit Jacob} the Incensing Priest hands
    the thurible to the Bishop. The Bishop intones the responsory
    \textit{Dirig\'atur} (this word only) and the chanters continue it to the
    end.

    In the meantime the Bishop incesnes the altar in the manner described
    above, going around the altar only \textit{once} by the Epistle side and
    returning by the Gospel side.

    \rubric Having arrived at the middle in front the Bishop gives the thurible
    to the Incensing Priest, who continues the incensation of the altar as
    before. At the end of the responsory the Bishop's miter is removed.
    \textit{Orémus; Flectámus génua; Leváte.} The Bishop recites the prayer
    \textit{Adsit, Dómine.}

    \rubric The Bishop intones the antiphon \textit{Mane surgens Jacob} (these
    three words only) which the chanters continue and to which the add the
    psalm \textit{Bonum est.} The \textit{Gl\'oria Patri} is not added and the
    antiphon is repeated. The Bishop resumes his miter after the intonation and
    anoints the altar a second time with the \textit{Oleum Catechumenorum} in
    the same manner as described above, reciting the following formula from
    Chart 4.

    \begin{quote}
        Sancti\cross ficétur et conse\cross crétur lapis iste. In nómine
        Pa\cross tris, et Fí\cross lii, et Spíritus \cross\ sancti: in honórem
        Dei, et gloriós\ae\ Vírginis Marí\ae, atque ómnium Sanctórum: ad nomen,
        et memóriam Sancti N. (\textit{Sanctórum N.N.}). Pax tibi.
    \end{quote}

    During this ceremony the Deacon and Subdeacon will hold the borders of the
    Bishop's cope. After the last unction the Bishop wipes his thumb in cotton, and
    the Deacon and Subdeacon give the \textit{Oleum Catechumenorum} and Chart 4 to
    the servers.

    \rubric Towards the end of the psalm \textit{Bonum est} the Bishop imposes
    incense in the thurible presented to him by the Incensing Priest, and
    blesses the incense in the usual manner.

    \blessincense

    As soon as the chanters have repeated the antiphon \textit{Mane surgens
    Jacob} the Incensing Priest hands the thurible to the Bishop, who intones
    the responsory \textit{Dirig\'atur} (this word only) and the chanters
    continue it to the end. In the meantime the Bishop incenses the altar in
    the manner described above, going around only \textit{once} by the Epistle
    side and returning by the Gospel side.

    \rubric Having arrived at the middle, in front, the Bishop give the
    thurible to the Incensing Priest, who continues the incensation of the
    altar as before. At the end of the responsory the Bishop's miter is
    removed. \textit{Orémus; Flectámus génua; Leváte.} The Bishop recites the
    prayers \textit{Adésto Dómine} and \textit{Omnípotens\dots altáre hoc.}

    \rubric After these prayers the server hands the vessel containing the
    Chrism to the Deacon and the Bishop intones the antiphon \textit{Unxit te
    Deus} (these three words only), which the chanters continue and to which
    they add the psalm \textit{Eructávit cor meum.}\footcite[The antiphon
    \textit{Unxit te Deus} may be repested after each verse of this
    psalm.][footnote 1, p. 104.]{consecranda} The \textit{Gl\'oria Patri} is not
    said and the antiphon is repeated.

    After intoning the antiphon the Bishop resumes his miter and then anoints
    the altar with Chrism in the manner described above, reciting the formula
    from Chart 4:

    \begin{quote}
        Sancti\cross ficétur et conse\cross crétur lapis iste. In nómine
        Pa\cross tris, et Fí\cross lii, et Spíritus \cross\ sancti: in honórem
        Dei, et gloriós\ae\ Vírginis Marí\ae, atque ómnium Sanctórum: ad nomen,
        et memóriam Sancti N. (\textit{Sanctórum N.N.}). Pax tibi.
    \end{quote}

    After the last unction the Bishop wipes his thumb with cotton, and the
    Deacon and Subdeacon give the vessel of Chrism and Chart 4 to the servers.

    \rubric Towards the end of the psalm \textit{Eruct\'avit cor meum} the Bishop
    imposes incense in the thurible presented to him by the Incensing Priest,
    blessing it in the usual manner. 

    \blessincense

    As soon as the chanters have repeated the antiphon \textit{Unxit te Deus}
    the Incensing Priest hands the thurible to the Bishop, who intones the
    responsory \textit{Dirig\'atur} (this word only), and the chanters continue
    it to the end. In the meantime the Bishop incenses the altar in the manner
    described above, going around the altar \textit{once}, passing \textit{this
    time} by the Gospel side and returning by the Epistle side.

    \rubric Having arrived at the middle, in front, the Bishop gives the
    thurible to the Incensing Priest who continues the incensation of the altar
    as before until the Bishop has anointed the twelve crosses on the inner
    walls of the church.

    \textsc{Note. ---} The server who has charge of the holy oils will carry
    the vessel of Chrism to the table and prepare the cruets containing the
    \textit{Oleum Catechumenorum} and the Chrism, and a large
    quantity of cotton and a towel.

    \rubric At the end of the responsory the Bishop's miter is removed.
    \textit{Orémus; Flectámus génua; Leváte.} The Bishop says the prayer
    \textit{Descéndat, qu\'\ae sumus.}

    \rubric The Bishop intones the antiphon \textit{Sanctificávit} (this word
    only), which the chanters continue and to which they add the psalm
    \textit{Deus noster refúgium.}\footcite[The antiphon \textit{Sanctificávit
    Dóminus} may be repeated after each verse of this psalm.][footnote 1, p.
    108.]{consecranda} The \textit{Glória Patri} is not said and the antiphon
    is repeated.

    \rubric Having intoned the antiphon \textit{Sanctificávit,} the Bishop
    resumes the miter. The Deacon receives from the cleric the cruets
    containing the Chrism and the Oil of the Catechumens which he hands to the
    Bishop. The latter takes the cruets in his right hand\footnote{The bishop's
    ring may be removed by the deacon before this anointing and received by a
    server on a salver. It is replaced after the washing of the bishop's
    hands.} (or one cruet in each hand) and pours the holy oils conjointly on
    the middle of the altar in a straight line from the Epistle side to the
    Gospel side. The Bishop gives the cruets to the Deacon, who hands them to
    the server to return them to the credence. The Bishop draws back the right
    sleeve of his cassock and alb\footnote{An elastic or rubber band may be
    used to keep the bishop's sleeve back. The deacon, subdeacon and masters of
    ceremonies must take great care to prevent the bishop's sleeves touching
    the altar. The bishop may also wear a linen gremiale for more protection
    (an amice also fulfils this purpose).} and with the palm of his right hand
    rubs spreads the holy oils over the entire surface of the altar, first
    towards the back of the altar and then towards the front of it. The Deacon
    and Subdeacon hold the borders of the Bishop's cope.

    \rubric After the unction the Bishop will wipe his hand and his ring first
    with cotton and then with the towel, after which he will adjust the sleeves
    of his cassock and his alb. The server carries the cruets back to the
    table. The Bishop, facing the altar and standing on the predella, intones
    the antiphon \textit{Ecce odor f\'ilii mei} (these four words only), which
    the chanters continue and to which they add the psalm \textit{Fundam\'enta
    ejus}. The \textit{Glória Patri} is not added and the antiphon is not
    repeated.

    While the chanters are singing the, server prepares a thurible and a boat,
    and another server prepares a salver with chrism and cotton, and Chart 5.

    \rubric After the psalm is completed, the Bishop with miter stands on the
    predella facing the altar and recited the invitatory \textit{Lápidem hunc.}
    He then intones the antiphon \textit{Lápides pretiósi} (these two words
    only), which the chanters continue and to which they add the psalm
    \textit{Lauda Jerúsalem} and the responsories \textit{H\ae c est Jerúsalem}
    and \textit{Platé\ae\ tu\ae.} The antiphon is not repeated.

    \section{Anointing of the Twelve Crosses on the Walls of the Church.}

    \rubric Having intoned the antiphon \textit{Lápides pretiósi,} the Bishop
    descends to the foot of the altar and takes the crosier in the left hand.
    Preceeded by the thurifer, the server with the Chrism, and cross-bearer and
    acolytes, the Bishop flanked by the ministers goes to the first cross. The
    cross-bearer and acolytes stand at the left of the Subdeacon, the server
    with the Chrism to the right of the Deacon. Crosier away; the Deacon at the
    Bishop's right holds the Chrism\footnote{Or he may give it to the bishop.}
    and the Subdeacon is at the Bishop's left holding Chart 5. The Bishop dips
    his right thumb into the Chrism and anoints the wall on the marked
    cross\footcite[\textit{Above} or \textit{below} the cross, if it be of
    marble or metal. See][note 2, p. 112.]{stehle} in the manner described
    above, saying:

    \begin{quote}
        Sancti\cross ficétur et conse\cross crétur hoc templum: In nómine Pa\cross
        tris, et Fí\cross lii, et Spíritus \cross\ sancti: in honórem Dei, et
        gloriós\ae\ Vírginis Marí\ae, atque ómnium Sanctórum: ad nomen, et memóriam
        Sancti N. (\textit{Sanctórum N.N.}). Pax tibi.
    \end{quote}

    He then wipes his thumb with cotton handed to him by the deacon. If the oil
    flow down the wall, the Deacon wipes it away with cotton.

    \rubric After the unction the Bishop descends and goes to the next cross
    marked on the wall towards the left of the first cross, i.e., on the
    Gospel side, proceeding towards the front of the church. The bishop then
    facing the cross on the wall which he just anointed puts incense into the
    thurible, the Deacon offering the boat, and blesses the incense in the
    cense, saying: 

    \blessincense

    The Deacon then removes the miter and hands the censer to the Bishop, who,
    having bowed with his ministers to the cross, incenses it with three
    swings.

    After the incensation all bow to the cross; the Bishop gives the thurible
    to the Deacon who hands it to the thurifer; the Bishop resumes the miter
    and the crosier, and all proceed to the second. 
    
    \rubric This ceremony is carried out at each of the twelve crosses marked
    on the wall, except that it is not necessary to put incense into the
    thurible at every cross, but it will be sufficient to renew it when
    necessary, i.e., at every third or fourth cross. The crosses are anointed
    in order, first on the Gospel side, then at the front of the church, i.e.,
    on either side of the main door, afterwards on the Epistle side, beginning
    at the front of the church and finishing at the cross opposite the Epistle
    corner of the altar or behind the altar.

    \rubric While the last cross is being anointed one server gets ready the
    grains of incense on a salver and another the aspergillum. A server takes
    away the Chrism; the cross-bearer and the acolytes go to the Gospel side of
    tha altar and stand \textit{in plano;} the thurifer carries the thurible to
    the sacristy; the Bishop and his ministers go to the foot of the altar,
    where the Bishop gives the crosier away and they ascend the predella and
    stand before the middle of the altar.

    \rubric The Incensing Priest goes to the Bishop and presents the thurible.
    The Bishop imposes incense in the usual manner, saying:

    \blessincense

    The Bishop, facing the altar, intones the antiphon \textit{\AE dificávit
    Móyses} (these two words only), which the chanters continue. After
    intoning, the bishop receives the thurible from the Incensing Priest,
    incenses\footcite[ It seems that a single swing to the middle, one to the
    left, and one to the right suffices. ][]{consecranda:1956} this time only
    the table of the altar and then gives the thurible to the priest who
    continues the incensation as before.

    \section{Burning of the Incense on the Altar.}

    \rubric After the antiphon \textit{\AE dificávit Móyses} has been sung, the
    Bishop, still wearing the miter, recites the invitatory \textit{Dei Patris
    omnipoténtis.}

    \rubric At the end of this invitatory a server carrying the grains of
    incense on a salver and another server with the holy water ascend to the
    first step below the predella; the Bishop turns towards these two servers
    and, miter removed, blesses the grains of incense, saying the oration
    \textit{Dómine Deus.} The Deacon hands the aspergillum to the Bishop, who
    sprinkles the grains in the middle, left and right. The aspersorium is
    handed back to the server, who places it on the credence table.

    \rubric The Bishop receives the miter from the Deacon and the salver
    containing the grains of incense from the server. The latter goes to the
    Epistle corner \textit{in plano,} where he will receive the salver from the
    Deacon after all the grains have been placed on the altar. A server brings
    to the altar the five crosses made of wax tapers, and stands at the right
    of the deacon.

    \rubric The Bishop, having turned to the altar, takes the grains of incense
    from the salver and forms with five grains a cross at each of the five
    places on the table of the altar where he before made the unctions with the
    Oil of Catechumens and Holy Chrism, observing the order in the following
    plan.\footnote{The center cross should not be placed over the sepulcher.}

    \crossplan

    After having made each cross, he places one of the crosses made of wax
    tapers on the grains, before making the next cross.\footnote{Marginucci,
    Lib. VII, cap. XVI, n. 112, foot-note, says that it is the custom
    (\textit{presumably in Rome}) to attach beforehand five grains of incense
    to the crosses made of wax tapers, and then the bishop needs only to place
    the taper-crosses at their places with the grains of incense turned
    downwards. He remarks, ``\textit{H\ae c ratio est commodior atque
    expeditior et pr\ae scriptio rubric\ae\ eodem modo impletur.}''} As soon as
    the Bishop has made the fifth cross a server(s) lights the four ends of
    each cross of wax tapers.

    The Incensing Priest carries the censer to the sacristy and his duty
    ceases.\footnote{The rubrics do not determine when the function of the
    incensing priest finishes. Ending his function after lighting the candles
    is the opinion of Sthele, and seems the most probable, since blessed
    incense is already burning on the altar and the thurifications of the
    incensing priest seem superfluous at this point.} The salver is put
    on the credence.

    \rubric A cushion is placed on the lowest step of the altar in the middle
    and, when the crosses have been lit, the Bishop and his ministers descend
    to the foot of the altar. The Deacon removes the Bishop's miter and
    zucchetto, after which all kneel. The Bishop, kneeling on the cushion,
    intones the \textit{Allel\'uia} and the versicle \textit{Veni sancte
    Spíritus} (these words only), which the chanters continue and to which they
    add the two antiphons \textit{Ascéndit} and \textit{Stetit Angelus}.

    \rubric The \textit{Allel\'uia} and the versicle \textit{Veni sancte
    Spíritus\dots accende} having been sung, all rise and the Deacon puts the
    zucchetto on the Bishop. When the chanters have finished the antiphon
    \textit{Stetit Angelus} the Bishop says \textit{Orémus}, to which the Deacon
    and Subdeacon add \textit{Flectámus génua; Leváte.} The Bishop recites the
    prayer \textit{Dómine sancte.}

    \rubric As soon as the grains of incense on the altar are consumed, servers
    scrape with spatulas the burnt incense and wax from the altar and place the
    scrapings into a vessel prepared for that purpose. The scrapings are
    afterwards thrown into the sacrarium. Afterwards, two priests rub off with
    cotton and sponges the oil on the table of the altar and at its four
    corners and wipe these places with coarse towels or cloths.\footnote{If the
    priests purifying the altar would get in the way of the anointing of the
    front of the altar and the anointing of the juncture, this purification is
    postponed until after these anointings.}

    \rubric After the prayer \textit{Dóminus sancte} the Bishop once again says
    \textit{Orémus}, to which the Deacon and Subdeacon add \textit{Flectámus
    génua; Leváte.} The Bishop recites the prayer \textit{Deus omnípotens.}

    \rubric After this prayer, with his hands extended on his breast, the Bishop
    recites in a \textit{moderate tone} of voice (or sings \textit{tono feriali})
    the Preface. He says the conclusion in a low tone.

    \section{Anointing of the Front of the Altar.}

    \rubric The Bishop resumes the miter and ascends the predella with his
    ministers. A server carries on a salver a vessel of Chrism and cotton, and
    gives it to the Deacon. The Bishop intones the antiphon \textit{Confírma
    hoc Deus} (these three words only), which the chanters continue and to
    which they add the psalm \textit{Exsúrgat Deus.} The antiphon is not
    repeated.

    \rubric As soon as the Bishop has intoned the antiphon \textit{Confírma hoc
    Deus,} he dips his thumb into the Chrism and makes with it the sign of the
    cross on the front of the altar, half-way between the altar and the
    predella, saying nothing.\footnote{If a metal cross is affixed to the front
    of the altar, it may be removed for this occasion, or the unction may be
    made \textit{above} this cross.} If the front of the altar is not solid
    this unction is made on the anterior part of the table, or if a column
    supports the table in front at the center this unction is made on the front
    of the column's cap. When making this unction the bishop does not kneel. If
    the holy oil should flow down the front of the altar, the deacon wipes it
    off with cotton. After the unction, the bishop wipes his thumb with cotton.

    \rubric A server receives the Chrism from the Deacon and carries it to the
    posterior corner of the altar at the Gospel side. The Bishop and his
    ministers go to the foot of the altar and stand \textit{in plano} facing
    the altar.

    After the psalm the Deacon removes the miter and the Bishop says the prayer
    \textit{Majestátem.}

    \section{Anointing of the Juncture of the Table and the Support.}

    \rubric After this prayer the Bishop resumes his miter and goes with his
    assistants to the posterior corner of the altar at the Gospel side and,
    having dipped his thumb into the Chrism, makes with it three times the sign
    of the cross over the junction of the table and support, as if to join them
    together, drawing the upright line from the table to the support and the
    transverse line over the juncture of the table and support, saying:

    \begin{quote}
        In nómine Pa\cross tris, et Fí\cross lii, et Spíritus \cross\ sancti.
    \end{quote}

    He does the same in succession at the anterior corner on the Gospel side, at
    the posterior corner on the Epistle side and lastly at the anterior corner on
    the Epistle side. If the holy oil should flow down the support, the deacon will
    wipe it off with absorbent cotton.

    \rubric The bishop returns with his ministers to the middle of the altar
    \textit{in plano.} The server carries the Chrism to the table. The Bishop's
    miter is removed and he says the prayer \textit{Súpplices.}

    \rubric At the end of this prayer the bishop, with his ministers, goes to the
    faldstool which is placed \textit{in plano} at the Epistle corner of the altar
    steps and sits. Two servers, one with the ewer and basin and the other carrying
    a tray with cotton balls and alcohol swabs and a towel, go to the Bishop. The
    Bishop cleans his hands, washes and dries them.

}

\chap{Blessing of the Ornaments}{

    \section{Blessing of the Altar-cloths, Vases and Ornaments for the Consecrated
    Church and Altar.}

    \rubric When the Bishop has washed his hands and the altar has been washed,
    servers bring to the faldstool the altar-cloths and other altar and church
    ornaments to be blessed by the Bishop.\footcite[In case they should be very
    numerous, they may be left on a table to which the bishop goes and there
    blesses them.][footnote 4, p. 128.]{consecranda} At the same time a server
    takes the aspersorium and stands to the right of the Deacon. The Deacon removes
    the miter and the Bishop rises and blesses the altar cloths and other ornaments
    with the prayer \textit{Omnípotens.}

    After this the Deacon hands the aspergillum to the Bishop who sprinkles the
    articles. The Bishop then sits and resumes the miter.

    \rubric The sacristans cover the altar with the cerecloth\footnote{A linen
    cloth, waxed on one side, which is commonly called the \textit{Chrismale.} It
    must be of the exact size of the mensa, and it is placed under the linen
    altar-cloths, the waxed side being turned towards the table. The cerecloth is
    not necessary unless the oil remains on the stone despite the efforts of the
    clergy in charge of cleaning the altar.} over which they place the three
    altar-cloths. The cross and candlesticks are also put in the proper places, and
    the predella and steps of the altar are covered with carpets.

    When the sacristans begin to cover the altar the Bishop rises, and, having
    turned towards the altar, intones the antiphon \textit{Circumd\'ate Lev\'it\ae}
    (these two words only), which the chanters continue and to which they add the
    antiphons, responsory and psalm that follow, during which the Bishop, wearing
    the miter, remains standing. The antiphon \textit{In velaménto,} which precedes
    the psalm, is not repeated.

    \section{Incensation of the Altar.}

    \rubric When the altars have been covered and ornamented and the psalm
    \textit{Deus Deus meus} is finished, the Bishop sits and in the usual manner
    puts incense into the thurible held before him by the thurifer.

    \blessincense

    The Bishop then rises, receives the crosier and goes with his ministers to the
    foot of the altar, where he lays aside the crosier and the Deacon removes the
    miter. The Bishop bows to the cross on the altar, and the Deacon and Subdeacon
    make at the same time a simple genuflection, after which they ascend to the
    predella. The Deacon, having received the thurible from the thurifer, hands it
    to the Bishop, who intones the antiphon \textit{Omnis terra} (these two words
    only), which the chanters continue to the end. After the Bishop has intoned and
    antiphon, he incenses the mensa by making \textit{once} with the thurible a
    cross over it.

    \rubric When the chanters have finished the antiphon, the Bishop again intones
    the same antiphon and incenses the mensa a second time in the same manner as
    above, while the chanters repeat the antiphon.

    \rubric At the end of the antiphon the Bishop intones the same antiphon a third
    time and incenses the altar a third time in the same manner as above, while the
    chanters repeat the antiphon.

    Then the Bishop gives the thurible to the Deacon who hands it to the thurifer.

    \rubric When the chanters have finished this antiphon the third time, the
    Bishop, standing on the predella and facing the altar, recites the prayers
    \textit{Descéndat} and \textit{Omnípotens} and the versicles which follow them.

    \rubric The Bishop with his ministers descends to the foot of the altar, where
    he resumes his miter, bows to the cross (the Deacon, the Subdeacon and other
    attendants make a \textit{simple} genuflection \textit{in plano}) and receives
    the crosier.

    Then the following is observed: either

    \begin{enumerate}

        \item All repair to the sacristy, in the order in which they came to the
            Church, to vest for Mass; or

        \item If the Consecrator is to sing the Mass, he may be accompanied to the
            faldstool by the ministers; the latter then go to the sacristy to put
            on the dalmatic and tunic and return to the throne or faldstool to vest
            the consecrator; or

        \item If the Consecrator is to celebrate a \textit{low} Mass, he is led to
            the faldstool by the ministers, who remove the Consecrator's miter and
            cope. The ministers then, preceded by the cross-bearer, acolytes and
            other assisting servers, go to the sacristy and divest, and the
            chaplains take their places at the side of the Consecrator, whom they
            assist in vesting. If the Mass is celebrated by a priest, it ought to
            be a Solemn High Mass.
        
    \end{enumerate}

    \rubric While the celebrant and his ministers are vesting for Mass, the altars
    are ornamented and the stoups at the entrance of the church are filled with
    holy water. If any of the holy water that was blessed \textit{at the door of
    the church} remain, it may be used for this purpose.

    \rubric After the Mass the ashes which were spread on the floor of the church
    for the tracing of the Greek and Latin alphabets are swept away by the
    sacristans, the church undergoes a cleaning and the soiled sponges, towels,
    cloths, cotton, etc., are removed.

}

\chap{Mass and Divine Office}{

    \section{Mass of the Consecration.}

    \rubric With regard to the Mass the following rules are to be observed:

    \begin{enumerate}[label=(\Roman*)]

        \item \textit{In genere} it wil be the Mass \textit{Terribilis est locus
            iste} as found in the \textit{Commune Dedicationis pro Anniversario,}
            with \textit{Gloria;} the prayer \textit{Deus qui invisibiliter,} etc.,
            found at the end of this Mass;\footnote{Those commemorations must be
            added which are not omitted even on double feasts of the first class,
            i.e., Sundays, major ferials, octave-days (not days \textit{within} an
            octave) and days within the privileged octaves of Christmas, Epiphany,
            and \textit{Corpus Christi.} --- \src, Feb. 23, 1884, n. 36.05, III
            ad 1.} \textit{Credo;} Preface, either \textit{de Octava} if it is
            \textit{proper,} otherwise \textit{de Trinitate} or \textit{Communis}
            according as the consecration takes place on a Sunday or week-day;
            Gospel of St. John at the end of Mass, unmless the commemoration of a
            Sunday, or of a ferial which has a proper Gospel was made, when the
            Gospel of the Sunday or ferial is read.

        \item On the following days the Mass \textit{Terribilis est locus iste}
            will be celebrated in the manner described above, but the commemoration
            of the feast celebrated on that day will be added \textit{sub unica
            conclusione} to the oration of the Mass \textit{Terribilis.}
        
            \begin{enumerate}[label=(\arabic*)]

                \item Circumcision, Sacred Heart;

                \item Immaculate Conception, Annunciation and Assumption of the
                    B.V.M.;

                \item Nativity of St. John the Baptist, St. Joseph, Ss. Peter and
                    Paul, All Saints;

                \item During the octaves of Epiphany, of Easter \textit{from
                    Wednesday to Saturday,} of Pentecost \textit{from Wednesday to
                    Saturday;}

                \item Vigils of Christmas and Pentecost.

            \end{enumerate}

        \item On the following days the current Mass is celebrated and a
            commemoration of the Dedication Mass, \textit{Terribilis,} is added
            \textit{sub unica conclusione.}

            \begin{enumerate}[label=(\arabic*)]

                \item Sundays: I Advent, I Lent, Passion, Palm, \textit{in Albis,},
                    Trinity;

                \item Feasts: Christmas, Epiphany, Easter (Sunday, Monday,
                    Tuesday), Ascension, Pentecost (Sunday, Monday, Tuesday),
                    \textit{Corpus Christi;}

                \item Ash-Wednesday and during Holy Week.\footnote{\src, Feb. 23,
                    1884, n. 3605, III ad 3.}

            \end{enumerate}

    \end{enumerate}

    \rubric At the end of Mass the Bishop gives the solemn blessing in the usual
    manner, \textit{at the altar} if he is the celebrant of the Mass.

    \section{Mass on the Feast and during the Octave.}

    \rubric The following rules are to be observed:

    \begin{enumerate}

        \item \textit{Regularly} Mass should not be celebrated in the church on
            that day \textit{before} the consecration.
        
        \item If the priests strictly attached to the church must for any reason
            celebrate Mass in the church \textit{before} its consecration on the
            day of the consecration, this Mass must be \textit{conformis
            officio;}\footnote{This presupposes that the church is a
            \textit{blessed} church.}

        \item All Masses celebrated in the church on the day of the consecration
            \textit{after} its consecration must be \textit{de Dedicatione,} as
            described above.

        \item If on days during the octave, on which it is allowed by the Rubrics,
            the \textit{Missa de octava} is delebrated it will be \textit{de
            festo,} as given above, except that the \textit{second} oration will be
            \textit{de B. Maria V.} and the \textit{third} will be
            \textit{Ecclesi\ae.} When, however, a special commemoration (of a
            \textit{simplex} feast) is to be made, it will be the \textit{second}
            an the \textit{third} will be \textit{de B. Maria V.,} and the
            commemoration \textit{Ecclesi\ae} will be omitted. When two special
            commemorations are to be made, the oration \textit{de B. Maria V.} is
            also omitted.

    \end{enumerate}

    \section{Divine Office on the Feast and during the Octave.}

    \rubric The Consecrator, even if he is not the Ordinary, may (is not obliged)
    recite the \textit{Officium de Dedicatione.}\footnote{\src, May 7, 1746,
    n.2390 ad III.}

    \section{Anniversary of the Consecration.}

    \rubric The rules given above concerning the Mass on the day of the
    consecration of a church and during its octave are applicable to the
    anniversary also, except the prayer which is \textit{Deus qui nobis per
    singulos annos.} If another dedication is to be commemorated, the prayer for
    this commemoration will be \textit{Deus qui invisibiliter} (with Secret and
    Postcommunion at Mass), proper of the consecration.

}

%%%% BIBLIOGRAPHY
\nocite{levav:churchconsecration}
\nocite{ml:1947}
\nocite{ml:1959}
\nocite{rc:ecclesiae}
\nocite{martinucci:3}
\printbibliography
    
\end{document}
