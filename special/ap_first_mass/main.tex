% !TEX program = lualatex
\documentclass[10pt,twocolumn]{article}

\usepackage{microtype}
\usepackage{fancyhdr}
\usepackage{fontspec}
\setmainfont{Minion 3}
\usepackage[english]{babel}

%\title{{\normalsize Assistant Priest}\\ {\footnotesize for a}\\ New Priest's First Mass}
\title{Assistant Priest at a First Mass}
%\author{Rev. Mr. Gregory R. Barnes}

\newcommand{\src}{
    \textsc{s.r.c.}
}
\newcommand{\ap}{
    \textsc{a.p.}
}
\newcommand{\ce}{
    \textsc{c.}
}
\newcommand{\mc}{
    \textsc{m.c.}
}

\begin{document}
\maketitle

\section*{Preliminary Observations}
\begin{enumerate}
    \item It appears that the \ap can even be a bishop, as prelates are not
        excluded, but rather included by the \src
    \item To summarize the role of the \ap: \textit{Stand to the missal side of
        the Celebrant, and stay out of his way.}
\end{enumerate}

\section*{Low Mass}

\subsection*{Vesting}
\begin{enumerate}
    \item \ap vested in surplice. If a prelate, the surplice must be worn over
        his rochet.
    \item The use of a stole is permitted, but does not seem to be in conformity
        with liturgical law.
    \item \ap does not wear biretta.
\end{enumerate}

\subsection*{Function}
\begin{enumerate}
    \item \ap processes in to the left and a little in front of the \ce
    \item He steps back to allow the server to take the biretta from the \ce
    \item He kneels to the right and behind the \ce for the prayers at the foot
        of the altar (the other server should be to the right of the \ap, so
        that the \ap is closest to the celebrant).
    \item \ap ascends the altar with the \ce, and accompanies him to the epistle
        side, remaining at his right.
    \item From the Introit to the Gospel, he stands to the right of the \ce and
        accompanies him to the middle of the altar for the \textit{Kyrie} and
        \textit{Gloria.} He may turn the pages of the missal, as an \mc would at
        high Mass, and indicate the prayers.
    \item The server moves the missal. \ap descends \textit{per breviorem} to
        the floor, at the center, genuflects, and goes to the left of the \ce
    \item After the Gospel, \ap allows the \ce the kiss the missal and move it
        towards the center himself. \ap now remains to the left of \ce, near the missal.
    \item The \ap has no function during the Offertory.
    \item \ap kneels on the footpace close to \ce at the Elevation, and elevates
        the chasuble.
    \item If Communion is distributed, the server recites the
        \textit{Confiteor} and holds the paten. \ap remains to the left of \ce
    \item After the ablutions, the servers move the missal, and the \ap returns
        to the Epistle side the same way he came.
    \item During the \textit{Placeat} \ap moves the the Gospel side corner,
        where he kneels for the blessing on the lowest step.
    \item \ap ascends to assist \ce with the Last Gospel, stading to his left.
    \item There are no prayers after low Mass. \ap precedes the \ce into the
        sacristy, standing a bit to his right.
\end{enumerate}

%\section*{Solemn Mass}

%\subsection*{Vesting}
%\begin{enumerate}
    %\item \ap vested in surplice, amice and cope. If a prelate, the surplice must be worn over
        %his rochet.
    %\item The use of a stole is permitted, but does not seem to be in conformity
        %with liturgical law.
%\end{enumerate}


\end{document}
