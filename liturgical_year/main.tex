% !TEX program = lualatex
\documentclass[10pt]{report}

\usepackage{microtype}
\usepackage{fancyhdr}
\usepackage{fontspec}
\setmainfont{Minion 3}
\usepackage[english]{babel}

\begin{document}

\chapter{Sunday Within the Octave of Corpus Christi}

\section{Overview}

This ceremony is nearly identical to that of Corpus Christi itself, with the
exception of the \textit{Asperges.} There is also a greater chance that there
will be a sermon on this day rather than on the feast itself, so that procedure
shall be explained here.

\section{Asperges}

\subsection{Exposition}

The procession and exposition are performed in the same manner as on the feast
itself.

After the incensation of the Blessed Sacrament, the ministers remain kneeling.
The MC gets the aspersorium (or better, an acolyte brings it to him) towards the
end of the hymn sung during the incensation, and he signals all in choir to rise
for the \textit{Asperges.}

\subsection{Asperges}

There are a few differences from the usual ceremony:

\begin{enumerate}

    \item Per the rules \textit{coram sanctissimo,} the \textit{oscula} are
        omitted.

    \item The altar is not sprinkled\footnote{Check this one.}.

    \item Ministers double genuflect before processing out of the sanctuary. All
        three must take care to walk obliquely down the aisle, to avoid turning
        their backs completely on the Blessed Sacrament.

    \item When turning back towards the altar, all three must take care not to
        complete a 360 degree turn.

\end{enumerate}

\section{Sermon}

\end{document}
