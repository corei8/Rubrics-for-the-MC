% !TEX program = lualatex
\documentclass[letterpaper]{report}
\title{Forty Hours Devotion}

% PACKAGES

\usepackage{enumitem}
\usepackage{rubrics}
\usepackage{microtype}

\newcommand\instr[1]{\textit{Instr. Clement.,} n. #1.}

\begin{document}
\maketitle

\chap{Preparations}{

    \section{Notes}

    \rubric The \itshape{Instructio Clementina} binds strictly only in Rome
    itself.\footnote{SRC 2403.} It is clear from the responses of the Sacred
    Congregation of Rites that the mind of the Church is that the
    \textit{Instructio} be followed in other places as well. It Indulgences
    cannot be gained unless the \textit{substance} of the \textit{Instructio}
    is followed.

    \rubric The processions on the first and third days may be omitted. In the
    United States this permission is given by a special Indult. The pastor is
    the judge of whether or not the processions can be held.

    \rubric In 1914, Pius X relaxed the rule of an uninterrupted 40 hours of
    adoration, in order to allow the practice of the Forty Hours to take place
    in more places. The Blessed Sacrament is then reserved the evening of the
    third day, in order to make up for lost time.

    \rubric Forty Hours' Devotion may be held at any time during the year,
    except from the morning of Holy Thursday to the morning of Holy
    Saturday.\footnote{SRC 1190.}

    \section{Decoration of the Church and Altar}

    \rubric A sign or banner should be placed on the outer door of the
    church.\footnote{\instr{2}} This should be decorated with a symbol of the
    Blessed Sacrament, and should inicate that Forty Hours' Devotion is being
    held in the church.

    \rubric No statues or reliquaries are to be on the altar.

    \rubric The Blessed Sacrament should be removed from the tabernacle for the
    entire exposition.

    \section{Cross, Antependium, and Tabernacle Veil}

    \rubric It is not necessary that there be a cross on the main altar, even
    during Mass. It should be on the altar for the first Mass, however, because
    the Blessed Sacrament is not yet exposed.

    \rubric The antependium and tabernacle veil should be white at all times.

    \section{Candles and Church Bell}

    \rubric Twenty candles should be arranged on the altar and these candles
    should remain burning throughout the exposition.

    \rubric On the day before the exposition, after the \textit{Angelus} has
    been sounded, the church bells should be rung more solemnly. They should be
    rung solemnly again on the morning of the exposition, and every hour during
    the exposition.

    \section{Masses of Expostion and Reposition}

    \rubric The Mass to be said on the day of exposition and on the day of
    reposition is the same, i.e., the sollamn votive Mass of the Blessed
    Sacrament, unless this Mass is not permitted. It should be celebrated as a
    Solemn Mass.\footnote{An Indult is necessary to substitute a High Mass or a
    Low Mass.}

\chap{Coram Sanctissimo}{

}

\end{document}
