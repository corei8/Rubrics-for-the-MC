\chap{Maundy Thursday}{

\section{Preparation}

Work in progress\dots

\section{Up to the \textit{Oleum Infirmorum}}

\rubric The seven SD\textsuperscript{O}s, seven D\textsuperscript{O}s, and
vested priests come before SD of the Mass in the procession.

\rubric \textit{Júdica me} is omitted.

\rubric Bells are rung throughout the read \textit{Glória}\footnote{``Feria V.
in C\oe na Domini in Missa organa pulsari possunt per integrum hymnum
Angelicum.'' S.R.C. n. 4067 ad 6.}; this is the last time the bells are rung
until Holy Saturday.

\rubric After pouring water at the Offertory, SD removes humeral veil and
stands to the left of B. He remains to B's left until B is incensed by D, when
he goes to his usual place at the foot of the altar. At the Offertory, D
incenses Priests, D\textsuperscript{O}s, and SD\textsuperscript{O}s with a
double each.

\rubric At the \textit{Sanctus} and for the rest of Mass a clapper is used
rather than bells.

\rubric SD approaches for \textit{Sanctus}. SD incenses at the elevation. AP
guides the B through the proper prayers of the Canon.

\rubric After the \textit{Nobis quoque peccatóribus}, MC1 brings empty chalice
with purificator to the Epistle side; AC1 brings wine and water. SD ascends to
the right of D. After the 2nd \textit{Meménto,} B stops after reading:
\textit{\dots largítor admítte. Per Christum Dóminum nostrum.} All genuflect,
B, D, SD to Epistle side. AP descends and waits. SD pours wine and water over
B's fingers. AC1 takes cruets back to credence. D presents purificator, B dries
fingers. D covers chalice with purificator, puts it to the Epistle side of the
purificator. MC1, MB, SB to places for B's descent.

\section{Oil of the Sick}

\rubric After the purification, B, D, SD to the center of the altar, genuflect
and descend, D and SD to the Gospel side. The crozier is presented without
\textit{oscula.}\footcite[184]{stehle} B and ministers process to table, with
MB \& SB following. B is between AP and D, with SD leading.

\rubric Once at the table crozier away, B sits. AP stands to the left of B to
attend to the Pontifical, D \& SD stand behind B.\footnote{As at the Introit,
SD standing to the left of D, both centered on the B.} AP goes before B, bows,
and says \textit{Oleum infirmórum.} He then bows and returns to his place to
the left of the Bishop.\footnote{Stehle says that AP remains to the left of the
Bishop.}

\rubric SD\textsuperscript{O} leaves his place\footnote{Or one of the SDs for
the oil ceremony.} and stands before the B, taking the place of the AP. ACs
with folded hands come from the credence table and stand behind
SD\textsuperscript{O}. All three proceed to the sacristy (ACs leading) with the
usual reverenced to B, and to the Blessed Sacrament. They are accompaned by
MC2.

\rubric SD\textsuperscript{O} and ACs return, SD\textsuperscript{O} carring the
ampulla covered in violet on his left arm\footnote{Or at least towards his left
side, as high as possible, if this be impossible.}. All three make the usual
reverences to the Blessed Sacrament and to B.

\rubric SD\textsuperscript{O} hands ampulla to AP saying in a low tone:
\textit{Oleum Infirmórum.} He remains near AP, to the left of B\footnote{It
seems appropriate that the AP be near B, and SD\textsuperscript{O} be to the
left of the AP.}.

\rubric AP presents oil to B, saying in a low tone: \textit{Oleum Infirmórum.}
AP places the ampulla on the table. D removes veils and cover of ampulla. B
rises with miter and reads exorcism.

\rubric B sits with miter for hand washing. MC1 can do this himself or BB \& CB
can. D ministers the towel, AP the ring. If SD\textsuperscript{O} is SD of the
Mass, then all wait for him to return. 

\rubric B receives crozier. All process to altar, SD leading, B following with
AP to his right and D to his left.

\rubric SD will stand to the Gospel side as he approaches the foot of the
altar. IBs stand to the Gospel side behind D.

\rubric At the foot of the altar, crozier away, miter off by D, zucchetto off
by MC1. All genuflect on one knee, ascend, and B continues Mass.

\section{\textit{Per quem hæc} to the Second Hand Washing}

\rubric No \textit{Pax}.

\rubric At the \textit{Agnus Dei} MC1 brings up the chalice prepared for the
Repository. It is recommended that he come slightly before the \textit{Agnus
Dei,} in order to stop the ministers if they begin the movement for the
\textit{Pax.}

\rubric At the \textit{Agnus Dei} stoles are to be distributed to the priests
and deacons who wish to receive Communion, but this excludes the
\textit{parati.}

\rubric After the consumption of the Precious Blood D covers chalice, B, D \&
SD genuflect. 

\rubric D moves the chalice of the Mass to the left part of the corporal. D
takes Repository chalice, holding it tilted towards B, with its base on the
corporal. B puts the Repository host into the chalice, by sliding it into the
cup from the paten.

\rubric D covers Repository chalice with pall and inverted paten, and places
over them a veil\footnote{Stehle says that this should be fastened at the node
of the chalice right away.}.

\rubric Distribution is in the usual manner.

\rubric After the tabernacle door is closed, all must take care to observe the
rules \textit{Coram Sanctíssimo.}

\rubric TBs remain in their places.

\rubric B consumes the contents of the Purification chalice.

\rubric B take the first ablution from the Purification
chalice\footcite[186]{stehle}, the second from the chalice of the Mass. MC1
takes away the Purification chalice after the first ablutions.

\rubric No zucchetto.

\rubric After genuflecting in the center, B washes his hands to the Epitle
side, \textit{in plano,} facing the people, no miter.

\section{Chrism}

\rubric B goes to the center with D \& SD and genuflects.

\rubric For the rest see 
     %\ref{totabelforoils}.

\rubric As B sits and give up crozier, TH comes and stands to the Epistle side
of the table.

\rubric AP goes before B, bows, and sings (\textit{tono lectionis\footcite[p.
186]{stehle}}): \textit{Oleum ad Sanctum Chrisma: Oleum Catechumenórum.} AP
returns to the left of B.

\rubric Imposition of incense.

\rubric Meanwhile ACs with candles stand between the table and the altar,
facing B. The D\textsuperscript{O}s stand before B, an SD\footnote{For the
balsam.} behind them. TH joins the formation between the ACs and the altar.

\rubric All make the usual reverences to B and the Blessed Sacrament as they
depart.

\rubric In the Sacristy, the D\textsuperscript{O}s take up the ampullæ with
white humeral veils: the Chrism has a white veil and the Catechumens a green.

\rubric In the procession back, a SD carries the cross between the
ACs\footcite[p. 187]{stehle}.

\rubric All make the usual reverences to the Blessed Sacrament. All make the
usual reverences to B except the TH, ACs, and SD with the cross, who stand off
to the Epistle side.

\rubric D\textsuperscript{O} carrying the oil of the Catechumens stands to the
Gospel side, facing B\footnote{In this way he avoids turning his back to the
Blessed Sacrament.}.

\rubric D\textsuperscript{O} carrying the Chrism hands the ampulla to AP,
saying nothing, with the vessel covered in the humeral veil. AP presents it to
B, saying nothing. SD with the balsam in the same manner presents his vessel to
AP, who presents it to B.

\rubric D removes cover of the balsam. Miter off, B rises and says
\textit{Dóminus vobíscum} and the orations for the blessing of the balsam.

\rubric Miter on, D uncovers ampulla of Chrism and takes a little of the oil
into a small metal cup. B pours the balsam into it and mixes both with a spoon
and says: \textit{Orémus Dóminum Deum, etc.}

\rubric B sits or stands\footnote{Whichever is more convenient.} and breathes
three times over the ampulla in the form of a cross. D and AP should make sure
that the miter stays on B's head, if there be a danger of it falling.

\rubric The Priests do the same.\footnote{For this and for the salutations, the
most senior priest goes first, from the Epistle side. He first stands almost at
the foot of the altar, reverences the Blessed Sacrament, then reverences B. He
approaches the table, breathes three times voer the ampulla, reverences B and
the Blessed Sacrament, then returns to his place. He is followed by the most
senior priest on the Gospel side, and so on. For the salutations, after his
reverences he makes his first salutation, then goes halfway to the table for
his second, then to the table for the third.}.

\rubric B stands with miter for the exorcism.

\rubric Miter removed, B begins Preface. 

\rubric At the words \textit{Hæc commíxtio liquórum,} B pours the balsam and
oil mixture into the ampulla. Afterwards, D\textsuperscript{O} takes the veil
from the ampulla and ties it around his neck.

\rubric B stands and sings \textit{Ave, Sanctum Chrisma} three time, each time
in a higher tone, then kisses the edge of the ampulla and sits with miter.

\rubric D moves the ampulla to the opposite side of the table. The priests
salute the oil, kneeling on both knees each time\footnote{``Genuflectionem
debent conficere.'' S.R.C n. 4269 ad 14.}.

\rubric D covers the ampulla and places it on the ``gospel side'' of the table.

\section{Holy Oil}

\rubric D\textsuperscript{O} carrying the oil of the Catechumens passes the
ampulla to AP, who places it before B, and returns to his place. 

\rubric D\textsuperscript{O} retains his veil.

\rubric D opens the ampulla.

\rubric B, and the priests after him, breathe three times on the oil.

\rubric B rises with miter and reads the exorcism.

\rubric Miter off. B reads \textit{Deus incrementórum, etc.}

\rubric Salutation of the oil, the same manner as above, using the formula:
\textit{Ave Sanctum Oleum.}

\rubric D covers the ampulla.

\rubric AP hands the ampullæ to the D\textsuperscript{O}s.

\rubric Incence is imposed\footcite[It appears that B does not need to impose
this incense, but it can be done by MC1 or MC2 or by TH himself.][p.
190]{stehle}.

\rubric The oils are taken back to the sacristy in the same way they came. The
Choir sings \textit{O Redemptór.}

\rubric B's hands are washed. Procession back to the altar is as described
above.

\section{Communion to the End of Mass}

\rubric B genulfects on the floor with ministers, genuflects at the center of
the altar, then reads the Communion prayer at the Epistle side.

\rubric When giving the Blessing, B stands slightly to the Gospel side, hold
crozier, but no miter.

\rubric Genuflection at the Last Gospel is made towards the Respository
chalice.

\rubric B and ministers genuflect at the center and proceed\footnote{B receives
miter and crozier as before on the step below the footpace.} \textit{per
breviorum} to the faldstool, which is turned towards the Gospel side, so that
B's back is not turned on the Blessed Sacrament.

\section{Altar of Repose}

\rubric D \& SD remove maniples\footnote{From here B is assisted by Assistant
Deacons (ADs). Since there is not a superabundance of deacons available, these
rubrics are written with D \& SD taking the places of the ADs.}. AP removes his
cope at the Epistle credence and dons a white stole. He joins the choir.

\rubric D \& SD assist B in removing miter, maniple, chasuble, dalmatic and
tunic. B receives a cope and precious miter.

\rubric Incense is imposed in two thuribles (TH2 first) without blessing or
\textit{oscula.}

\rubric B, flanked by the ministers, who are flanked by the THs, proceed to the
altar, double genuflect, and kneel on the lowest step (except THs, who kneel
\textit{in plano}). The Blessed Sacrament is incenced by B. B receives humeral
veil from MC1. MC2 forms the procession.

\rubric B, D rise and ascend. B kneels on the edge of the predella. D
genuflects at the center, takes up the Blessed Sacrament, and turns to B. D
hands the chalice to B, genuflects, B rising with him. D and SD change
places\footnote{SD ascends to the Epistle side as D goes to the Gospel side of
B, walking between the altar and B.} as B turns, and all three face the
procession.

\rubric MC1 signals all to rise as B is rising. When B is facing the
procession, all double genuflect\footnote{Double check this rubric. Fr. Cekada
mentions it.}.

\rubric Procession moves to the repository \textit{per breviorum} and without
clapper\footnote{Feb. 1, 1907 S.R.C. 4198\textsuperscript{xiv}}. The choir
sings \textit{Pange Lingua.}

\rubric Order of the procession:

\begin{enumerate}
    
    \item Crossbearer\footnote{Preferably a subdeacon} between ACs with
        candles;

    \item Chanters\footcite[``If they are clerics; if not they precede the
        cross-bearer''.][p. 191]{stehle};

    \item Clergy; the seven subdeacons two, two, three; the seven deacons in
        the same order; priests, two and two;

    \item Prelates (\textit{dignores ultimo loco}).

    \item D \& SD of the Mass\footnote{Assuming that there are assistant
        deacons for B. If not, then D \& SD accompany B, as these rubrics
        indicate.};

    \item AP and SB\footcite[``If the width of the aisle permits, the assistant
        priest may walk at the right of the bishop, slighlty in advance of the
        assistant deacon, and the staff-bearer may walk between the
        thurifers.''][footnote 2]{stehle};

    \item TBs;

    \item THs;

    \item B between D \& SD; umbrella\footnote{A canopy may be used, if the
        Respository is not in the same building.} covering B;

    \item MB; BB \& CB.

\end{enumerate}

\rubric All kneel upon entering the repository.

\rubric D \textit{in plano} receives the Repository chalice from B in the same
manner B received it. B and SD kneel \textit{in plano}\footnote{``Celebrans non
debet supremum Altaris gradum ante suppedaneum conscendere, ut Calicem Diacono,
stando porrigat; sed debet in plano\dots etc.'' S.R.C., n. 4251 ad 14.} as D
ascends and approaches the altar.

\rubric D places the chalice in the tabernacle, but does not close the door.
Humeral veil is removed from B.

\rubric D kneels to the right of B. \textit{Tangtum ergo} is begun by the
choir. Incense is imposed and the Blessed Sacrament is incensed by B as at
Benediction.

\rubric D rises and closes the tabernacle\footnote{Does D put away corporal?}.

\rubric After a short prayer\footcite[B is instructed to give a blessing at
this point with miter and crozier. Is this reserved to the Ordinary?][p.
192]{stehle}, MC1 signals all to rise and make a double genuflection. All then
go to the sacristy \textit{per breviorem}. As soon as B is out of the
Repository he receives zucchetto, miter and crozier.

}


