\chap{Good Friday}{

    \section{Preparation}

    Work in progress\dots

    \section{Preliminary Observations}

    \rubric All choir bows are omitted from the Adoration of the Cross until
    None of Holy Saturday, inclusive\footnote{S.R.C. n. 3059 ad 27.}.

    \rubric All holy water stoups should be removed.

    \section{Mass of the Catechumens}

    \rubric ACs and TH lead the ministers in, and do not have candles or
    thurible; holy water is not presented to C. C bows while ministers
    genuflect after all have taken off birettas.

    \rubric C and ministers kneel and prostrate themselves at the foot of the
    altar; all kneel.

    \rubric After a short prayer ACs rise and spread an altar cloth. This cloth
    should be folded in such a way that it only covers the back of the
    \textit{mensa}; MC places the missal and its stand on the Epistle side,
    opening it to the first Lesson.

    \rubric MC kneels to the right of D. MC waits for a moment then signals all
    to rise, the prostation being about the length of a \textit{Miserére}.

    \rubric C, D \& SD ascend; C kisses altar and goes to the Missal; D \& SD
    to their Introit positions. A lector\footnote{Preferably a real lector. If no cleric is
    available, SD sings the Lesson.} sings the Lession at a lectern where the
    Epistle is usually sung; C reads Lesson and Tract.

    \rubric Towards the end of the Tract, and at a signal from the MC,
    Ministers stand \textit{unum post alium}; all in choir rise.

    \rubric C sings \textit{Orémus}, D \& SD bowing with him. D sings
    \textit{Flectámus génua} and all genuflect. Upon genuflecting, SD sings
    \textit{Leváte} and all rise. C does not genuflect. MC should signal for
    all to kneel and rise at the appropriate times, but should not make this
    signal too loudly, since everyone already has somewhat of an audio cue
    already.

    \rubric With hands extended, C sings the Oration in the ferial tone.

    \rubric As soon as C begins the Oration, SD his removes folded chasuble and
    receives the \textit{Lectionarium} from MC. SD and MC genuflect in center
    then go to the place where the Epistle is usually sung, and SD sings the
    Second Lesson in the Epistle tone; choir sits. After the Lesson, SD and MC
    genuflect in the center and go immediately to the sedilia where SD resumes
    the folded chasuble. SD rejoins D and C.

    \rubric After C is finished with the Tract, C, D \& SD go to sedilia
    \textit{per breviorum.} At the beginning of the last verse of the Tract, C,
    D \& SD go \textit{per breviorum} to the missal, standing in the Introit
    position.

    \rubric Chanters of the Passion enter when C is at the Missal.

    \rubric C begins to read the Passion when the singers begin; after
    finishing\footnote{C reads the Passion, not genuflecting. He then recites
    the \textit{Munda cor} in place, bowing towards the Crucifix, then reads
    the \textit{pars Evangelii}.}, C, D \& SD turn to face where the Passion is
    being sung. D, C \& SD kneel towards the Cross after the words \textit{trádidit
    spíritum.} After the Passion, the chanters return to the sacristy; choir sits.

    \rubric D does not ask for C's blessing; choir stands after D takes his
    book from the altar. ACs do not carry candles; Th does not take part. C
    does not kiss the book, nor is he incensed. After the Gospel, D \& SD stand
    \textit{unum post alium.}

    \rubric If a sermon is to be preached, C, D \& SD go to the sedilia;
    afterwards they return to the missal \textit{per breviorum} and stand
    \textit{unum post alium.}

    \section{Litanical Prayers}

    \rubric D \& SD stand \textit{unum post alium} C for the Litanical Prayers.
    No bow is made at the \textit{Oremus} whe it begins the Litanical prayer. D
    sings the \textit{Flectámus genua} and all genuflect; SD sings the
    \textit{Leváte}, after which all rise. C sings the oration in the ferial
    tone, hands extended. The prayer for the Emperor is omitted\footnote{S.R.C.
    3103, 3}, as well as that for the pope. After the prayer for the Jews the
    \textit{Amen} is not said, nor is the \textit{Orémus} or \textit{Flectámus
    génua} sung.

    \rubric At the beginning of the oration \textit{pro haereticis}, the ACs
    speard the violet carpet up to the foot of the altar and place the cushion
    for the cross.

    \rubric After the Litancial Prayers, C, D \& SD go to the sedilia
    \textit{per breviorum}. ACs assist C \& SD in removing chasubles. All sit
    at this time. D leads the C and SD to the Epistle side, on the floor, and
    receives the cross from the ACs\footnote{This cross is supposed to be the
    one above the tabernacle, but this is not possible.}. D presents the cross
    to the C who holds it with both hands facing the people with the Corpus
    facing the people. SD stands to the left and D to the right of the C.

    \rubric D assists the C in upper part of the cross, until the transverse. C
    holds the cross to the height of his eyes. AC1 stands before C with missal
    (from the altar) opened to the \textit{Ecce lignum Crucis}. C sings the
    words \textit{Ecce lignum Crucis} in a low tone, and is joined by the D and
    SD for the rest of the chant. When the choir sings \textit{Veníte,
    adorémus} the ministers turn towards the cross and kneel on both knees, all
    others following their example. The C remains standing.

    \rubric This ceremony is repeated two more times, the second time is on the
    platform on the Epistle side and the third time is at the center. Each time
    the \textit{Ecce} is sung in a higher tone. The second time the head and
    right arm is uncovered, and the third time the cross is entirely unveiled.

    \rubric All remain kneeling after the third time. C, proceeded by the MC,
    descends to the front steps, a little to the Gospel side, and kneeling
    places the cross on the cushion. MC kneels to the right of the C. AC1 goes
    down with C and MC and kneels to the left of the C, missal closed. C, MC
    and AC1 rise, genuflect to the cross, and as they rise from this
    genuflection all rise.

    \rubric AC1 goes to the credence table with the missal. MC and C go to the
    sedilia. D \& SD go to the sedilia \textit{per breviorem}. C, D \& SD
    remove their maniples and shoes, MC assisting the C and ACs assisting the D
    and SD.

    \section{Adoration of the Cross}

    \rubric When the C rises, D \& SD rise as well and remain standing until
    the C returns. Accompanied by the MC, standing to his left, the C goes to
    the edge of the sanctuary. He makes three successive double genuflections,
    MC standing at his left. After the third genuflection, C kisses the feet of
    the Corpus. He then rises, makes a simple genuflection to the cross and
    goes to the sedilia, where the MC assists him in putting on his shoes,
    maniple and chasuble, in that order. ACs wash C's hands after he dons the
    chasuble, then the C sits and puts on his biretta.

    \rubric When the C has returned to the sedilia, D \& SD go to adore the
    cross in the same manner. D kisses the cross first. They return to the
    sedilia, put on shoes, maniples and SD puts on his folded chasuble, ACs
    assisting them. D \& SD sit and put on their birettas.

    \rubric Three servers take three missals (or one acolytes takes a missal,
    if the text is large enough), open them to the \textit{Improperia} and
    kneel before the C and ministers. C, D \& SD read the \textit{Improperia}
    as indicated in the missal. The \textit{Improperia} is read either in its
    entirety or only as much as time allows.

    \rubric The choir adores the cross in order of rank. Servers adore the
    cross after the clergy. At all times there should be three pairs before the
    crucifix, genuflecting in unison. The adoration should be made
    expeditiously.

    \rubric The people may either come up to the sanctuary and worship the
    cross in the same way, or another crucifix may be placed on a cushion at
    the Communion rail (in which case a priest with surplice and black stole
    places the cross on the cushion, rises and genuflects to the cross).

    \rubric In the United States the people usually come up to the Communion
    rail and kiss a small crucifix presented to them by the priest (or as many
    priests as necessary).\footcite[][]{hweekls} The priest wears a surplice
    and a black stole and carries a white cloth in his right hand with which he
    wipes the feet of the corpus each time after presenting the crucifix.

    \rubric While the Ministers are reciting the \textit{Improperia}, AC2
    lights the six candles on the altar and the AC candles. He brings the
    ablution cup and purificator on the altar, and unfolds the altar cloth, so
    that it now covers the front of the \textit{mensa}, and closes the
    tabernacle door. When the Ministers have finished reading, AC1 places the
    Missal and the missal stand on the Gospel side of the altar. Towards the
    end of the adoration, D receives the burse and purificator from the MC and
    takes to the altar \textit{per breviorem}.

    \rubric After unfolding the corporal, D descends the front steps of the
    altar, takes the Cross from the cushion, and places it on the atlar. While
    the D transfers the cross, everyone kneels. MC accompanies the D when
    transferring the cross. ACs remove the cushion and runner after the cross
    is removed.

    \textbf{We follow Fortescue on this??}

    \rubric ACs take their candles and go to the sacristy. THs with lit
    thuribles, CB, ACs with candles, and TBs with torches enter the sanctuary
    and line up before the altar, THs at the edge of the sanctuary. MC leads
    the ministers to the altar. All genuflect and the THs lead the procession
    \textit{per breviorum} to the repository. The ministers walk single file,
    and only wear birettas if the altar of repose is some distance from the
    altar.

    \rubric At the repository the THs kneel at either side and kneel
    immediately. CB and ACs let other pass and then stand in the center. TBs
    stand to either side and then kneel in the center after everyone is
    through. C, D \& SD make a double genuflect before the repository, and
    kneel on the lowest step.

    \rubric After a moment of adoration, D goes up to the altar, genuflects,
    and opens the tabernacle door, but does not remove the Blessed Sacrament. D
    genuflects again, and returns to the right of C. The ministers make a
    medium body bow and rise to fill thuribles. TH2's thurible is filled first.
    There are no kisses or blessings. Th1's thurible is used for the
    incensation of the Blessed Sacrament.

    \rubric After the incensation MC places the humeral veil on C's shoulders.
    D rises, goes to the altar, genuflects, takes the chalice and, without
    placing it on the altar, hands it to the C, who is kneeling on the top
    step, SD to his left. D genuflects and C rises and turns towards the people
    as D rises. D goes to the right of the C (from behind) and SD goes to his
    left (from in front, as C is turning).

    \rubric Chanters begin the \textit{Vexilla Regis}, THs and TBs rise and the
    procession begins slowly \textit{per breviorum} back to the main altar.

    \section{Mass of the Presanctified}

    \rubric Upon returning to the sanctuary, CB goes to the sacristy
    immediately. ACs go to the credence table. TBs kneeel in their usual
    position for the Canon. THs go to either side and kneel. SD kneels on the
    first step, next to Th2.

    \rubric D kneels on the floor in front of the bottom step at the right of
    the C. C turns to D and gives him the chalice. D rises and waits until C
    genuflects to the Blessed Sacrament. C kneels on the bottom step and
    removes the humeral veil, which is taken by the MC. D goes to the center of
    the altar and places the chalice on the corporal. D unties the ribbon but
    does not remove the veil. D genuflects and returns to the right of C.

    \rubric Incense is imposed in Th1's thurible and the Blessed Sacrament is
    incensed in the usual manner. Ministers rise, ascend to the predella and
    all genuflect. The SD goes to the right of D immediately and genuflects
    when he arrives there.

    \rubric D removes the veil from the chalice and gives it to the MC. He
    removes the paten and pall from the chalice, without removing it from the
    corporal. While D holds the paten over the corporal, C tilts the chalice
    and lets the Host slide onto the paten. D hands the paten to C, but without
    the usual kisses. C places the host on the corporal with no offering and no
    sign of the cross.

    \rubric Ac1 hands the cruets to the SD. D takse the chalice in his left
    hand. Holding it above the \textit{mensa}, he receives the wine form the SD
    and pours it into the chalice. SD pours the water without asking for a
    blessing. D returns the chalice to the C, who sets it on the corporal
    immediately. D covers it with the pall. SD genuflects, goes to the left of
    C, and genuflects again.

    \rubric Incense is imposed without kisses or blessing. \textit{Oblata},
    cross and altar are incensed as usual, all genuflecting before and after
    the \textit{oblata} is incensed. After the incensation, D receives the
    thruible from C and gives it to Th. No person is incensed.

    \rubric Acs stand with the lavabo dish and towel at the Epistle side,
    facing the wall. D \& SD stand as for an oration, while C goes down to the
    floor and washes his hands, facing the people, without reciting
    \textit{Lavabo}. C goes to the center of the altar, D \& SD go up to flank
    him, and all three genuflect.

    \rubric D \& SD stand \textit{unum post alium}. C bows, placed his hands
    on the altar and recites the \textit{In spiritu humilitatis} in a low tone.
    C kisses the altar, genuflects, and turns to face the people, stepping
    slightly to the Gospel side. He says \textit{Oratre, fratres}, turns left
    back to the altar, without completing the circle, and genuflects. D \& SD
    do not say the \textit{Suscipiat}.

    \rubric C bows at \textit{Oremus} and sings the \textit{Pater noster} in
    the ferial tone, holding his hands extended. After answering \textit{Amen}
    silently, the C sings the \textit{Libera nos, quæsumus} in the tone of the
    ferial oration. He does not take the paten not make the sign of the cross.

    \rubric After the \textit{Libera nos, quæsumus}, D \& SD genuflect with
    the C and kneel behind him on the edge of the platform. C pushes the Host
    on the paten in the usual way, takes the paten in his left hand and the
    Host in his right. Keeping his eyes on the host he elevates It above the
    paten high enough so that everyone can see It.

    \rubric D goes immediately to the right of C and SD to his left. D removes
    the pall from the chalice and covers it again after the C has dropped a
    particle of the host into it. D \& SD genulfect with C and change places,
    genuflecting again.

    \rubric Bowing with hands joined on the altar, C says the \textit{Perceptio
    Corporis} silently. He genuflects with the ministers and consumes the Host
    in the usual manner. D \& SD bow during the C's communion.

    \rubric At a signal from C, SD uncovers the chalice. C gathers the
    fragments and consumes the wine into which the Particle of the Host was
    dropped, omitting the sign of the cross and saying nothing. TBs rise,
    genuflect at the center and return to the sacristy. All in choir sit.

    \rubric Ac1 passes the cruets to SD, who pours wine and water into the
    chalice. The first ablution of wine is omitted. SD places the purificator
    on the C's fingers and takes the pall. D \& SD change places, genuflecting
    at the center.

    \rubric During the ablution, Ac2 extinguishes the ACs candles and takes the
    chalice veil to the Gospel side of the altar.

    \rubric D goes to the credence table and takes off his broad stole and puts
    on the folded chasuble. D returns to the right of C \textit{per breviorum}
    and genuflects.

    \rubric C places the chalice on the corporal for the SD and says
    \textit{Quod ore sumpsimus}, keeping his hands joined. SD closes the
    missal, purifies and builds up the chalice and takes it to the credence
    table.

    \rubric Th and ACs line up a short distance from the foot of the altar.

    \rubric C remains at the center until the SD returns to his left. C, D and
    SD turn and descend the front steps to the floor. They genuflect on the
    floor, put on their birettas, and all process to the sacristy.

    \rubric During the recitation of Vespers, the ACs strip the altar and the
    credence table. When Vespers are finished they extinguish the six candles
    on the altar.

    \section{Exposition of the Relic of the Cross}

    }


