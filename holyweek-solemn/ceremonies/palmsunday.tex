\chap{Palm Sunday}{

\section{Preparation}

\begin{enumerate}[label=\Roman*.]

    \item At the high altar:

        \begin{enumerate}[label=\arabic*.]

            \item Crucifix covered in a violet veil, six candles of
                bleached\footcite[][154 d, p. 258.]{ml:1959} wax, lighted.

            \item Palm branches between the candlesticks.

            \item Violet antependium.

        \end{enumerate}

    \item On a special credence are the palms, covered with a violet cloth.
        This credence table is placed behind and to the gospel side\footnote{An
        ordinary performing this ceremony would have the table to his left,
        between the altar and his throne;\cite[][footnote 2, p.
        166.]{stehle} a priest would have the palms either on the in the center
        before the altar or on the floor to the epistle
        side\cite[][footnote 5, p. 11.]{hweekls} of the altar. The exact
        position of the palms should be dictated by whatever is more
        convenient.} of the faldstool.

    \item On the credence table:

        \begin{enumerate}[label=\arabic*.]

            \item Broad stole for the deacon.

            \item Violet ribbon to fasten the palm on the processional cross.

        \end{enumerate}

    \item In the sacristy:

        \begin{enumerate}[label=\arabic*.]

            \item Violet vestments, with a cope for the Bishop instead of a
                chasuble.

            \item Three violet stoles and maniples with cinctures, albs and
                amices for the chanters of the Passion.

            \item Books for the chanters of the Passion.

            \item Violet folded chasuble with cinctuers, alb and amice for the
                (subdeacon) cross-bearer.

        \end{enumerate}

\end{enumerate}

\section{Blessing of Palms.}

\rubric The procession enters as usual, but the bishop retains the miter and
crosier and proceeds to the faldstool immediately after genuflecting, where he
sits. The violet cover of the psalms is removed by an acolyte. Book and candle
bearers come before the bishop, who reads the aniphon \textit{Hosanna filio
David.}

\rubric Deacon removes the miter, bishop rises and sings \textit{Dominus
vobiscum} and the oration \textit{Deus, quim dilígere} in the ferial time.
Meanwhile the subdeacon goes to the credence, removes his folded chasuble. The
bishop then sits and recieves the miter from the deacon. 

\rubric With the usual reverences, the subdeacon goes to the usual place where
the Epistle is sung and sings the Lesson. Afterwards he kisses the bishop's
hand and puts on his folded chasuble. When the subdeacon leaevs the bishop the
book and candle bearers kneel before the bishop and he reads the Lesson,
Gradual and Gospel. The choir sings the Gradual.

\rubric The deacon takes off the folded chasuble at the sedilia, puts on the
broad stole and brings the Book of Gospels to the altar. He kisses the bishop's
ring, says the \textit{Munda cor} and sings the Gospel, all the ceremonies
prescribed for a Solemn Pontifical Mass being observed.

\rubric The deacon and subdeacon return to the sedilia and remove maniples, and
the deacon resumes the folded chasuble.

\rubric After the assistant priest has incensed the bishop, the book and candle
bearers come to the throne, and the deacon removes the miter. The bishop rises
and with \textit{joined} hands sings \textit{Dominus vobiscum,} the oration
\textit{Auge fidem,} and \textit{manibus junctis} the Preface. The choir sings
the \textit{Sanctus} in the ferial tone and the bishop recites it with his
assistants. He sings \textit{Dominus vobiscum} and (\textit{manibus junctis})
the five prayers from the Missal.

\rubric At the oration \textit{Benedic, qu\ae sumus Domine,} the thurifer and
the acolyte with the aspersorium come to the right of the faldstool. After the
prayer the bishop, standing\footnote{Martinucci says that he can sit.}, imposes
incense with the usual blessing, and sprinkles and incenses the palms. He then
sings \textit{Dominus vobiscum} and the oration \textit{Deus, qui Filium} in
the ferial tone.

\section{Distribution of Palms.}

\rubric The Bishop sits and receives his gold miter. The bishop receives a palm
from the \textit{dignior} of the clergy (assistant priest), who kisses the palm
and then the hand of the bishop. The bishop hands and palm to the deacon, who
receives it with the prescribed reverence. The linen gremiale is imposed. The
\textit{dignior} and the clergy, according to rank, receive palms.\footnote{The
manner of receiving the palm is to kiss first the palm and then the hand (ring)
of the one presenting the palm.} Meanwhile the choir sings \textit{Púeri
Hebr\ae órum.} After the distribution, the bishop washes his hands and the
linen gremiale is removed. A priest vested in violet stole distributes palms to
the people. 

\rubric Acolytes come to either side of the faldstool and stand with their
candlesticks. Miter removed, the bishop rises, and sings \textit{Dóminus
vobíscum} and the last oration \textit{Omnípotens sempitérne Deus} from the
missal, held by the book-bearer. The bishop sits, receives the miter imposes
incense and receives his palm.

\section{Procession}

\rubric The subdeacon takes the processional cross from the cross-bearer and
all line up for the procession, carrying palms in their outside hands. The
bishop, deacon and assisting priest stand at the foot of the altar, MC1 signals
all to genuflect and the church bells are pealed until the liturgical
procession has exited the church. The chanters sing all or some of the
antiphons in the Missal, according to the length of the procession. On the
return about four of the chanters enter the church and close the door.

\rubric The thurifer remains outside ans stands to the right if AC1. The
cross-bearer stands between the acolytes and turns the cucifix towards the
people\footcite[][147, p. 170.]{stehle}. The clergy separate into two divisions
and form a circle with the bishop in the middle and all face the church. The
chanters within the church turn toward the door and sing \textit{Glória laus}
which the others outside the church repeat. The verses are sung in the same
alternating manner,and either all or some of the verses are sung.

\rubric When the chanters have finished the subdeacon turns the crucifix and
with the base of the cross knocks on the door, which is opened immediately. The
procession enters the church  and \textit{Ingrediénte Dómino} is sung by the
chanters.

\rubric The bishop genuflects at the center, inline with the faldstool, and
goes to the faldstool with his ministers. The cope is removed and the bishop
vests in chasuble. Once the subdeacon has returned all three go to the foot of
the altar and Mass begins.

\section{Differences in the Mass.}

\rubric When the subdeacon sings \textit{Ut in nómine Jesu\dots infernórum,}
all kneel in their places. The bishop retains his miter and kneels before the
faldstool.

\rubric After the subdeacon has kissed the bishop's hand and received the
blessing, the bishop reads the Epistle, the Gradual, the Tract to the Passion
exclusive.\footnote{S.R.D. n. 3059, ad III.} Meanwhile, MC2 puts the missal
upon the altar at the epistle side, opened to the Passion.

\section{Reading of the Passion.}

\rubric Once the bishop has finished the Tract he recieves the crosier. The
bishop stands and with his minsters goes to the foot of the altar. Miter and
crosier away, the bishop and his minsters reverence the cross, ascend and go to
the epistle corner, where the bishop reads the Passion up to the \textit{Altera
autem die.} At \textit{Emísit spíritum} all kneel at their places. When the
bishop is about to begin the Passion, MC2 signals all to take up their palms.

\rubric Upon completing the Passion the bishop goes to the faldstool, receives
the miter, and says the \textit{Munda cor meum.} Meanwhile MC2 gives the Book
of Gospels to the subdeacon, who kneels before the bishop as before. The bishop
reads the \textit{Altera autem die.}

\rubric The Gospel movement is as usual, but with the following differenes. The
deacon removes his folded chasuble and puts on the broad stole before receiving
the Book of Gospels. The acolytes assist without candles, but have their hands
folded. The bishop does not hold the crosier but his palm.

}

%\chap{Palm Sunday}{

%\section{Preparation}

%\begin{enumerate}[label=\Roman*.]

    %\item At the high altar:

        %\begin{enumerate}[label=\arabic*.]

            %\item Crucifix covered in a violet veil, six candles of
                %bleached\footcite[][154 d, p. 258.]{ml:1959} wax, lighted.

            %\item Palm branches between the candlesticks.

            %\item Violet antependium.

        %\end{enumerate}

    %%\item On a special credence are the palms, covered with a violet cloth.
        %%This credence table is placed behind and to the gospel side\footnote{An
        %%ordinary performing this ceremony would have the table to his left,
        %%between the altar and his throne;\cite[][footnote 2, p. 166.]{stehle} a
        %%priest would have the palms either on the in the center before the
        %%altar or on the floor to the epistle side\cite[][footnote 5, p.
        %%11.]{hweekls} of the altar. The exact position of the palms should be
        %%dictated by whatever is more convenient.} of the faldstool.

    %\item On a special credence are the palms, covered with a violet cloth.
        %This credence table is placed behind and to the gospel side\footnote{An
        %ordinary performing this ceremony would have the table to his left,
        %between the altar and his throne; a priest would have the palms either
        %on the in the center before the altar or on the floor to the epistle
        %side of the altar. The exact position of the palms should be dictated
        %by whatever is more convenient.} of the faldstool.

    %\item On the credence table:

        %\begin{enumerate}[label=\arabic*.]

            %\item Broad stole for the deacon.

            %\item Violet ribbon to fasten the palm on the processional cross.

        %\end{enumerate}

    %\item In the sacristy:

        %\begin{enumerate}[label=\arabic*.]

            %\item Violet vestments, with a cope for the Bishop instead of a
                %chasuble.

            %\item Three violet stoles and maniples with cinctures, albs and
                %amices for the chanters of the Passion.

            %\item Books for the chanters of the Passion.

            %\item Violet folded chasuble with cinctuers, alb and amice for the
                %(subdeacon) cross-bearer.

        %\end{enumerate}

%\end{enumerate}

%\section{Blessing of Palms.}

%\rubric The procession enters as usual, but the bishop retains the miter and
%crosier and proceeds to the faldstool immediately after genuflecting, where he
%sits. The violet cover of the psalms is removed by an acolyte. Book and candle
%bearers come before the bishop, who reads the aniphon \textit{Hosanna filio
%David.}

%\rubric Deacon removes the miter, bishop rises and sings \textit{Dominus
%vobiscum} and the oration \textit{Deus, quim dilígere} in the ferial time.
%Meanwhile the subdeacon goes to the credence, removes his folded chasuble. The
%bishop then sits and recieves the miter from the deacon. 

%\rubric With the usual reverences, the subdeacon goes to the usual place where
%the Epistle is sung and sings the Lesson. Afterwards he kisses the bishop's
%hand and puts on his folded chasuble. When the subdeacon leaevs the bishop the
%book and candle bearers kneel before the bishop and he reads the Lesson,
%Gradual and Gospel. The choir sings the Gradual.

%\rubric The deacon takes off the folded chasuble at the sedilia, puts on the
%broad stole and brings the Book of Gospels to the altar. He kisses the bishop's
%ring, says the \textit{Munda cor} and sings the Gospel, all the ceremonies
%prescribed for a Solemn Pontifical Mass being observed.

%\rubric The deacon and subdeacon return to the sedilia and remove maniples, and
%the deacon resumes the folded chasuble.

%\rubric After the assistant priest has incensed the bishop, the book and candle
%bearers come to the throne, and the deacon removes the miter. The bishop rises
%and with \textit{joined} hands sings \textit{Dominus vobiscum,} the oration
%\textit{Auge fidem,} and \textit{manibus junctis} the Preface. The choir sings
%the \textit{Sanctus} in the ferial tone and the bishop recites it with his
%assistants. He sings \textit{Dominus vobiscum} and (\textit{manibus junctis})
%the five prayers from the Missal.

%\rubric At the oration \textit{Benedic, qu\ae sumus Domine,} the thurifer and
%the acolyte with the aspersorium come to the right of the faldstool. After the
%prayer the bishop, standing\footnote{Martinucci says that he can sit.}, imposes
%incense with the usual blessing, and sprinkles and incenses the palms. He then
%sings \textit{Dominus vobiscum} and the oration \textit{Deus, qui Filium} in
%the ferial tone.

%\section{Distribution of Palms.}

%\rubric The Bishop sits and receives his gold miter. The bishop receives a palm
%from the \textit{dignior} of the clergy (assistant priest), who kisses the palm
%and then the hand of the bishop. The bishop hands and palm to the deacon, who
%receives it with the prescribed reverence. The linen gremiale is imposed. The
%\textit{dignior} and the clergy, according to rank, receive palms.\footnote{The
%manner of receiving the palm is to kiss first the palm and then the hand (ring)
%of the one presenting the palm.} Meanwhile the choir sings \textit{Púeri
%Hebr\ae órum.} After the distribution, the bishop washes his hands and the
%linen gremiale is removed. A priest vested in violet stole distributes palms to
%the people. 

%\rubric Acolytes come to either side of the faldstool and stand with their
%candlesticks. Miter removed, the bishop rises, and sings \textit{Dóminus
%vobíscum} and the last oration \textit{Omnípotens sempitérne Deus} from the
%missal, held by the book-bearer. The bishop sits, receives the miter imposes
%incense and receives his palm.

%\section{Procession}

%\rubric The subdeacon takes the processional cross from the cross-bearer and
%all line up for the procession, carrying palms in their outside hands. The
%bishop, deacon and assisting priest stand at the foot of the altar, MC1 signals
%all to genuflect and the church bells are pealed until the liturgical
%procession has exited the church. The chanters sing all or some of the
%antiphons in the Missal, according to the length of the procession. On the
%return about four of the chanters enter the church and close the door.

%\rubric The thurifer remains outside ans stands to the right if AC1. The
%cross-bearer stands between the acolytes and turns the cucifix towards the
%people\footnote[][147, p. 170.]{stehle}. The clergy separate into two divisions
%and form a circle with the bishop in the middle and all face the church. The
%chanters within the church turn toward the door and sing \textit{Glória laus}
%which the others outside the church repeat. The verses are sung in the same
%alternating manner,and either all or some of the verses are sung.

%\rubric When the chanters have finished the subdeacon turns the crucifix and
%with the base of the cross knocks on the door, which is opened immediately. The
%procession enters the church  and \textit{Ingrediénte Dómino} is sung by the
%chanters.

%\rubric The bishop genuflects at the center, inline with the faldstool, and
%goes to the faldstool with his ministers. The cope is removed and the bishop
%vests in chasuble. Once the subdeacon has returned all three go to the foot of
%the altar and Mass begins.

%\section{Differences in the Mass.}

%\rubric When the subdeacon sings \textit{Ut in nómine Jesu\dots infernórum,}
%all kneel in their places. The bishop retains his miter and kneels before the
%faldstool.

%\rubric After the subdeacon has kissed the bishop's hand and received the
%blessing, the bishop reads the Epistle, the Gradual, the Tract to the Passion
%exclusive.\footnote{S.R.D. n. 3059, ad III.} Meanwhile, MC2 puts the missal
%upon the altar at the epistle side, opened to the Passion.

%\section{Reading of the Passion.}

%\rubric Once the bishop has finished the Tract he recieves the crosier. The
%bishop stands and with his minsters goes to the foot of the altar. Miter and
%crosier away, the bishop and his minsters reverence the cross, ascend and go to
%the epistle corner, where the bishop reads the Passion up to the \textit{Altera
%autem die.} At \textit{Emísit spíritum} all kneel at their places. When the
%bishop is about to begin the Passion, MC2 signals all to take up their palms.

%\rubric Upon completing the Passion the bishop goes to the faldstool, receives
%the miter, and says the \textit{Munda cor meum.} Meanwhile MC2 gives the Book
%of Gospels to the subdeacon, who kneels before the bishop as before. The bishop
%reads the \textit{Altera autem die.}

%\rubric The Gospel movement is as usual, but with the following differenes. The
%deacon removes his folded chasuble and puts on the broad stole before receiving
%the Book of Gospels. The acolytes assist without candles, but have their hands
%folded. The bishop does not hold the crosier but his palm.

%}
