\documentclass[letterpaper]{report}
\title{Private Consecration of a Portable Altar}

\usepackage{rubrics}

\begin{document}

\maketitle

\preface{These rubrics are for a Bishop accompanied by two chaplains, with no chant.}{

	A portable altar may be consecrated on any day, during the morning hours.
	The ceremony is performed in the church, sacristy or any other suitable
	place\footnote{Schulte, p. 232, n. 1.}.

	For a private consecration, the crozier is not used at any
	time\footnote{Moretti, p. 477, n. 3053.}.

	After the consecration, Mass is celebrated over the  consecrated altar.
	This Mass may be said by the consecrating bishop or by a priest delegated
	by him.\footnote{Menghini, p. 315, n. 4. Therefore it is not necessary that
	the Bishop himself fulfill the rubric \textit{jejuno tamen stomacho,} as is
	in the Pontificale.}
	
	As this ceremony is particularly long and complicated, and is of rare
	occurance, I have attempted to make these rubrics as complete and practical
	as possible. The personel for the complete ceremony are often lacking, so
	much research has been done to ensure that these notes provide the most
	convenient way of going about the ceremony as possible.

	These notes were first used in late 2021, for the consecration of the altar
	of Fr. Guépin (\dagger\ February 8\textsuperscript{th} 2023) in Nantes,
	France. Bp. Sanborn consecrated the altar, accompanied by the author, then
	merely tonsured, and M. l'Abbé Henry Chappot de la Chanonie, then a deacon,
	who was ordained two days after the consecration.
}

\chap{Preparation}{
\section{Setup} 
%! fix the setup
\begin{enumerate}
	\item At the main altar:
	\begin{enumerate}
		\item Stone, placed in the middle of the altar, if possible.
		\item Faldstool covered in white to the Epistle side, before the step.
	\end{enumerate}
	\item On the altar to be consecrated\footnote{If the stone to be consecrated is small, then it is placed on a table which is covered by a white cloth. The sole purpose of this cloth is to absorb any oil that may flow from the stone.}:
	\begin{enumerate}
		\item A silver tray with the relic and three grains of incense; the tray should be covered with a silk, red veil\footnote{Moretti, p. 477 n. 3053.}
	\end{enumerate}
	\item On the same table, or on another table, which is covered by a white cloth:
	\begin{enumerate}
		\item One small vessel of holy water with aspergillum, the aspergil being entwined with hyssop, or some other herbs\footnote{Schulte, p. 235 n. 6.}.
		\item One small vessel with water to be consecrated.
		\item One small vessel with the Oil of the Catechumens.
		\item One small vessel with Chrism.
		\item Small quantity of ash in a metal dish.
		\item Small quantity of salt in a metal dish.
		\item Small quantity of wine in an ampulla.
		\item One finger towel.
		\item Aspergillum.
		\item Small trowel.
		\item Dry\footnote{Martinucci, Lib. VII, cap. XVI, n. 88, footnote. The Pontifical prescribes tempering the cement with water before it is blessed. This is not feasible, and consequently it is better to keep a sufficient amount of Gregorian water in a seperate vessel to be poured over the dry cement shortly before it is to be used for closing the sepulchre.} cement for sealing the \textit{sepulchrum.}
	\end{enumerate}
	\item On the credence:
	\begin{enumerate}
		\item White stole.
		\item Cloth-of-gold miter.
	\end{enumerate}
	\item The servers:
	\begin{enumerate}
		\item Master of Ceremonies (MC)
		\item Thurifer (TH) who is a priest.
		\item Miter Bearer (MB)
		\item Two Chaplains (Chs)
	\end{enumerate}
\end{enumerate}
\section{Preparation of the Bishop}

B vested in a white stole over his rochet and the simple miter; for the private
consecration, B does not use crozier at any time.

The Bishop washes his hands\footnote{Menghini, p. 317, n. 9} before vesting.
}


\chap{Before the altar}{
\begin{enumerate}
	\item B before the altar says: \textit{Deus omnipoténtem...}
	\item Without miter, which is removed by MC or Ch, B genuflects before the altar to be consecrated, saying: \textit{Deus, in adjutórium meum inténde.} Chs respond: \textit{Dómine, ad adjuvándum me festína.} then B rises. Without miter, B says: \textit{Glória Patri...,} choir responding.
	\item This responsory is said three times, each time in a higher voice.
\end{enumerate}
}


\chap{Blessing the Water}{
\section{Exorcism of the Salt and the Water}
\begin{enumerate}
	\item B receives miter from MC or Ch.
	\item B says \textit{Exorcízo te, creatúra salis...\footnote{Note that this exorcism is particular to this ceremony.}}. If the church is not being consecrated, then in the place of \textit{hujus ecclésiæ et altáris} B reads only \textit{hujus altáris\footnote{ibid. n. 10}.}
	\item B then without miter reads the versicles and the oration.
	\item B receives the miter for the exorcism of the water, then removes it for the versicles and the oration, as before.
\end{enumerate}
\section{Blessing of the Ashes}
\begin{enumerate}
	\item Without miter, B, standing in the same place, blesses the ashes.
	\item B takes the salt and mixes it with the ashes in the form of a cross saying: \textit{Commíxtio salis, et cíneris páriter fiat. Pa \cross\ tris, et Fí \cross\ lii, et Spíritus \cross\ Sancti. \rbar Amen.}
	\item Taking a handful of the salt and ashes, B sprinkles the mixture three times into the water saying: \textit{Commíxtio salis, cíneris, et aquae páriter fiat. Pa \cross\ tris, et Fí \cross\ lii, et Spíritus \cross\ Sancti. \rbar Amen.}
\end{enumerate}
\section{Blessing of the Wine}
\begin{enumerate}
	\item Without miter, B, standing in the same place, blesses the wine; at this time, TH should prepare his coals.
	\item B takes the wine and pours it into the water in the form of a cross saying: \textit{Commíxtio vini, salis, cíneris et aquae páriter fiat. Pa \cross\ tris, et Fí \cross\ lii, et Spíritus \cross\ Sancti. \rbar Amen.}
	\item Without miter B reads the oration: \textit{Omnípotens sempitérne Deus...} If the church is not being consecrated, then in the place of \textit{hujus ecclésiæ et altáris} B reads only \textit{hujus altáris.}
\end{enumerate}
}


\chap{Signing the Stone}{
\section{Signing with the water}
\begin{enumerate}
	\item With miter, B takes the Gregorian water with the thumb of his right hand and makes a cross in the middle of the stone saying: \textit{Sancti \cross\ ficétur, et conse \cross\ crétur haec tábula,} blessing with his right hand in the usual manner at \textit{In nómine Pa \cross\ tris, et Fí \cross\ lii, et Spíritus \cross\ Sancti. Pax tibi.} Note that one cross is traced at the word \textit{Sanctificétur} and a second cross is traced at the word \textit{Consecrétur.}\footnote{Schulte, p. 244, n. 19.}
	\item This signing is repeated in each corner of the altar, first on the Gospel side, back, then on the Epistle side, front, then on the Gospel side, front, then the Epistle side, back.
\begin{center}
	\begin{tabular}{ | l c r | }
		\hline
		\cross\ \textbf{2} &                      & \textbf{5} \cross\ \\
							 & \cross\ \textbf{1} &                      \\
		\cross\ \textbf{4} &                      & \textbf{3} \cross\ \\
		\hline
	\end{tabular}
\end{center}
\item B wipes his thumb in a towel provided by MC, and is then handed the aspergillum.
\item B intones the antiphon \textit{Aspérges me} which he recites with Chs. Psalm 50 is then recited, followed by the antiphon; \textit{Glória Patri} is omitted.
\item ``During the recitation of the \textit{Aspérges} and \textit{Miserére} B sprinkles the stone around the edge, beginning at the middle, in front, then at his right, at the back and at his left, and finally, in front, to the middle. In this manner he sprinkles the stone three times.''\footnote{Schulte, p. 246 n. 20.}
\end{enumerate}
\section{Blessing of the Cement}
\begin{enumerate}
	\item The stone is wiped clean with a linen cloth by one of the chaplains; TH should be at hand at this point.
	\item B, without miter says the versicles and then the oration \textit{Deus, qui es visibilium.}
	\item MC or a Ch presents to B the dry cement which B blesses\footnote{This cement is blessed by the rite prescribed for the consecration of a fixed altar, even though there is no mention of this in the Pontifical, S.R.C., May 10, 1890, n. 3726 ad I: \textit{...Dubium I. An caementum prò firmando in Altari portatili sepulcri lapide benedicendum sit ritu prò Altaris fixi consecratione praescripto? Affirmative.}}, saying: \textit{Summe Deus...}
\end{enumerate}
}


\chap{Incensation of the Stone}{
\begin{enumerate}
	\item B imposes incense and blesses it with the usual formula.
	\item B receives miter, censor
	\item B incenses the stone while reciting \textit{Dirigatur oratio mea sicut incensum in conspectu tuo, Domine} with the chaplains. The stone is incensed three times in the manner in which he sprinkled it before.
	\item TH continuously\footnote{Martinucci, Lib. VI., Cap. XX, n. 14, foot-note.} incenses the stone during the following ceremonies\footnote{Schulte, p. 248, footnote: ``If only one stone is being consecrated, the incensing priest remains standing near the stone. If several stones are being consecrated, he may either remain standing in one place, or he may move around the table on which the stones are placed.''}.
\end{enumerate}
}


\chap{Anointing with Holy Oil}{
\section{First anointing}
\begin{enumerate}
	\item After the incesation, B intones the antiphon \textit{Erexit Jacob.} The ministers continue the antiphon with B and recite Psalm 83 alternately with him, the \textit{Glória Patri} begin omitted; the antiphon is repeated.
	\item After the recitation of the antiphon and the psalm, B stands with miter, takes some Oil of the Catechumens on his right thumb, and signs the stone five times with clear crosses, in the same manner as before with the Gregorian water. 
	\item The same formula is pronounced for each cross as before: \textit{Sancti \cross\ ficétur, et conse \cross\ crétur haec tábula, In nómine Pa \cross\ tris, et Fí \cross\ lii, et Spíritus \cross\ Sancti. Pax tibi.}
\end{enumerate}
\section{First incensation}
\begin{enumerate}
	\item B cleanses his finger.
	\item TH receives incense from B, which is blessed in the usual manner.
	\item B and CHs recite antiphon \textit{Dirigatur...} while B incenses altar as before, but this time only once.
	\item B retuns censor to TH, who continues incensing.
	\item After the conclusion of the antiphon, B, without miter, says \textit{Orémus,} the ministers answering \textit{Flectámus génua, Leváte.} B recites the oration \textit{Adsit, Dómine.}
\end{enumerate}
\section{Second anointing}
\begin{enumerate}
	\item B intones the antiphon \textit{Mane surgens Jacob;} Chs recite the antiphon and Psalm 91 as before, the \textit{Glória Patri} begin omitted.
	\item After the recitation of the antiphon and the psalm, B stands with miter, takes some Oil of the Catechumens on his right thumb and again anoints the stone, as before.
\end{enumerate}
\section{Second incensation}
\begin{enumerate}
	\item B cleanses his finger.
	\item TH receives incense from B, which is blessed in the usual manner.
	\item B and CHs recite antiphon \textit{Dirigatur...} while B incenses altar as before, but this time only once.
	\item B retuns censor to TH, who continues incensing.
	\item After the conclusion of the antiphon, B, without miter, says \textit{Orémus,} the ministers answering \textit{Flectámus génua, Leváte.} B recites the oration \textit{Adésto, Dómine.}
\end{enumerate}
\section{Third anointing}
\begin{enumerate}
	\item B intones the antiphon \textit{Unxit te, Deus;} Chs recite the antiphon and Psalm 45 as before, the \textit{Glória Patri} begin omitted.
	\item After the recitation of the antiphon and the psalm, B stands with miter, takes some Chrism on his right thumb and again anoints the stone, as before.
\end{enumerate}
\section{Third incensation}
\begin{enumerate}
	\item B cleanses his finger.
	\item TH receives incense from B, which is blessed in the usual manner.
	\item B and CHs recite antiphon \textit{Dirigatur...} while B incenses altar as before, but this time only once.
	\item B retuns censor to TH, who continues incensing.
	\item After the conclusion of the antiphon, B, without miter, says \textit{Orémus,} the ministers answering \textit{Flectámus génua, Leváte.} B recites the oration \textit{Exáudi nos, Deus noster.}
	\item During this prayer\footnote{Schulte, p. 256, n. 31}, one of the attendants adds some of the Gregorian water to the cement and mixes it\footnote{The cement should be about the same consistency as toothpaste.}.
\end{enumerate}
}


\chap{Anointing the Confession}{
\begin{enumerate}
	\item Having concluded the oration, B with miter takes Chrism with his right thumb and anoints the confession in the inside saying \textit{Conse \cross\ crétur, et sancti \cross\ ficétur hoc sepélcrum. In nómine Pa \cross\ tris, et Fí \cross\ lii, et Spíritus \cross\ Sancti. Pax huic domui.} As with the previous unction, B traces crosses at the words \textit{Consecrétur} and \textit{Sanctificétur.}
	\item B, without miter, places the relics and three grains of incense\footnote{These grains shoul be very small. Note that the Pontifical seems to iterchangeably use the terms \textit{franckincense} and \textit{incense}, but the authors only mention the use of incense.} in the sepulcher. The dish with the prepared cement is brought to the B.
	\item B uses a small trowel to spread the cement over the ledge on the inside of the sepulcher and then covers the sepulcher with the small slab. Someone may assist B with the cement.
	\item The Chs immediately remove any cement that may remain on the top of the stone.
	\item B stands without miter and says: \textit{Orémus. Deus, qui ex ómnium cohabitatióne...}
\end{enumerate}
}


\chap{Pouring of the Oils}{
\begin{enumerate}
	\item B intones the antiphon \textit{Ecce odor;} 
	\item B resumes the miter\footnote{Schulte, p. 259, n. 35}.
	\item B and Chs continue the antiphon, recite Psalm 86 alternately, the \textit{Glória Patri} begin omitted, and repeat the antiphon.
	\item After the psalm B takes cruets containing Chrism and Oil of Catechumens (either both in his right hand or one in each hand) and coinjointly pours these oils on the stone.
	\item B passes the cruets to Ch1, draws back the sleeves of his right arm, and with the palm of his right hand rubs the holy oils over the entire surface of the stone.
	\item B then purifies his hand.
	\item B retaining miter says: \textit{Lápidem hunc...}
	\item B intones the \textit{Aedificavit Moyses} which he continues with Chs.
	\item Afterwards, B with miter says: \textit{Dei Patris omnipoténtis...}
\end{enumerate}
}


\chap{Burning of Incense}{
\begin{enumerate}
	\item Ch2 holds\footnote{This is not placed on the altar because of the oils.} salver with grains of incense to be burned.
	\item B stands without miter says the versicles and the oration \textit{Dómine, Deus omnípotens...}
	\item B sprinkles the incense (middle, left, right) with holy water and resumes the miter.
	\item With his own hand B makes\footnote{S.R.C., Jan 14, 1910, n. 4244 ad IV: textit{Episcopum consecrantem in praedictis actionibus posse adiuvari a Sacerdotibus.}} five crosses of five grains; each cross is placed over the one of the five spots previously anointed. Over each cross of incense, as soon as it is made, he places a cross of thin candle (taper) over the incense\footnote{The five grains of incense may be attached to the taper-crosses beforehand, and then the bishop needs only to place the taper-crosses at their places with the grains of incense turned downwards.— Martinucci, Lib. VII, cap. XVI, n. 112, foot-note.}.
	\item \item MC lights the four ends of each cross once the fifth one is made\footnote{Schulte, p. 261, n. 39}.
	\item The duty of the incensing priest not ceases and the thruible is returned to the TH.
	\item All of the crosses having been lit, B kneels with bared head before the altar and recites with the ministers the verse: \textit{Allelúia\footnote{The \textit{Allelúia} being omitted if the ceremony take place between Septuagesima and Easter.}. V. Veni, Sancte Spíritus,...}
	\item After the versicle is concluded, B stands before the altar, receives zuchetto, not miter, and recites with Chs the antiphons \textit{Ascéndit fumus} and \textit{Stetit Angelus.}
	\item After the conclusion of the antiphons, B says \textit{Orémus,} the Chs answering \textit{Flectámus génua, Leváte.} B recites the oration \textit{Hujus altáris.}
\end{enumerate}
}


\chap{Preface}{
\section{Singing the Preface}
\begin{enumerate}
	\item After the grains are consumed, Ch removes the burt incense and wax from the stone with a spatula. These remains are later depoited in the secrarium.
	\item B recites the oration \textit{Súpplices tibi.}
	\item After the oration, B, in the same place and with hands extended, reads the preface in the middle voice.
	\item He reads the conclusion \textit{Per Dominum nostrum...} in a lower voice, so it can be heard only by the bystanders.
\end{enumerate}
\section{After the Preface}
\begin{enumerate}
	\item B intones and then recites with the Chs the antiphon \textit{Confírma hoc,} which is repeated.
	\item B reads the oration: \textit{Quaesumus omnípotens Deus.}
	\item TH presents the thurible; B imposes incense and blesses it in the usual manner. B receives the miter and the thurible.
	\item B intones and then recites with the Chs the antiphon \textit{Omnis terra adoret.}
	\item B incenses the altar in the form of a cross while he recites the antiphon\footnote{\textit{Moretti,} p. 479}.
	\item B without miter says oration \textit{Descéndat, quaesumus.}
\end{enumerate}
}


\chap{Conclusion}{
\begin{enumerate}
	\item After the oration, B proceeds to the faldstool, where he first purifies his hands and washes them, then removes the miter, stole and pectoral cross and resumes the mantelletta with the pectoral cross.
	\item Meanwhile, the altar is cleansed and prepared for Mass.
	\item If B is to say the Mass, then he prepares himself; otherwise another priest may prepare for Mass.
	\item The Mass is of the day of the dedication of an altar.
\end{enumerate}
}


\references{
\begin{itemize}
	\item Menghini, I. B. M. \textit{Manuale Sacrarum Cæremoniarum, vol. II.} Editio terita. Ratisbonæ - Romæ - Neo Eboraci: Fredericus Pustet, 1915
	\item Benedicto XIV. \textit{Pontificale Romanum, pars prima.} Mechliniæ: H. Dessain, 1862
	\item Moretti, Aloisius. \textit{De Sacris Functionibus, vol. IV.} Turin, Marii E. Marietti, 1938
	\item Schulte, Rev. A. J. \textit{Consecranda.} New York - Cincinnati - Chicago: Benziger Brothers, 1907
\end{itemize}
}
\end{document}
