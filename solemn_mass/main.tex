\documentclass[letterpaper, twocolumn]{article}
\title{Solemn Mass}
\author{Gregory R. Barnes}

\usepackage{enumitem}
\usepackage{rubrics}

\begin{document}
	\maketitle

	\section{Preparation}

	\begin{enumerate}[label=\Roman*.]
		\item Altar
			\begin{enumerate}[label=\arabic*.]
				\item Missal on the Epistle side open to the Introit.
				\item Antependium and tabernacle veil the color of the office.
				\item Altar cards.
				\item Six candles of bleached wax, lit.
			\end{enumerate}
		\item Epistle celebrantredence
			\begin{enumerate}[label=\arabic*.]
				\item Chalice, covered in humeral veil the color of the Mass, rather than the chalice veil as in low Mass.
				\item Chalice veil is placed behind the chalice.
				\item Cruets.
				\item Small dish and finger towel for Lavabo.
			\end{enumerate}
	\end{enumerate}

	\section{The Procession.}

	\rubric After ministers are vested, the thurifer receives incense from the
	celebrant, deacon ministering, asking for the blessing.

	\rubric Once all is prepared MC says \textit{Proce damus,} and the
	procession begins.

	\section{From the Confession to the Introit.}

	\rubric All servers/choir members genuflect upon arriving at the altar.
	Choir members, if they be processing two by two, bow slightly to each other
	after reverencing the cross, and then turning towards one another, go to
	their places in choir.

	\rubric Subdeacon and deacon flank celebrant at the foot of the altar.
	Subdeacon and deacon remove their birettas before the celebrant, deacon
	taking celebrant's biretta.

	\rubric MC takes birettas from deacon, then from subdeacon, not genuflecting as he
	crosses the center.

	\rubric MC signals a genuflection for the ministers only, genuflecting with
	them. MC signals for all to kneel, except for the ministers, who begin the
	prayers at the foot of the altar.

	\rubric MC places the birettas at the sedillia, then joins the ministers
	for the prayers, kneeling at the right a slightly behind deacon, responding with
	the subdeacon and deacon. Meanwhile an acolyte should close the sanctuary gates.

	\rubric After \textit{Orémus} celebrant ascends the altar, subdeacon and
	deacon helping with his alb, themselves ascending with him. MC signals for
	all to rise.

	\rubric The celebrant kisses the altar. Meanwhile, MC signals the TH to
	receive incense from celebrant, deacon assisting and asking for the
	blessing in the usual manner.

	\rubric As incense is being imposed, MC takes the missal from the altar and
	holds it, standing to the Epistle side. 

	\rubric After the imposition, thurifer hands thurible to deacon, takes the
	boat from deacon and joins MC, standing to his left. The celebrant incenses
	the altar, deacon to his right and subdeacon to his left.

	\rubric When celebrant arrives at the Epistle side, completing the
	incensation, deacon and subdeacon stand before him, facing the Gospel side;
	deacon receives the thurible, then both decend to the floor, turning
	towards one another while descending. MC stands behind deacon, thurifer
	stands behind subdeacon.

	\rubric The deacon incenses celebrant with three doubles, deacon,
	subdeacon, MC and thurifer bowing profoundly before and after.

	\rubric The celebrant stands before the missal with MC to his right; deacon
	is behind and to the right of the celebrant, and subdeacon and behind and
	to the right of the deacon. This is the \textit{Introit Position.} The
	celebrant read the Introit in a low tone.

	\section{The Singing of the Epistle and Gospel.}

	\rubric The celebrant says the \textit{Kyrie} in the same place,
	alternating with the deacon and subdeacon. If there is time to sit the
	celebrant and his ministers go \textit{per breviorem} to the sedillia.
	Other wise they remain in the same place, ministers standing \textit{unum
	post alium.}

	\rubric Towards the end of the \textit{Kyrie} the celebrant goes to the
	center of the altar. If he is coming from the faldstool, he and the
	ministers rise, go to the foot of the altar, genuflect, and ascend,
	ministers fist assisting with the celebrant's alb, then standing
	\textit{unum post alium.} If coming from the missal, the celebrant and
	ministers together go to the center of the altar, the ministers
	\textit{unum post alium.}

	\rubric If there be a \textit{Gloria} the celebrant intones and then the
	ministers ascend and flank him. They recite the \textit{Gloria} and then
	genuflect and go \textit{per breviorum} to the sedillia, where they sit
	until the name of Our Lord occurs for the final time. Then they proceed to
	the center of the altar as described above. \textit The ministers stand
	\textit{unum post alium,} and the celebrant kisses the altar, turns and and
	says \textit{Dóminus vobíscum.}

	\rubric If there be no \textit{Gloria,} the celebrant kisses the altar,
	turns and says \textit{Dóminus vobíscum.} 

	\rubric The celebrant and ministers return to the missal, where the
	celebrant reads the orations. At the beginning of the final oration, MC
	signals the deacon to attend to the missal, and the MC gets the book of
	Epistles. MC hands\footnote{The manner of handing the book is as follows:
	the one handing the book (held with the leaves of the book to the right)
	bows to the receiver and passes the book to him, the receiver then bowing
	to the one who just handed it.} the book to the subdeacon and stands at his
	left side for the remainder of the oration. As soon as Our Lord's name has
	been said (or one the \textit{Qui tecum,} etc. oration has begun), the MC
	and subdeacon go to the center, genuflect, bow to the choir, and stand to
	the epistle side, where the epistle usually is sung.

	\rubric Once the celebrant begins to read the epistle, MC signals subdeacon
	to begin singing it. When finished the subdeacon returns to the center with
	the MC, bows to the choir and genuflects, then goes to the epistle side and
	kneels before the celebrant who blesses him and then places his hand on the
	epistle for the subdeacon to kiss. Meanwhile the deacon stands back a few
	feet to make room for the subdeacon.

	\rubric The subdeacon return to the floor and return the book to the MC.
	Subdeacon then moves the missal to the gospel side, genuflecting on the
	bottom step as he passes the center. The celebrant says the \textit{Munda
	cor} at the center and reads the gospel, subdeacon attending. Once the
	gospel is begun the deacon, who had turned to face the missal, goes to the
	floor and faced the MC, who approaches the deacon with the book of gospels.

	\rubric MC passes the book to the deacon, who goes to the center with the
	MC at his left, genuflects, and ascends to place the book on the center of
	the altar, leaves to the left, and then stands behind the celebrant, facing
	the missal, some steps back. The MC stands one pace back from his previous
	position.

	\rubric The subdeacon responds \textit{Laus tibi Chríste} after the gospel.
	The thurifer at this time ascends to the center of the altar where he meets
	the celebrant and the deacon and incense is imposed. The acolytes with
	candles join the MC at the foot of the atlar, standing behind him. The
	thurifer stands at the MC's right after receiving incense, and the
	subdeacon stands before the MC. The deacon kneels on the edge of the
	footpace and reads the \textit{Munda cor.} Afterwards the deacon stands,
	gets the book, turns to the gospel side to face the celebrant, and kneels
	for a blessing saying \textit{Jube Domne benedícere.} The celebrant blesses
	the deacon and then puts his hand on the book so that the deacon can kiss
	it. Then the deacon rises and stands to the right of the subdeacon. MC
	signals all to genuflect, bow to the choir, and then all turn inward to go
	to where the gospel will be sung.

	\rubric The acolytes stand facing the epistle side with the subdeacon
	between them; the deacon stands before the subdeacon, to whom he passes the
	book, and he has the MC to his right and thurifer to his left. The deacon
	introduces the Gospel and signs himself. MC recievs the thurible and passes
	it to the deacon, who incenses the text with three doubles, center, left
	and right. The deacon passes the thurible back to the MC and sings the
	gospel. Shortly before the occurance of our Lord's name, the MC turns
	slightly to the celebrant, so that he can turn to the cross to bow.

	\rubric Once the gospel is concluded, the deacon points out the pericope,
	and the subdeacon point to it, and keeps pointing to it until he approaches
	the celebrant. The subdeacon and the MC go by the shortest path to the
	celebrant, who kisses the pericope. The subdeaon then bows to the celebrant
	and goes to the floor on the epistle side where he gvies the book book back
	to the MC. Meanwhile the D, with thurifer to his left and the acolytes
	behind, goes to the center, bows to the choir, genuflects, receives the
	thurible and incenses the celebrant with three doubles. The deacon and
	subdeacon then resume \textit{unum post alium} positions as the rest of the
	gospel procession returns to their positions.

	\rubric If the \textit{Credo} be read, then the same proceedure is followed
	as for the \textit{Gloria,} the celebrant and ministers rising at
	\textit{expécto.} If the the \textit{Et incarnátus est} will be sung while
	the ministers are in transit, then they bow at the altar and kneel on the
	edge of the footpace.

	\rubric If there be no \textit{Credo,} or after returning to the altar, the
	minister stand \textit{unum post alium,} the celebrant kisses the atlar and
	says facing the people \textit{Dóminus vobíscum.} The he sings
	\textit{Orémus} and reads the offertory verse.

	\section{Offertory}

	\rubric 

	\section{Distribution of Communion.}

	\section{The Ablutions.}

	\section{From the Postcommunions to the Last Gospel.}

	\section{Conclusion of the Rite.}

	\section{The \textit{Asperges} or \textit{Vidi Aquam}.}

	\section{The Sermon.}

	\section{Folded Chasubles.}

	\rubric The ministers wear folded chasubles\footnote{The folded chasuble is
	not a special vestment, like a tunic or dalmatic, but is a chasuble, such
	as is worn by the priest at Mass, but with the front of the chasuble pinned
	up. Historically, this comes from the practice of the minsters of pinning
	or lashing up their vestments as they went from church to church in Rome.}
	on various penitential days throughout the liturical year. If there are no
	folded chasubles available, the ministers do not wear tunic and dalmatic,
	but the alb only.

	\rubric The minsters never sing with the folded chasubles. Before receiving
	the book for the epistle, the subdeacon removes his chasuble, and dons it
	again after he receives the blessing from the celebrant. The deacon removes
	his folded chasbuble before receiving his book for singing the gospel, and
	he puts on the broad stole. The deacon does not remove the broad stole
	until after the second ablutions.

	\section{The singing of the Passion.}

	\section{The reading of the Passion.}
	
	%\bibliography

\end{document}
