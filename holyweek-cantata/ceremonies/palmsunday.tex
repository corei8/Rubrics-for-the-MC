\chap{Palm Sunday}{

	\section{Preparation}

	\begin{enumerate}[label=\Roman*.]

		\item At the high altar:

			\begin{enumerate}[label=\arabic*.]

				\item Crucifix covered in a violet veil, six candles of
					bleached\footcite[][154 d, p. 258.]{ml:1959} wax, lighted.

				\item Palm branches between the candlesticks.

				\item Violet antependium.

			\end{enumerate}

        \item On a special credence are the palms, covered with a violet cloth.
            This credence table is placed on the floor to the epistle
            side\footcite[][footnote 5, p. 11.]{hweekls}, next to the altar.

		\item On the credence table is a violet ribbon to fasten the palm on
			the processional cross.

			\begin{enumerate}[label=\arabic*.]

				\item Violet ribbon to fasten the palm on the processional cross.

				\item Ewer and basin for washing the C's hands after the
					distribution.

			\end{enumerate}
		\item In the sacristy:

			\begin{enumerate}[label=\arabic*.]

				\item Violet vestments, with a cope for the priest instead of a
					chasuble.

				\item Three violet stoles and maniples with cinctures, albs and
					amices for the chanters of the Passion.

				\item Books for the chanters of the Passion.

				\item Violet folded chasuble with cinctures, alb and amice for
					the (subdeacon) cross-bearer.

			\end{enumerate}

	\end{enumerate}

	\section{Blessing of Palms.}

    \rubric The \textit{Asperges} ceremony proceeds as usual, but with the
    omission of the \textit{Gloria Patri}. After the \textit{Asperges}, TH
    prepares his thurible as usual.

	\rubric C ascends the altar after the \textit{Asperges} as the schola begins
	singing the \textit{Hosanna filio David}. C kisses the altar and goes to
	the Missal at the Epistle side. C reads the antiphon \textit{Hosanna filio
	David}.

    \rubric After the schola is finished singing, C reads the Oration, then
    sings the Lesson, and then reads the two Responsories that follow.

	\rubric C keeps his hands joined and bows profoundly to the cross as he
	says the \textit{Munda cor} and the \textit{Dominus sit}. He sings the
	Gospel, kissing the Missal as usual.

	\rubric C sings the oration \textit{recto tono} and then the Preface,
	keeping his hands joined. The bells are not rung by the acolytes.

	\rubric C sings the five orations of the blessing of the palms. C places
	his left hand on the altar when he makes the sign of the cross over the
	palms, and the MC has to hold the edge of his cope. At the beginning of the
	fifth oration, Ac1 takes the aspersorium and stands to the left of the TH.

	\rubric At the end of the fifth oration, C imposes incense, sprinkles the
	palms with holy water, and then incenses them. C goes to the center of the
	altar.

	\section{Distribution of Palms Without Another Priest}

    \rubric MC places the C's palm on the \textit{mensa} at the center. C picks
    it up, kisses it, and gives it back to the MC, who passes it to Ac1, who
    places it on the credence table. Meanwhile, all of the servers line up to
    receive palms.

	\section{Distribution of Palms With Another Priest}

	\rubric MC takes the C's palm. C turns towards the people as soon has he
	has reached the center. The senior clergyman goes to the center,
	genuflects, and ascends to the top step. He does not wear a stole. MC hands
	the palm to the senior clergyman who presents it to the C. C kisses it,
	does not kiss the clergyman's hand, and takes the palm, which he hands to
	the MC immediately. C then passes the palms to the servers (who lined up
	meanwhile), but first to the senior clergyman and the rest of the clergy.

	\rubric Servers then ascend to the top step two by two and receive the
	palms from C, kissing first the palm and then the C's hand. Each pair then
	rises, descends the steps, genuflects and each goes back to his place.

	\rubric After the distribution at the altar, C distributes the palms to the
	faithful at the communion rail, MC to his right, and ACs supplying MC with
	plenty of palms. Ac2 ties a palm to the processional cross. Another priest
	in surplice and stole may assist in distributing palms.

	\rubric After the distribution, C goes to the center of the altar and
	genuflects on the floor. C then goes to the Epistle corner and washes his
	hands. C goes \textit{per breviorem} to the Missal, and MC joins him,
	walking \textit{per longiorum}.

	\rubric C sings \textit{Dominus vobiscum} and the final Oration
	\textit{Omnipotens sempiterne Deus}.

	\section{Procession}

	\rubric C imposes incense. Afterwards MC goes to the credence table and
	gets the C's palm and biretta, and the procession forms as usual, everyone
	carrying palms either in the right or in the outside hand.

    \rubric MC hands the C his palm, who turns to face the people after the
    procession is formed. C sings \textit{Procedamus in pace} and the schola
    answers \textit{In nomine Christi. Amen.} At this signal, Th begins the
    procession. One of the church bells is pealed while the procession exists
    the church.

	\rubric C and MC go to the center of the altar, descend to the floor,
	genuflect, and then join the end of the procession, MC passing the C his
	biretta.

	\rubric All or some of the antiphons in the missal may be sung by the
	schola.

	\section{Gloria Laus}

	\rubric When the procession returns to the vestibule, CB, ACs and Th step
	to the left. Two or four of the chanters enter the church\footnote{These
		are the numbers mentioned by the \textit{Missale Romanum}, but a larger
	number is not forbidden.} and the CB and ACs stand before the door. Th stands
	to the right of Ac1. The schola stands behind the CB and ACs, facing the
	door.\footnote{If the schola consist of women or religious Sisters, they stand
		behind the C, or at least outside of the procession, if standing behind the
	C will prevent them from hearing the chanters inside the church.} The rest of
	the procession lines up behind the schola, all facing the door. Ushers should
	be close to the front of the procession, to assist with the doors.

	\rubric With the doors closed, the chanters inside of the church begin the
	\textit{Gloria laus}. After every two verses, the schola outside repeats
	the first two verses. Not all of the hymn need be sung.\footnote{The exact
	stopping point should be determined beforehand.} When the chanters inside
	have stopped singing, CB knocks the door with the foot of the cross and the
	door opens from the inside. CB and ACs must be prepared to move aside in an
	orderly fashion as the ushers assist with opening the doors all the way.

	\rubric As the procession enters the church, TH again leading, the schola
	sings the responsory \textit{Ingrediente Domino} without the \textit{Gloria
	Patri}.

	\rubric All genuflect and go to their places as they enter the sanctuary. C
	goes to the foot of the altar with the MC, removes his biretta, genuflects,
	and goes to the sedilia. C gives his palm to the MC, removes the cope and
	puts on his maniple and chasuble. C and MC go to the foot of the altar as
	usual for the prayers at the foot of the altar.

	\section{Differences in the Mass.}

	\rubric The Mass must be celebrated by the one who blessed the palms.

	\rubric Omit \textit{Judica me} and the \textit{Gloria Patri} after the
	Introit and the \textit{Lavabo}.

	\rubric At the Epistle, all genuflect on one knee for the words \textit{ut
	in nomine Jesu\dots infernorum}.

	\rubric After the Gradual and Tract (either after reading or after the
	schola has finished chanting them) C goes to the center while the Missal is
	being transferred, and then immediately to the Gospel corner, without
	saying the \textit{Munda cor}. C reads the first part of the Passion in a
	loud tone. After the words \textit{emisit spiritum} C kneels, and everyone
	with him at MC's signal, facing the altar for about the length of a
	\textit{Pater noster}. MC gives the signal for all to rise and C continues
	up to the \textit{pars Evangelii} of the Passion. 

    \rubric C then goes to the center of the altar, imposes incense, and says
    the \textit{Munda cor} in the usual manner. The Gospel procession takes
    place in the usual manner, except the ACs carry palms instead of candles,
    and MC remains with C, to his left. C then goes back to the Gospel side and
    sings the \textit{pars Evangelii} of the Passion in the usual Gospel
    tone\footnote{S.R.C. 4031, 2.}, without singing the \textit{Dominus
    vobiscum} nor signing the Missal.

	}
