\documentclass[letterpaper]{report}
%\documentclass[english, twoside, oldfontcommands, 10pt]{memoir}
\title{Corpus Christi}

\usepackage{rubrics}
\usepackage{booktabs}
\usepackage{csquotes}
\usepackage[backend=biber,style=reading,style=verbose-ibid]{biblatex}
\addbibresource{~/Library/texmf/bibtex/bib/rubrics.bib}
\usepackage{enumitem}

\usepackage{import}

\newcommand{\pbr}[0]{\textit{per breviorum}}

\begin{document}

\maketitle

\singleChap{Corpus Christi}{

\section{Preparation}

\begin{enumerate}

    \item Two hosts are prepared, both on the Mass paten, to be consecrated together.

    \item Monstrance for the procession on the Epistle credence, veiled.

\end{enumerate}

\section{Beginning of the Ceremony}

\rubric AP ensures that SD takes his place immediately after the consumption of
the host. 

\rubric At the consumption of the Precious Blood, MC1 gets the veiled
monstrance and stands on the floor to the Epistle side. After the consumption
of the Precious Blood, MC1 brings the monstrance to the altar, removes the veil
and takes it to the credence table. MC should take care that he genuflects
before the after going to the altar. 


\rubric SD covers the chalice with the pall and D \& SD change
places.\footnote{Since the Blessed Sacrament is on the mensa, both genuflect in
place, switch sides immediately behind the celebrant, deacon taking the inside,
and genuflect again.} D moves the chalice to the left side of the corporal and
places the monstrance in the center of the corporal, open, facing B. 

\rubric B and ministers genuflect. B takes the host and places it in the
lunette. B puts the lunette in the monstrance, and D closes the monstrance and
turns it so that it faces the people, uncovered\footnote{S.R.C. 2990.}.

\rubric Mass continues, the ceremonies prescribed \textit{coram Sanctissimo
exposito} being observed.

\rubric TBs remain in their places after Communion. THs prepare coals at the
Communion Verse.

\rubric B does not receive his zucchetto after Communion.\footnote{It seems
that B's hands would be washed on the floor as on Holy Thursday. This being the
case, the zucchetto is used only for this hand washing and must be removed
afterwards.}

\rubric After the Last Gospel, B, D \& SD go to the center, genuflect and go to
the faldstool \pbr. B receives his zuchetto, miter and crosier once his has
stepped off the predella. D \& SD assist B in removing his chasuble and maniple
and in putting on the cope. D \& SD remove their maniples.

\rubric While the ministers are vesting, ACs and CB stand at the front of the
sanctuary. THs enter with lit coals.

\rubric Once the ministers are ready, TH2 approaches and recives incense in his
thurible.\footnote{It appears that B does this sitting and with miter.} TH1
receives incense afterwards and TH2 leads the THs and ministers to the foot of
the altar where they line up THs to the outside, and all make a double
genuflection \textit{in plano} and then the ministers kneel on the bottom step.

\rubric TH1 passes the thurible to D, who passes it to B, who incesnes the
Blessed Sacrament with three double swings, D \& SD holding the cope, all
bowing before and after. D returns the thuribile to TH1.

\section{Procession}

\rubric MC1 places the humeral veil on B's shoulders. Ministers rise together:
B \& SD stand on the step below the footpace, but D goes up to the altar. B \&
SD kneel on the top step when D genuflects in the usual manner, off to the
Epistle side and facing the Cross. D take the monstrance and hands it to B, who
receives it kneeling on the top step. D genuflects.

\rubric B rises and faces the people, turning to his right. D \& SD change
places behind B and stand on the top step with him, facing the people, holding
the ends of the cope. The chanters intone \textit{Pange Lingua} while everyone
is still kneeling. UB prepares his umbrella.

\rubric Immediately after the intonation, MC1 signals all rto rise, turn and
process out of the church. The choir may sing hymns found in the Roman Ritual
(IX, V), but hymns in the vernacular are forbidden\footnote{S.R.C. 3975 ad 5}.

\rubric Birettas are not worn in the procession. Bishops follow the
canopy\footnote{I recall that prelates process behind the canopy even when
another bishop is celebrating, but cannot find the reference.}

\section{Benediction}

\rubric Upon entering the church, all genuflect to the cross, go to their
places and kneel immediately.

\rubric When near the altar, D kneels before B. B hands the monstrance to D,
genuflects with SD and rises as D rises. D ascends the altar and places the
monstrance on the tabor,\footnote{During the procession, the sacristan may put
    a tabor on the altar for use during Benediction. If tonsured, he may spread
    the corporal on the tabor.} with the prescribed genuflections.

\rubric B \& SD kneel on the lowest step, MC1 removes the humeral veil from
B. \textit{Tantum Ergo} is repeated.

\rubric The Blessed Sacrament is incensed and Benediction is given in the usual
manner.

}


\printbibliography

\end{document}
