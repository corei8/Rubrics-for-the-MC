\documentclass[letterpaper]{report}
%\documentclass[english, twoside, oldfontcommands, 10pt]{memoir}
\title{Corpus Christi}

\usepackage{rubrics}
\usepackage{booktabs}
\usepackage{csquotes}
\usepackage[backend=biber,style=reading,style=verbose-ibid]{biblatex}
\addbibresource{~/Library/texmf/bibtex/bib/rubrics.bib}
\usepackage{enumitem}

\usepackage{import}

\newcommand{\pbr}[0]{\textit{per breviorum}}

\begin{document}

\maketitle

\section{Preparation}

\begin{enumerate}

    \item Two hosts are prepared, both on the Mass paten, to be consecrated together.

    \item Monstrance for the procession on the Epistle credence, veiled.

\end{enumerate}

\section{Beginning of the Ceremony}

\rubric AP ensures that SD takes his place immediately after the consumption of
the host. 

\rubric At the consumption of the Precious Blood, MC1 gets the veiled
monstrance and stands on the floor to the Epistle side. After the consumption
of the Precious Blood, MC1 brings the monstrance to the altar, removes the veil
and takes it to the credence table. MC should take care that he genuflects
before the after going to the altar. 


\rubric SD covers the chalice with the pall and D \& SD change
places.\footnote{Since the Blessed Sacrament is on the mensa, both genuflect in
place, switch sides immediately behind the celebrant, deacon taking the inside,
and genuflect again.} D moves the chalice to the left side of the corporal and
places the monstrance in the center of the corporal, open, facing B. 

\rubric B and ministers genuflect. B takes the host and places it in the
lunette. B puts the lunette in the monstrance, and D closes the monstrance and
turns it so that it faces the people, uncovered\footnote{S.R.C 2990.}.

\rubric Mass continues, the ceremonies prescribed \textit{coram Sanctissimo
exposito} being observed.

\rubric B does not receive his zucchetto after Communion.

\rubric After the Last Gospel, 

\section{Procession}

\rubric MC1 places the humeral veil on B's shoulder. D, with the prescribed
genuflections, take the monstrance and hands it to B, who receives it kneeling
on the top step, SD next to him. D genuflects and stands to the right of B.

\rubric The chanters intone \textit{Pange Lingua}. B then rises and stands
turned towards the people, Between D \& SD who hold the ends of his cope.

\rubric ...

\section{Benediction}

\rubric When near the altar, D kneels before B. B hands the monstrance to D,
genuflects with SD and rises as D rises. D ascends the atlar and places the
monstrance on the tabor, with the prescribed genuflections.

\rubric B \& SD kneel on the lowest step, MC1 removes the humeral veil from B.

\printbibliography

\end{document}
