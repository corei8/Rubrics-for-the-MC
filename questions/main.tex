% !TEX program = lualatex
\documentclass[10pt]{article}

\usepackage{microtype}
\usepackage{fancyhdr}
\usepackage{fontspec}
\setmainfont[Ligatures=TeX]{Minion 3}
\usepackage[english]{babel}

\begin{document}

\title{Various Liturgical Questions and Observations}
\author{Fr. G. R. Barnes}
\date{\today}

\maketitle

\tableofcontents

\pagebreak

\abstract{
    This document is a collection of questions and observation which have arisen
    that do not have an appropriate location in any of my other notes. Some
    attempt may be made at a later date to organize everything according to
    topic, but right now we have to be satisfied with the mess that is this
    research paper, due to lack of time.
}

\pagebreak

%\pagenumbering{arabic}

\section{Does a bishop wear a surplice over his rochet when hearing confessions
and distributing Communion?}

\dots

\section{In which way is the crozier held for bishops without jurisdiction?}

\dots

\section{Does a bishop have to remove his zuccheto when passing infront of the
Blessed Sacrament?}

\dots

\section{May a bishop give away his crozier when preaching?}

The reason for giving the crozier away would be to unencumber the bishop from
the awkward movements that sometimes arise from having his left hand occupied
with the crozier and trying to manage sermon notes with his gloved right hand.

I am inclined to say that the answer is No, because it seems inappropriate that
the bishop should preach to his flock lacking the symbol of jurisdiction.

However, in the case of one who does not possess the jurisdiction, having been
given faculties to preach by the Ordinary, he would not necessarily have the
right to use a crozier. This being similar to our situation, I am inclined to
say that ordinarily, for Ordinaries, preaching without the crizier would be at
least inappropriate, but for non-Ordinaries, it is a situation that could really
occur. 

\end{document}
