\chap{Vigil of Pentecost - Missa Cantata}{

\section{Preparation}

\begin{enumerate}

	\item Candles remain unlit for the prophecies, and get lit during the
		Litany of the Saints. The ACs candles start on the Epistle side
		ceredence table. The processional cross is placed near the same
		credence.

	\item The organ may not be played until after the \textit{Gloria} is
		intoned by C.

	\item The vestments for the ceremonies before the Mass are violet. The
		vestments for the Mass are red.

	\item \textit{Flectamus genua} is not said after the \textit{Oremus}
		because it si Paschal time.

	\item If there is a baptismal font, it must be blessed on this
		day.\footnote{S.R.C. 3331.} This ceremony is the same as that
		of Holy Saturday, except for the prayer recited before the
		entrance of the baptistry.
	
	\item The Pascal Candle is placed on the credence table, to be used for
		the blessing of the font. After the blessing of the font, the
		candle is extinguished and brought into the sacristy. It is
		forbidden to have the candle lit for the Mass of this
		day\footnote{S.R.C. 4048, 10.}.

\end{enumerate}

\section{Prophecies}

\rubric The procession is in the usual manner. C kisses the altar and then goes
to the missal at the Epistle side and waits for the lectors to begin the
prophecies.

\rubric ACs accompany Prophecy singers to the lectern, which is in the usual
spot for singing the Epistle.

\rubric There is no \textit{Flectámus génua} because it is Paschal Time.

\rubric During the 6\textsuperscript{th} Prophecy, MC prepares a cushion at the
foot of the altar, if it is going to be used for the prostration; ACs light
candles.

\section{Blessing of the Font}

\rubric After the Prophecies are finished, C goes to sedilia \pbr; C removes
namiple and chasuble and puts on a violet cope.

\rubric CB with cross and ACs with lit candles go to the center of the
sanctuary. TH takes the Paschal candle and walks behind them, to lead the
procession. The schola lines up before CB and ACs.

\rubric MC stands with C at the foot of the altar, holding biretta.

\rubric MC signals for a genuflection, and all process to the font, the schola
singing \textit{Sicut cervus}.

\rubric At the entrance, TH, ACs and CB turn to face C, without entering the
baptistry.

\rubric MC holds the Missal and C sings, without biretta, the Oration
\textit{Concéde, qu\'æsumus}\footnote{This oration is proper to the vigil of
Pentecost, but the rest of the ceremony is exactly the same as on Holy
Saturday.} in the ferial tone.

\rubric Once the Oration is finished, T stands close to the font, preferably to
the right of D, but not blocking MC's way. CB and ACs stand in the back of the
baptistry, facing the font. The schola enters the baptistry and stands in front
of the CB and ACs. C approaches the lectern and removes birettas.

\rubric C sings \textit{Dóminus vobíscum, Orémus} and the first oration
\textit{recto tono.}

\rubric C begins Preface with hands joined. \textit{\dots gratiam de Spiritu
Sancto}: C divides water in the form of a cross. MC ministers towel\footnote{C
dries his hand every time he touches the water.}. \textit{\dots inficiendo
corrumpat}: C touches water. \textit{\dots indulgentiam consequatur}: C makes
the sign of the cross over the water three times without touching it.
\textit{\dots ferebatur}: C sprinkles water\footnote{MC makes sure that the
Missal is closed before C sprinkles.} in the four carndinal directions. C sings
\textit{Hæc nobis præcepta} in the Lesson tone, breathes over the water three
times in the form of a cross, and continues \textit{Tu has símplices aquas} in
the same tone. MC takes Paschal Candle from TH.

\rubric Once C has he has finished singing MC gives him the Paschal Candle. C
sings \textit{Descendat} in the preface tone, raises Candle out of the water,
then lowers it again, deeper than before. C sings \textit{Descendat} in a
higher tone, raises Candle out of the water, then lowers it yet again, this
time to the bottom. C sings \textit{Descendat} for a final time, in a higher
tone than before.

\rubric With the Candle still in the water, C breathes three times over the
water in the form of a letter Psi $\textrm{\psi}$. C sings \textit{Totámque
hujus\dots effectu} and removes the Candle. MC retuns the Candle to TH, who
dries it.

\rubric C continues in the preface tone, singing as far as \textit{infántiam
renascátur} and then reads the conclusion in a loud voice; schola answers
\textit{Amen.}

\rubric MC fills aspersory with water and hands the sprinkler to C with the
usual kisses. C blesses himself, sprinkles those who are around the font and then,
sprinkles the congregation.

\rubric \textbf{N.B.} If there is another priest present, he may sprinkle the
congregation. Wearing a surplice and violet stole, he presents the sprinkler to
the C with kisses. After the C has blessed him and the Ministers, he receives
the sprinkler from the C with kisses. With the MC at his right, he sprinkles
the congregation\footcite[p. 230]{hweekls}.

\rubric MC pours the needed quantity of water into the baptismal font and puts
the tray of oils on edge of the font.

\rubric C pours a little of the Oil of the Catechumens in the form of a cross
into the water saying in a loud voice \textit{Sanctificétur et fecundétur}.

\rubric C pours Chrism in the same way, but saying \textit{Infúsio Chrísmatis.}

\rubric C takes both oils and pours them at the same time into the water saying
the \textit{Commixtio Chrismatis} and making the sign of the cross three times.

\rubric C mixes the oil and water with his right hand. C cleanses his hands, MC
ministering the towel\footnote{MC should cover the font, lest any of the
Baptismal Water be taken (Holy Saturday, Brooksville, FL, 2021).}.

\rubric The procession goes back the same way it came. The schola begins the
Litany of the Saints as soon as the procession leaves the baptistry; the
invocations are doubled.

\rubric TH puts the Candle in the holder and turns it so that the cross faces
the people; CB and ACs go to the credence.

\rubric C goes to the sedillia removes cope.

\section{Litany of the Saints}

\rubric At a signal from MC, C goes to the the altar, kneels and prostrates. MC
kneels behind and to the right of C, signalling all to kneel.

\rubric At \textit{Per sanctam resurrectiónem tuam} ACs and TH go to the center
of the sanctuary; MC gets biretta.

\rubric At \textit{Peccatóres} Ministers rise. All genuflect and go to the
sacristy, ministers wearing birettas.

\section{First Mass of Easter}

\rubric C vests in white vestments. Altar is prepared for Mass.

\rubric At \textit{Christe audi nos} MC gives a signal and all bow to the Cross
in the sacristy. Procession in as usual\footcite[If the baptismal font was
blessed, the AC's candles are already on the credence table; if the font was
not blessed, the ACs carry them.][footnote, p. 233]{hweekls}.

\rubric No Introit.

\rubric Bells are rung throughout the sung \textit{Glória}.

\rubric Acolytes do not carry candles at the Gospel procession.

\rubric No Creed.

\rubric Mass concludes in the usual manner.

}

