\chap{Good Friday}{

\section{Preparation}

Work in progress\dots

\section{Preliminary Observations}

\rubric All choir bows are omitted from the Adoration of the Cross until None
of Holy Saturday, inclusive\footnote{S.R.C. n. 3059 ad 27.}.

\rubric All holy water stoops should be removed.

\section{Mass of the Catechumens}

\rubric ACs and TH do not have candles or thurible; holy water is not presented
to C.

\rubric C bows while ministers genuflect after all have taken off birettas.

\rubric C and ministers kneel and prostrate themselves at the foot of the
altar; all kneel.

\rubric After a short prayer ACs rise and spread an altar cloth. This cloth
should be folded in such a way that it only covers the back of the
\textit{mensa}; MC places the missal and its stand on the Epistle side,
opeining it to the first Lesson.

\rubric MC kneels to the right of D.

\subsection{First Lesson}\label{flectamus}

\rubric MC kneels for a moment then signals all to rise, the prostation being
about the length of a \textit{Miserére}.

\rubric C, D \& SD ascend; C kisses altar and goes to the Missal; D \& SD to
Introit positions.

\rubric A lector\footnote{Preferably a real lector. If no cleric is available,
SD sings the Lesson.} sings the Lession at a lectern where the Epistle is
usually sung; C reads Lesson and Tract.

\rubric Towards the end of the Tract, ministers \textit{unum post alium}; all
in choir rise.

\rubric C sings \textit{Orémus}; C \& SD bowing with him.

\rubric D sings \textit{Flectámus génua} and all genuflect.

\rubric Upon genuflecting, SD sings \textit{Leváte} and all rise.

\rubric C does not genuflect.

\rubric With hands extended, C sings the Oration in the ferial tone.

\rubric As soon as C begins the Oration, SD removes folded chasuble and
receives \textit{Lectionarium} from MC.

\rubric SD and MC genuflect in center then go to the place where the Epistle is
usually sung; SD sings the Second Lesson in the Epistle tone; choir sits

\rubric SD and MC genuflect in the center and go immediately to the sedilia
where Sd resumes the folded chasuble.

\rubric SD rejoins D and C.

\rubric After C is finished with the Tract, C, D \& SD go to sedilia
\textit{per breviorum.}

\rubric At the beginning of the last verse of the Tract, C, D \& SD go
\textit{per breviorum} to the missal, standing in the Introit position.

\rubric Chanters of the Passion enter when C is at the Missal.

\rubric C begins to read the Passion when the singers begin; after
finishing\footnote{C reads the Passion, not genuflecting. He then recites the
\textit{Munda cor} in place, bowing towards the Crucifix, then reads the
\textit{pars Evangelii}.}, C, D \& SD turn to face where the Passion is being
sung.

\rubric D, C \& SD kneel towards the Cross after the words \textit{trádidit
spíritum.}

\rubric After the Passion, the singers return to the sacristy; choir sits.

\rubric D does not ask for C's blessing; choir stands after D takes his book
from the altar.

\rubric ACs do not carry candles; TH does not take part.

\rubric C does not kiss the book, nor is he incensed.

\rubric After the Gospel, D \& SD \textit{unum post alium.}

\rubric If a sermon is to be preached, C, D \& SD to sedilia; afterwards they
return to the missal \textit{per breviorum} and stand \textit{unum post alium.}

\section{Litanical Prayers}

\section{Adoration of the Cross}

\section{Mass of the Presanctified}

\section{Exposition of the Relic of the Cross}

}


