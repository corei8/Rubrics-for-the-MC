\chap{Good Friday}{

    \section{Preparation}

    Work in progress\dots

    \section{Preliminary Observations}

    \rubric After the adoration of the cross, all, even, the celebrant,
    genuflect to the cross.

    \rubric All of the \textit{oscula} are omitted.

    \section{Vesting of the Bishop}

    \rubric B does not bless the ACs after he washes his hands.

    \rubric D \& SD vest the bishop \textit{in albis.}

    \rubric SD gives B his maniple immediately after the chasuble.

    \section{Mass of the Catechumens}

    \rubric The procession proceeds to the altar as ususal, but without the
    processional cross or the SD carrying his book, although he processes in
    his usual place alone.

    \rubric At the foot of the altar, all genuflect \textit{in plano.} The
    bishop prostrates on the faldstool, AP to his right and D \& SD to his left
    prostrate on cushions. They remain prostrate for about the length of a
    \textit{Miserere}. All others kneel.

    \rubric ACs extend the altar cloth on the altar, doubling it back from the
    front of the altar. The missal and stand are placed on the altar on the
    Epistle side.

    \rubric B and ministers rise at MCs signal. Mc2 takes the faldstool from
    before B and places it on the Epistle side of the predella, facing the
    Gospel side. B ascends, kisses the altar, and sits in the faldstool with
    miter and gremiale. The assistants, standing on the step below the
    platform, make a reverence to the cross and to the bishop and sit in a
    line on the platform facing the people: AP nearest B, at his feet, D to the
    right of AP, SD to right of D.

    \rubric A lector, vested in surplice, stands \textit{in plano} on the
    Epistle side and sings the lesson in the tone of the prophecy, with the
    usual reverences before and after, but without kissing the hand of the
    bishop. The choir sings the tract.

    \rubric Ministers rise and attend B, who remains seated and reads the
    lesson and tract, without the bugia, book held by SD.\footnote{``Episcopus
        legit ex libro Prophetiam sine candela accensa; haec verba intelligenda
        sunt ita, ut Palmatoria nullo modo, neque extincta, adhibenda sit.''
        S.R.C. n. 4257 ad 6.} AP \& D stand to B's left.

\rubric Miter removed, B rises, and sings \textit{Orémus}. D sings
\textit{Flectámus génua} and SD \textit{Leváte}. All kneel on Mc1's signal,
except B, who sings the oration in the ferial tone. All then sit as before.

\rubric During the oration, SD goes to the credence, removes his folded
chasuble, and receives the Epistolarum from Mc2. When all have seated, SD sings
the Epistle in the usual manner.\footnote{If any of the ministers are to sing
the Passion, they should leave now to the sacristy and vest.} He makes the
usual reverences, but does not receive a blessing from B. He then resumes
the folded chasuble and kneels before B, who reads the Epistle and Tract.

\rubric While the choir sings the tract, the minsters of the Passion enter.
Each carries his book before his breast, they genuflect on to the cross, and go
to the lectern on the Gospel side, and sing the Passion.

\rubric D removes the gremiale and miter and B sings the Passion at the Epistle
corner, from beginning to end, without kneeling, and without saying the
\textit{Munda cor.} When he has finished the Passion, he turns to face the
chanters, and at \textit{tradidit spiritum} kneels on a cushion placed before
him by Mc1.

\rubric After the singing of the Passion, B sits with miter and gremiale. D
takes off his chasuble at the credence table, and puts on the broad stole. He
takes the Evangeliarum and goes to the altar as usual, saying \textit{Munda
cor} on the top step, etc. D takes the book and goes without
candles\footnote{ACs assist with hands folded.} or incense to sing the Gospel
in the ferial tone.

\rubric B stands without miter at the epistle corner, turned towards D. AP
stands \textit{in plano} to the left of B.

\rubric At the end of the Gospel, B does not kiss the book: SD hands it
immediately to Mc2.

% \subsection{First Lesson}\label{flectamus}

\section{Litanical Prayers}

\rubric Sermon. If B preaches, he sits and faces the people. If the sermon be
preached by another, B sits with miter and gremiale.

\rubric After the sermon, B stands before the missal at the Epistle side and
sings \textit{Oremus, etc.}. AP stands to the right of B; D \& SD stand as for
the orations during Mass. D says \textit{Flectamus genua} and SD
\textit{Levate} as in the missal.

\rubric At the fifth oration, ACs spread a violet carpet at the foot of the
altar steps. They place a violet cushion with a veil for adoration of the Cross
on the bottom altar step.

\section{Unveiling \& Adoration of the Cross}

\rubric After the last oration, Mc2 places the faldstool on the platform on the
Epistle side.

\rubric B sits and D \& SD assist him in removing his chasuble. SD then removes
his folded chasuble. B receives miter from D and stands. Mc2 places the
faldstool in its usual place at the steps on the Epistle side, where B sits
while D \& SD remove their vestments.\footcite[][]{sterkyFP:2}

\rubric B, accompanied by D \& SD, descends \textit{in planum} at the epistle
side and stands, facing the people. D receives the cross from Mc2, the Corpus
facing the people. D hands the cross to B. AP stands before B, holding the
missal. SD stands to the left of B.

\rubric B uncovers the top of the cross as far as the cross-piece. D \& SD
assist B in uncovering the cross. B raising the Cross sings \textit{Ecce Lignum
Crucis}. B and ministers continue to sing \textit{In quo salus mundi pependit.}
The choir and all the clergy, with the exception of B, kneel on both knees at
Mc1's signal, for the words \textit{Venite adoremus}, sung by the choir.

\rubric This ceremony is repeated on the predella, at the Epistle corner, in a
higher tone, the head and right arm of the Corpus being uncovered.

\rubric This ceremony is repeated for a last time on the predella, at the
middle, in a still higher tone, the entire Cross being uncovered.

\rubric After the last time, all remain kneeling. B, accompanied by Mc1, goes
directly to the cushion on on the last step, kneels and places the cross on the
cushion.

\rubric B then rises, genuflects, and goes to the faldstool on the platform on
the epistle side of the altar, where he sits and receives the miter. Mc1
removes B's shoes. D removes miter and B removes his maniple.\footcite[][p.
204, note 6.]{stehle} Accompanied by D \& SD (but only to the first
genuflection, when Mc1 alone accompanied B during his adoration), B performs
the triple adoration, kneeling first near the altar, then halfway to the cross,
then beside the cross, where he kisses the feet of the Corpus. Then B rises,
genuflects on one knee, and goes to the faldstool, accompanied by D \& SD.

\rubric While B is adoring the cross, AP, D \& SD remove their shoes, but not
their vestments in preparation for adoring the cross.

\rubric  B puts on his shoes with the aid of D \& SD, and then resumes chasuble and
maniple. B sits and receives the miter from D. 

\rubric AP, D \& SD perform the adoration in the same manner as B, AP kneeling
between D and SD.

\rubric AP, D \& SD return to the sedilia and put on their shoes. The Bb holds the
missal before B, open to the \textit{Improperia}. Two more missals are held
before D \& SD by two servers. B reads the \textit{Improperia} alternately with
D \& SD.

\rubric During the reading of the \textit{Improperia}, all the crucifixes in
the church are uncovered. Candles on the altar and on the credence tables are
lit, torches and candles for the procession from the Altar of Repose are lit.

\rubric At the conclusion of the \textit{Improperia}, D \& SD, with the
prescribed reverences, unfold the cloth so that the \textit{mensa} is covered.
D takes the burse to the altar, unfolds the corporal, and places the
purificator near it. AP transfers the missal to the Gospel side.

\rubric At the conclusion of the adoration, D, accompanied by Mc1, goes to the
Cross, genuflects, and takes it to the credence table, without making any
reverences.\footnote{It appears that the rubric for kneeling at this transfer
of the cross is necessary only when the cross is places above the altar.}

\rubric ACs remove the cushion and carpet. At this time Ths prepare their
thuribles for the procession. Th2 goes to the altar of repose and waits there.

\section{Procession to the Repository}

\rubric ACs wash B's hands.

\rubric Incense is imposed, but not blessed.

\rubric Mc2 arranges the procession: in the following order: TH, CB with his
cross uncovered, flanked by ACs with lighted candles, clergy, SD, B flanked by
D \& AP.

\rubric Miter removed, all genuflect to the cross, D imposes the miter, and all
proceed to the Altar of Repose, the insignia bearers following behind. 

\rubric At the entrance of the chapel, D removes miter and zucchetto. All
genuflect \textit{in plano} before the altar. B kneels, AP kneels to the
side,\footcite[][]{sterkyFP:2} D \& SD on either side of B.

\rubric A priest in surplice and black stole unlocks and opens the
tabernacle.\footnote{In the absence of an available priest, it seems that the
deacon should supply this function.}

\rubric Incense is imposed in two thuribles and B incenses Blessed
Sacrament, using the thurible from Th1.

\rubric Mc1 puts the humeral veil on the shoulders of B, and D fastens it. D
rises, ascends, genuflects, takes the chalice from the taberncale and hands it
to B, who kneels on the edge of the predella. D covers the chalice with the ends
of the humeral veil and genuflects to Blessed Sacrament.

\rubric B rises, turns towards the people, and the procession goes to the altar
\textit{per viam longiorem}. Meanwhile, the choir sings \textit{Vexilla Regis}.
The order of the procession is the same as that of Holy Thursday, except AP
stands before Ths.

\section{Mass of the Presanctified}

\rubric At the altar, CB and ACs put aside cross and candles and kneel. The
clergy, with lighted candles, kneel in a semicircle before the altar. THs and
TBs kneel, the latter in their ususal positions, and the former on either side
of the altar, before the predella.

\rubric AP kneels on the gospel side on the lowest step, until he is needed for
the missal, which he removes during the incensation, and SD on the gospel side,
to the left of center. D kneels \textit{in plano} and receives the chalice from
the bishop, who ascends and places the chalice on the altar.

\rubric B, having given the chalice to D, genuflects and kneels on the lowest
step. Mc1 removes the humeral veil.

\rubric D removes the silk ribbon from the chalice, and arranges the veil as at
the beginning of Mass, genuflects, descends and assists in the incensation.

\rubric AP assits B in imposing incense, and B incenses Blessed
Sacrament, kneeling between D \& SD. Th2 retires to the sacristy.

\rubric B ascends the altar, D \& SD on the right and AP on the left. All
genuflect.

\rubric D removes the veil from the chalice, then the paten and the pall. D
holds the paten with both hands before B, who inverts the chalice and allows
the Sacred Host to fall onto the paten.\footnote{If either the bishop or the
    deacon comes into contact with the Sacred Host, he immediately purifies his
fingers using the ablutions cup.}

\rubric B receives the paten from D (\textit{sine osculis}) places the Sacred
Host on the corporal, and places the paten on the corporal to the right.

\rubric D does not purify the chalice, holds it slightly above the altar and
pours wine into it, and the SD adds a few drops of water. B receives the
chalice (\textit{sine oscula}) and places it on the corporal, saying nothing
and not making the sign of the cross, and D covers it with the pall.

\rubric B imposes incense, omitting \textit{Per intercessionem} and not
blessing the incense, AP ministering, and incenses the \textit{oblata} saying
\textit{Incensum istud; Dirrigatur, etc.,} as usual.\footnote{S.R.C. n. 2003.}
Neither B nor anyone else is incensed.

\rubric B stands \textit{in plano} at the epistle side, facing the people,
washes his hands, without miter and omitting the \textit{Lavabo.}

\rubric B returns to the middle of the altar, says \textit{In spiritu
humilitatis, etc.} and turned towards the people, standing slightly to the
gospel side, says the \textit{Orate fratres}, and turns back to the altar
without completing his turn. The \textit{Suscipiat} is not said.

\rubric B sings in the ferial tone with hands joined \textit{Oremus. Præceptis,
etc.,} then with hands extended \textit{Pater noster, etc.} The choir answers
\textit{Sed libera nos a malo,} and B says (\textit{submissa voce})
\textit{Amen.}

\rubric With hands still extended, B says in the ferial tone \textit{Libera
nos, etc.,} but does not make the sign of the cross with the paten. The choir
answers \textit{Amen.}

\rubric Minsters kneel as at the elevation. B genuflects, places the paten
under the Sacred Host and holds the paten with his left hand, while with his
right he elevates the Sacred Host. The Blessed Sacrament is not incensed, the
chasuble is not raised, no sign is made with the clapper.

\rubric The ministers rise. D uncovers the chalice. B, without genuflecting,
divides the Sacred Host into three parts and drops the smallets into the
chalice, omitting the sign of the cross and the prayer.

\rubric The \textit{Agnus Dei} and the \textit{Pax} are omitted.

\rubric Before communion B says only one prayer \textit{Perceptio Corporis,
etc.,} then \textit{Panem cœlestem} and \textit{Domine non sum dignus,} as
usual, and communicates.

\rubric D uncovers the chalice; B and his assistants genuflect and B gathers
the fragments, omitting the \textit{Quid retribuam, etc.,} and holding the
paten, as usual, receives the Sacred Particles with the wine, without making
the sign of the cross with the chalice.

\rubric Mc2 gets the zucchetto and places it on B's head.

\rubric D pours wine and water over the fingers of B, as usual, but the
\textit{Corpus tuum Domine, etc.} is not said. B dries his fingers, receives
the ablution, inclines moderately, and with his hands joined before his breast
says in a clear tone of voice the prayer \textit{Quod ore sumpsimus, etc.}

\rubric D goes immediately after pouring wine and water for the ablutions to
the credence and removes his broad stole and dons the folded chasuble.

\rubric Meanwhile, SD arranges the chalice and takes it to the credence; AP
closes the missal; the clergy extinguish their candles.

\rubric B goes to the epistle side, receives the miter from D and washes his
hands, AP ministering the towel.

\rubric B goes to the foot of the altar where D removes the miter. All line up
in the usual manner to process into the sacristy. At Mc1's signal, all
genuflect to the cross and retire to the sacristy, where B is devested in the
usual manner.

% \rubric Acs enter the sanctuary as B is being devested and remove the cloth
% from the altar.

% \section{Vespers}

\section{Exposition of the Relic of the Cross}

\rubric The servers necessary are MC and TH, and two ACs with lighted candles.
Two candles are lighted on the altar.

\rubric All process in as usual, no bells being sounded, and C, vested in
surpice, red stole and red chasuble, carrying a covered reliquary containing a
relic of the True Cross. All line up as usual and genuflect to the cross. ACs
place their candles on either side of the altar, as at Benediction of the
Blessed Sacrament, but remain standing.

\rubric C goes immediately to the center of the altar, and places the relic on
it and uncovers it.

\rubric C bows profoundly and joins MC and TH at the foot of the altar. Incense
is imposed as at Benediction, with the exceptions that it is not blessed and C
and servers do not kneel.

\rubric C incenses the relic with three doubles, servers holding the ends of
his cope, and all genuflecting before and after, remaining standing.

\rubric After the thurible is returned to TH, all genuflect to relic and cross
together and retire to the sacristy.

\rubric When it is time to remove the relic, all enter as before.

\rubric After genuflecing, incense is imposed, and the relic is incensed as
before. C ascends the altar and genuflects, MC signals all to kneel. With a
single sign of the cross, C gives benediction with the relic, then turns back
to the altar and replaces the relic.

\rubric MC signals all to stand. C covers the relic, and ACs bring their
candles to the front of the predella.

\rubric C descends with the relic. All genuflect to the cross, and retire to
the sacristy.

}


