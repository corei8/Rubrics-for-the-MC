\chap{Solemn Pont. Good Friday}{

\section{Preparation}

    Work in progress\dots

\section{Preliminary Observations}

    \rubric After the adoration of the cross, all, even, the celebrant,
    genuflect to the cross.

    \rubric All of the \textit{oscula} are omitted.

    \rubric When B sits in the faldstool, the ministers arrange themselves on
    the highest step, sitting in a straight line. AP is at B's feet, D to his
    right and SD to D's right. Bb and Mb sit on the Epistle
    side behind B.

    \rubric When the ministers and insignia bearers are to sit, the stand in
    their place, bow to the altar, then bow to B, turn to their right, and then
    sit. When they rise, they stand all together, turn to their left, fist bow
    to the altar, and then bow to the bishop.

\section{Vesting of the Bishop}

    \rubric B does not bless the ACs after he washes his hands.

    \rubric D \& SD vest the bishop \textit{in albis.}

    \rubric SD gives B his maniple immediately after the chasuble.

\section{Mass of the Catechumens}

    \rubric The procession proceeds to the altar as ususal, but without the
    processional cross or the SD carrying his book, although he processes in
    his usual place alone.

    \rubric At the foot of the altar, all genuflect \textit{in plano.} The
    bishop prostrates on the faldstool, AP to his right and D \& SD to his left
    prostrate on cushions. They remain prostrate for about the length of a
    \textit{Miserere}. All others kneel.

    \rubric ACs extend the altar cloth on the altar, doubling it back from the
    front of the altar. The missal and stand are placed on the altar on the
    Epistle side, and the AC opens the missal.

    \rubric B and ministers rise at MCs signal. Mc2 takes the faldstool from
    before B and places it on the Epistle side of the predella, facing the
    Gospel side. B ascends, D passing to his right side and followed by the
    insignia bearers, AP going to the Epistle corner, kisses the altar, and
    sits in the faldstool with miter and gremiale.

    \rubric The assistants\footnote{This term refers to the ministers, the
    book-bearer and the miter-bearer.}, standing on the step below the
    platform, make a reverence to the cross and to the bishop and sit in a line
    on the platform facing the people, observing the rubric above.

    \rubric A lector, vested in surplice, stands \textit{in plano} on the
    Epistle side and sings the lesson in the tone of the prophecy, genuflecting
    to the altar and bowing to B before and afterwards as usual, but without
    kissing B's hand nor receiving his blessing. The choir sings the tract.

    \rubric Assistants rise in the manner described above and attend B. Bb
    stands before the altar and genuflects, then bows to B and takes the missal
    from its stand on the altar and kneels before B. AP stands to B's left, D
    to B's right and SD stands behind AP. B reads the Lesson and
    Tract.\footnote{``Episcopus legit ex libro Prophetiam sine candela accensa;
    haec verba intelligenda sunt ita, ut Palmatoria nullo modo, neque extincta,
adhibenda sit.'' S.R.C. n. 4257 ad 6.} When B is finished Bb puts the book back
on its stand and returns to his place behind B.


    \rubric When the chanters are nearing the end of the Tract, the Mb
    approachs. At the end of the Tract, SD removes the gremiale and D removes
    the miter. B rises and Mc1 removes the faldstool. B rises and faces the
    missal, AP assisting, and D \& SD stand \textit{unum post alium}.

    \rubric Miter removed, B rises, and sings \textit{Orémus}. D sings
    \textit{Flectámus génua} and SD \textit{Leváte}. All kneel on Mc1's signal,
    except B. B sings the oration in the ferial tone.

    \rubric During the oration, SD goes to the credence, removes his folded
    chasuble, and receives the Epistolarum from Mc2. Mc1 places the faldstool
    on the predella; Mb and Gb stand by the faldstool. B sits, D imposes miter
    and the Assitants sit. When all have seated, SD sings the Epistle in the
    usual manner.\footnote{If any of the ministers are to sing the Passion,
    they should leave now to the sacristy and vest.} He makes the usual
    reverences, but does not receive a blessing from B. He then resumes the
    folded chasuble. The choir sings the Tract once SD has finished singing.

    \rubric Bb takes the missal from the altar and kneels before B, who reads
    the Second Lesson and Tract. SD returns to his place behind AP.

    \rubric While the choir sings the Tract, the deacons of the Passion
    enter.\footnote{If the Subdeacon is to be one of the chanters of the
    Passion, the other chanters wait for the Subdeacon to get vested and then
process out.} Each carries his book before his breast, all three genuflect to
the cross, bow to B, and go to the lecterns on the Gospel side.

    \rubric SD removes the gremiale, D removes the miter and B stands to face
    the missal on the altar. Mc2 removes the faldstool. Chanters begin singing
    the Passion once B is ready and reading the Passion. The Passion is read
    from beginning to end, without kneeling, and without saying the
    \textit{Munda cor.} When he has finished the Passion, he turns to face the
    chanters. D \& SD stand in their Introit positions and AP stands nears the
    missal. Once B is finished reading, he turns to his left and faces the
    chanters of the Passion. AP, D \& SD stand to the left of B, in that order.
    Whenever the name of \textit{Jesus} occurs, all four turn and bow towards
    the the cross. At \textit{tradidit spiritum}\footnote{The first master of
        ceremonies should listen for the words \textit{consummatum est.}} all
        turn towards the altar and kneel. \footnote{If the Assistant Priest,
        Deacon or Subdeacon are singing the Passion, their tasks are taken over
    by the masters of ceremonies.}

    \rubric After the singing of the Passion, the chanters of the Passion
    retire in the same manner they entered and Mc2 replaces the faldstool on
    the altar. B sits with miter (D) and gremiale (SD). D takes off his
    chasuble at the credence table, and puts on the broad stole. D takes the
    Evangeliarum, goes to the foot of the altar and reverences the altar and B,
    ascends to the altar and places the book there, and kneels on the edge of
    the footpace for the \textit{Munda cor meum.} D rises, takes the book and
    goes down to the foot of the altar to the right of SD. ACs assist with
    hands folded, TH does not attend. All reverence first the altar, then B,
    and go to the ususal place for the Gospel.

    \rubric Once D has left the altar, Mc1 and AP remove the gremile and Mc1
    removes the miter. When D begins to sing the Gospel, B stands at the
    epistle corner, turned towards D. AP stands to the left of B.

    \rubric At the end of the Gospel SD hands the book immediately to Mc2. D \&
    SD return to the altar and stand by B. D retains his broad stole.

\section{Sermon}

    % TODO: figure out all the scenarios for this.
    \rubric If B preaches, he sits and faces the people. If the sermon be
    preached by another, B sits with miter and gremiale, with the faldstool on
    the Epistle side, but turned to face the Preacher. Ministers sit on the top
    step during the sermon.\footcite[][]{sterckyFP:2}

\section{Litanical Prayers}

    % TODO: explain this a little bit more
    \rubric Ministers rise, SD removes gremiale and D removes miter. Mc2
    removes faldstool. B stands before the missal at the Epistle side and sings
    \textit{Oremus, etc.}. AP stands to the right of B; D \& SD stand as for
    the orations during Mass. D says \textit{Flectamus genua} and SD
    \textit{Levate} as in the missal.

    \rubric The oration for the Emperor is omitted.

    \rubric At the 8\textsuperscript{th} oration, for the Jews, the
    \textit{Flectámus génua} is omitted.

    \rubric At the 5\textsuperscript{th} oration, ACs spread a violet carpet at
    the foot of the altar steps. They place a violet cushion with a veil for
    adoration of the Cross on the bottom altar step.

\section{Unveiling \& Adoration of the Cross}

    \rubric After the last oration, Mc2 places the faldstool on the predella on
    the Epistle side. AP takes the missal and gives it to Bb and both descend
    the steps on the Epistle side and stand some distance from the altar.

    \rubric B sits and removes his chasuble assisted by D \& SD, who give it to
    Ac1 who takes it to the credence. Ac2 approaches with a tray; D removes the
    zucchetto and gives it to Ac2.

    % \rubric B sits and D \& SD assist him in removing his chasuble. SD then removes
    % his folded chasuble. B receives miter from D and stands. Mc2 places the
    % faldstool in its usual place at the steps on the Epistle side, where B sits
    % while D \& SD remove their vestments.\footcite[][]{sterkyFP:2}

    \rubric B, accompanied by D \& SD, descends \textit{in planum} at the
    epistle side and stands, facing the people. D receives the cross from Mc2,
    the Corpus facing the people. D hands the cross to B and stands to his
    right. AP stands before B, holding the missal, Bp to his right. SD stands
    to the left of B.

    \rubric B uncovers the top of the cross as far as the cross-piece. D \& SD
    assist B in uncovering the cross. B raising the Cross to the lefel of his
    eyes sings \textit{Ecce Lignum Crucis}. B must take care that he begins in
    a tone low enough to allow him to raise the pitch two times. B and
    ministers continue to sing \textit{In quo salus mundi pependit.} The choir
    and all the clergy, with the exception of B, kneel on both knees at Mc1's
    signal for the words \textit{Venite adoremus}, sung by the choir. Before
    kneeling, AP closes the missal.

    \rubric This ceremony is repeated on the predella, at the Epistle corner, B
    standing in the place where the Introit is read, SD standing on the step
    below the footpace. B removes the veil from the head and right arm of the
    Corpus, holds the cross a little higher and sings in a higher tone.

    \rubric This ceremony is repeated for a last time on the predella, at the
    middle, holding the cross higher and singing in a still higher tone, the
    entire Cross being uncovered. B passes the veil to SD, who gives it to Ac1.

    \rubric After \textit{Veníte adorémus}, all remain kneeling. AP, D \& SD,
    still kneeling, turn to face the cushion on which B will place the cross,
    and AP returns the missal to Bp. B, accompanied by Mc1, walks slowly down
    the Gospel side to the cushion on the last step, kneels and places the
    cross on the cushion. Mc2 places the faldstool on the the predella.

    \rubric B then rises, and everyone rises at the same time. B, AP, D \& SD
    genuflect together towards the cross. AP, D \& SD join B at the foot of the
    altar and conduct him to the faldstool. AP descends from the Epistle side;
    Mb stands near D.

    % TODO: look up this one.
    \rubric All of the crosses are uncovered at this time.

    \rubric B sits; D imposes zucchetto (brought by Ac1) and miter and SD
    removes B's maniple\footcite[][p. 204, note 6.]{stehle}. Mc1 removes B's
    shoes. 


    \rubric Accompanied by D \& SD and followed by Mb (but only to the first
    genuflection, when Mc1 alone accompanies B during his adoration). D removes
    the miter and zucchetto and D, SD \& Mb return to the foot of the altar, on
    the Epsitle side.

    \rubric B performs the triple adoration, Mc1 to his left, kneeling first at
    the end of the carpet, then halfway to the cross, then beside the cross,
    where he kisses the feet of the Corpus. Then B rises, genuflects on one
    knee, and goes to the faldstool, accompanied by D \& SD, D first imposing
    the miter and zucchetto.

    \rubric B sits in the faldstool. Mc1 replaces B's shoes. D removes the
    miter. Acs bring the maniple and chasuble, with which D \& SD vest B, then
    D imposes miter and D \& Mc1 impose the gremiale.

    % TODO: make sure that this is accurate
    \rubric While B is adoring the cross, AP, D \& SD remove their shoes and
    maniples, but not their other vestments in preparation for adoring the
    cross.

    \rubric AP, D \& SD perform the adoration in the same manner as B, AP
    kneeling between D and SD.

    \rubric All the other clergy and the servers adore the cross in the same
    manner.


    % TODO: work on this...
    \rubric AP, D \& SD return to the sedilia and put on their shoes and
    maniples. The Bb holds the missal before B, open to the
    \textit{Improperia}. Two more missals are held before D \& SD by two
    servers. B reads the \textit{Improperia} alternately with D \& SD.

    \rubric Another priest, or several, if there are many to adore the cross,
    vested in surplice and black stole, carries another smaller crucifix to the
    communion rail, where the faithful adore the cross. The priest carries a
    white cloth in his left hand, which he uses to wipe the feet of the Corpus
    after it is kissed.
    %...


    \rubric During the reading of the \textit{Improperia}, all the crucifixes
    in the church are uncovered. Candles on the altar and on the credence
    tables are lit, torches and candles for the procession from the Altar of
    Repose are lit.

    \rubric At the conclusion of the \textit{Improperia}, D \& SD first bow to
    B, then unfold the altar cloth so that the \textit{mensa} is covered. D
    takes the burse to the altar (brought to him by Ac1), unfolds the corporal,
    and places the purificator near it. Bb transfers the missal to the Gospel
    side.

    \rubric At the conclusion of the adoration, D, accompanied by Mc1, goes to
    the Cross, genuflects, and takes it to the credence table, without making
    any reverences.\footnote{It appears that the rubric for kneeling at this
        transfer of the cross is necessary only when the cross is placed above
        the altar.} From this moment on everyone, even B, reverences the cross
        by genuflecting.

    \rubric ACs remove the cushion and carpet. At this time Ths prepare their
    thuribles for the procession. Th2 goes to the altar of repose and waits
    there.

\section{Procession to the Repository}

    \rubric ACs wash B's hands.

    \rubric Incense is imposed, but not blessed.

    \rubric Mc2 arranges the procession: TH, CB with his cross uncovered,
    flanked by ACs with lighted candles, clergy, and SD, who stands alone.

    
    \rubric B rises with miter and is flanked by D \& AP. Once all three are
    facing the altar and standing \textit{in plano}, D removes miter and
    zucchetto. At Mc1's signal, all genuflect to the cross, turn and begin
    processing to the Altar of Repose \textit{per breviorem}. D imposes
    zucchetto and miter.

    \rubric At the entrance of the Repository, D removes miter and zucchetto.

    \rubric All double genuflect \textit{in plano} before the altar as they
    enter and then go to their places, which are the same as the day before.
    Th1 is joined by Th2, who kneels to the left of SD. B kneels flanked by D
    \& SD; AP kneels to the side.\footcite[][]{sterkyFP:2} CB and the acolytes
    stand near the entrance of the Repository.

    \rubric A priest in surplice and black stole unlocks and opens the
    tabernacle.\footnote{In the absence of an available priest, it seems that
    the deacon should supply this function.}

    \rubric Incense is presented by D and imposed in both thuribles without
    blessing. B incenses Blessed Sacrament, using the thurible from Th1.

    \rubric Mc1 puts the humeral veil on the shoulders of B, and D fastens it.
    D rises, ascends, genuflects, takes the chalice from the tabernacle,
    \textit{does not place the chalice on the altar} and hands it to B, who
    kneels on the edge of the predella, head bowed profoundly. B places his
    right hand on top of the chalice and D covers the chalice with the ends of
    the humeral veil and genuflects to Blessed Sacrament. Meanwhile, Ub gets
    the umbrellino and stands ready.

    \rubric B rises, turns towards the people, and the procession goes to the
    altar \textit{per viam longiorem}. Ub holds the umbrellino over the Blessed
    Sacrament and the Bishop. Meanwhile, the choir sings \textit{Vexilla
    Regis}. The order of the procession is the same as that of Holy Thursday,
    except AP stands before Ths.

\section{Mass of the Presanctified}

    \rubric At the altar, CB and ACs put aside cross and candles and kneel at
    the credence table. The clergy, with lighted candles, kneel in a semicircle
    before the altar. THs and TBs kneel, the latter in their ususal positions,
    the former on either side of the altar, before the predella.

    \rubric AP kneels on the gospel side on the lowest step, until he is needed
    for the missal, which he removes during the incensation, and SD on the
    gospel side, to the left of center. D kneels \textit{in plano} and receives
    the chalice from the bishop, who ascends and places the chalice on the
    altar.

    \rubric B, having given the chalice to D, genuflects and kneels on the
    lowest step. Mc1 removes the humeral veil.

    \rubric D removes the silk ribbon from the chalice, and arranges the veil
    as at the beginning of Mass, genuflects, descends and assists in the
    incensation.

    \rubric AP assits B in imposing incense, and B incenses Blessed Sacrament,
    kneeling between D \& SD. Th2 retires to the sacristy.

    \rubric B ascends the altar, D \& SD on the right and AP on the left. All
    genuflect.

    \rubric D removes the veil from the chalice, then the paten and the pall. D
    holds the paten with both hands before B, who inverts the chalice and
    allows the Sacred Host to fall onto the paten.\footnote{If either the
        bishop or the deacon comes into contact with the Sacred Host, he
        immediately purifies his fingers using the ablutions cup.}

    \rubric B receives the paten from D (\textit{sine osculis}) places the
    Sacred Host on the corporal, and places the paten on the corporal to the
    right.

    \rubric D does not purify the chalice, holds it slightly above the altar
    and pours wine into it, and the SD adds a few drops of water. B receives
    the chalice (\textit{sine oscula}) and places it on the corporal, saying
    nothing and not making the sign of the cross, and D covers it with the
    pall.

    \rubric B imposes incense, omitting \textit{Per intercessionem} and not
    blessing the incense, AP ministering, and incenses the \textit{oblata}
    saying \textit{Incensum istud; Dirrigatur, etc.,} as usual.\footnote{S.R.C.
    n. 2003.} Neither B nor anyone else is incensed.

    \rubric B stands \textit{in plano} at the epistle side, facing the people,
    washes his hands, without miter and omitting the \textit{Lavabo.}

    \rubric B returns to the middle of the altar, says \textit{In spiritu
    humilitatis, etc.} and turned towards the people, standing slightly to the
    gospel side, says the \textit{Orate fratres}, and turns back to the altar
    without completing his turn. The \textit{Suscipiat} is not said.

    \rubric B sings in the ferial tone with hands joined \textit{Oremus.
    Præceptis, etc.,} then with hands extended \textit{Pater noster, etc.} The
    choir answers \textit{Sed libera nos a malo,} and B says (\textit{submissa
    voce}) \textit{Amen.}

    \rubric With hands still extended, B says in the ferial tone \textit{Libera
    nos, etc.,} but does not make the sign of the cross with the paten. The
    choir answers \textit{Amen.}

    \rubric Minsters kneel as at the elevation. B genuflects, places the paten
    under the Sacred Host and holds the paten with his left hand, while with
    his right he elevates the Sacred Host. The Blessed Sacrament is not
    incensed, the chasuble is not raised, no sign is made with the clapper.

    \rubric The ministers rise. D uncovers the chalice. B, without
    genuflecting, divides the Sacred Host into three parts and drops the
    smallets into the chalice, omitting the sign of the cross and the prayer.

    \rubric The \textit{Agnus Dei} and the \textit{Pax} are omitted.

    \rubric Before communion B says only one prayer \textit{Perceptio Corporis,
    etc.,} then \textit{Panem cœlestem} and \textit{Domine non sum dignus,} as
    usual, and communicates.

    \rubric D uncovers the chalice; B and his assistants genuflect and B
    gathers the fragments, omitting the \textit{Quid retribuam, etc.,} and
    holding the paten, as usual, receives the Sacred Particles with the wine,
    without making the sign of the cross with the chalice.

    \rubric Mc2 gets the zucchetto and places it on B's head.

    \rubric D pours wine and water over the fingers of B, as usual, but the
    \textit{Corpus tuum Domine, etc.} is not said. B dries his fingers,
    receives the ablution, inclines moderately, and with his hands joined
    before his breast says in a clear tone of voice the prayer \textit{Quod ore
    sumpsimus, etc.}

    \rubric D goes immediately after pouring wine and water for the ablutions
    to the credence and removes his broad stole and dons the folded chasuble.

    \rubric Meanwhile, SD arranges the chalice and takes it to the credence; AP
    closes the missal; the clergy extinguish their candles.

    \rubric B goes to the epistle side, receives the miter from D and washes
    his hands, AP ministering the towel.

    \rubric B goes to the foot of the altar where D removes the miter. All line
    up in the usual manner to process into the sacristy. At Mc1's signal, all
    genuflect to the cross and retire to the sacristy, where B is devested in
    the usual manner.


\section{Exposition of the Relic of the Cross}

    \rubric The servers necessary are MC and TH, and two ACs with lighted
    candles. Two candles are lighted on the altar.

    \rubric All process in as usual, no bells being sounded, and C, vested in
    surpice, red stole and red chasuble, carrying a covered reliquary
    containing a relic of the True Cross. All line up as usual and genuflect to
    the cross. ACs place their candles on either side of the altar, as at
    Benediction of the Blessed Sacrament, but remain standing.

    \rubric C goes immediately to the center of the altar, and places the relic
    on it and uncovers it.

    \rubric C bows profoundly and joins MC and TH at the foot of the altar.
    Incense is imposed as at Benediction, with the exceptions that it is not
    blessed and C and servers do not kneel.

    \rubric C incenses the relic with three doubles, servers holding the ends
    of his cope, and all genuflecting before and after, remaining standing.

    \rubric After the thurible is returned to TH, all genuflect to relic and
    cross together and retire to the sacristy.

    \rubric When it is time to remove the relic, all enter as before.

    \rubric After genuflecing, incense is imposed, and the relic is incensed as
    before. C ascends the altar and genuflects, MC signals all to kneel. With a
    single sign of the cross, C gives benediction with the relic, then turns
    back to the altar and replaces the relic.

    \rubric MC signals all to stand. C covers the relic, and ACs bring their
    candles to the front of the predella.

    \rubric C descends with the relic. All genuflect to the cross, and retire
    to the sacristy.

}

