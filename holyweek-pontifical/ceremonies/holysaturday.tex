\chap{Solemn Holy Saturday}{

\section{Preparation}

Work in progress\dots

\section{Blessing of the Fire}

\rubric Everyone must be out of the chapel.

\rubric TH\footnote{With boat and empty thurible.} and ACs\footnote{AC1 has the
aspersory, AC2 has the grains of incense.} lead the procession in to the
sanctuary, AC1 in the middle; SD, without maniple, follows them, carrying the
cross; C, in cope, follows, flanked by D (right) and MC (left); both C and D
wear birettas but neither has a maniple. TH and ACs stop at the edge of the
sanctuary and face the altar. 

\rubric SD stops between\footnote{If the choir is not taking part in the
procession, SD is in the middle of the sanctuary. Otherwise he is closer to the
TH and ACs depending on the number of people in the procession.} the altar and
the TH and ACs. C, D and MC walk up to the foot of the altar. The schola
follows SD immediately, and the rest of the choir walks between C and the
schola\footnote{For Brooksville, it is better if the TH and ACs leave first, SD
following them, and all four stop at their pre-determined positions. Then C
follows with D and MC, and they are followed by the choir, which is followed by
the schola. Thus the correct procession is formed with as little trouble as
possible.}.

\rubric All genuflect to the Cross except C, who bows, and SD, who does
nothing.

\rubric All turn and process to the fire.

\rubric TH and ACs turn to their right and stand to the ``gospel side'' of the
lectern; SD steps aside before reaching the fire; schola walks to the left of
the fire, standing at 9 o'clock\footnote{The lectern being 12 o'clock.}; the
rest of the procession stands to the right and left of the fire, out of the
way, but retaining their order as much as possible; C, D and MC go to the right
and left of the fire and stand before the lectern; SD stands with the fire
between him and C.

\rubric D takes C's biretta with kisses and hands his and C's biretta to MC.

\rubric C reads \textit{Dóminus vobíscum} and the three Orations; D attends to
cope.

\rubric At the fourth Oration, C blesses incense, AC2 approaching. TH removes
the coals from the fire\footnote{If the coals fall from their place and are
lost, TH takes some of the embers of the fire.}.

\rubric Imposition of incense; sprinkling of fire and incense; incensation.

\rubric D removes violet folded chasuble and stole and puts on white maniple,
stole and dalmatic. D's biretta and violet vestments get taken to the sedilia
by a sacristan.

\rubric Second imposition of incense, this one for the procession.

\section{Procession into Church}

\rubric TH (left) and AC2 (right) lead; SD with processional cross; D with
triple candle and reed\footnote{Both referred to as ``reed'' in these notes.}
and AC1 (left); C and MC (left); schola; rest of choir.

\rubric Just inside the doors of the church the procession\footnote{The
``procession'' is considered to be TH, ACs, SD, D, C and MC. Instructions for
stopping are relative to C's position.} stops. 

\rubric AC1 lights one of the candles on the reed. D raises the reed and sings
\textit{Lumen Christi} and all (but SD) genuflect. Schola answers \textit{Deo
grátias} and all rise.

\rubric Procession continues and this same ceremony happens in the middle of
the church, in a higher tone.

\rubric Procession continues until the procession is in the sanctuary and this
ceremony happens for a third time in a yet higher tone.

\rubric Order at the foot of the altar, from Gospel side: TH, SD, C, D, AC1,
AC2.

\rubric MC stands behind and to the right of D; D gives the reed to
AC1\footnote{AC1 does not genuflect with the reed, but does make the choir
bows.}.

\rubric All bow to choir\footnote{Therefore, MC should wait until the choir is
in position before signalling the bow.} then genuflect.

\rubric C ascends and kisses the altar, then stans to the Epistle side.

\rubric MC gives \textit{Lectionarium} to D, who ascends immediately and asks C
for the blessing. D does not say the \textit{Munda cor} or kiss C's hand.

\rubric D returns to the floor, where all genuflect and bow to the choir, then
go\footnote{This turn should be a ``hinged motion'', TH being the hinge in this
case.} to the lectern for the \textit{Exsultet.}

\rubric C turns to face D.

\section{Paschal Candle}

\rubric D incenses book immediately; he begins the \textit{Exsultet} without
\textit{Dóminus vobíscum} and without signing; TH gets rid of his thurible.

\rubric At the proper times, D puts grains of incense\footnote{Research what is
needed for valid grains.} into the Paschal candle, lights it and pauses at the
words \textit{apis mater edúxit} while all the light in the chapel are lighted.

\rubric After the \textit{Exsultet} SD goves cross to TH; AC1 puts the reed on
its stand.

\rubric All return to the foot of the altar the same way they came; all
genuflect and bow to the choir.

\rubric C descends \textit{per breviorum} and precedes D \& SD to the sedilia.

\rubric C takes off cope and puts on violet maniple and chasuble.

\rubric D removes his white vestments and puts on violet maniple, stole and
folded chasuble.

\rubric SD puts on violet maniple.

\rubric Ministers return to the Missal \pbr.

\section{Prophecies}

\rubric ACs accompany Prophecy singers to the lectern, which is in the usual
spot for singing the Epistle.

\rubric D \& SD in Introit positions for the reading of the Prophecies. Towards
the end of each Prophecy, D \& SD stand \textit{unum post alium.}

\rubric After the first eleven Prophecies, C sings \textit{Orémus}; D sings
\textit{Flectámus génua} as he genuflects; SD sings \textit{Leváte} as he
rises. C does not genuflect.

\rubric After the Oration, D \& SD return to Introit positions.

\rubric C, D \& SD may sit while the Prophecies are being sung, and for the
singing of the Tracts.

\rubric During the 12\textsuperscript{th} Prophecy, MC prepares cushions at the
foot of the altar, if they are going to be used for the prostration; ACs light
candles.

\section{Blessing of the Font}

\rubric After the Prophecies are finished, Ministers go to sedilia \pbr;
maniples off, C removes chasuble and puts on a violet cope.

\rubric CB with cross and ACs with lit candles go to the center of the
sanctuary. TH takes the Paschal candle and walks behind them, to lead the
procession. The schola lines up before CB and ACs.

\rubric MC lines ministers up at the foot of the altar, with birettas.

\rubric MC signals for a genuflection, then a choir bow, and all process to the
font, the schola singing \textit{Sicut cervus}.

\rubric At the entrance, TH, ACs and CB turn to face C.

\rubric MC holds the Missal and C sings, without biretta the Oration
\textit{Omnípotens sempitérne Deus} in the ferial tone.

\rubric Once the Oration is finished, T stands close to the font, preferably to
the right of D, but not blocking MC's way.

\rubric CB and ACs stand in the back of the baptistry, facing the font.

\rubric The schola enters the baptistry and stands in front of the CB and ACs.

\rubric Ministers approach the lectern and remove birettas.

\rubric C sings \textit{Dóminus vobíscum, Orémus} and the first oration
\textit{recto tono.}

\rubric C begins Preface with hands joined. \textit{\dots gratiam de Spiritu
Sancto}: C divides water in the form of a cross. D ministers towel\footnote{C
dries his hand every time he touches the water.}. \textit{\dots inficiendo
corrumpat}: C touches water. \textit{\dots indulgentiam consequatur}: C makes
the sign of the cross over the water three times without touching it.
\textit{\dots ferebatur}: C sprinkles water\footnote{MC makes sure that the
Missal is closed before C sprinkles.} in the four directions. C sings
\textit{Hæc nobis præcepta} in the Lesson tone, breathes over the water three
times in the form of a cross, and continues \textit{Tu has simplices aquas} in
the same tone. D takes Paschal Candle from TH.

\rubric D gives Candle to C have he has finished singing and helps C lower it
into the water. C sings \textit{Descendat} in the preface tone, raises Candle
out of the water, then lowers it again, deeper than before. C sings
\textit{Descendat} in a higher tone, raises Candle out of the water, then
lowers it yet again, this time to the bottom. C sings \textit{Descendat} for a
final time, in a higher tone than before.

\rubric With the Candle still in the water, C breathes three times over the
water in the form of a letter Psi $\textrm{\psi}$. C sings \textit{Totámque hujus\dots
effectu} and removes the Candle. D retuns the Candle to TH, who dries it.

\rubric C continues in the preface tone, singing as far as \textit{infántiam
renascátur} and then reads the conclusion in a loud voice; schola answers
\textit{Amen.}

\rubric MC fills aspersory with water and hands it to D, who hands the
sprinkler to C with the usual kisses. 

\rubric C blesses himself, sprinkles those who are around the font and then,
flanked by D \& SD and proceded by MC, sprinkles the congregation.

\rubric \textbf{N.B.} If there is another priest present, he may sprinkle the
congregation. Wearing a surplice and violet stole, he presents the sprinkler to
the C with kisses. After the C has blessed him and the Ministers, he receives
the sprinkler from the C with kisses. With the MC at his right, he sprinkles
the congregation\footcite[p. 230]{hweekls}.

\rubric MC pours the needed quantity of water into the baptismal font.

\rubric D hands C Oil of Catechumens.

\rubric C pours in the form of a cross a little of the oil into the water
saying in a loud voice \textit{Sanctificétur et fecundétur}.

\rubric C pours Chrism in the same way, but saying \textit{Infúsio Chrísmatis.}

\rubric C takes both oils and pours them at the same time into the water saying
the \textit{Commixtio Chrismatis} and making the sign of the cross three times.

\rubric C mixes the oil and water with his right hand. C cleanses his hands, D
ministering the towel\footnote{MC should at this time cover the font, lest any
of the Baptismal Water be taken (Holy Saturday, Brooksville, FL, 2021).}.

\rubric MC gives birettas to D \& SD.

\rubric The procession goes back the same way it came. 

\rubric The schola begins the Litany of the Saints as soon as the procession
leaves the baptistry; the invocations are doubled.

\rubric In the sanctuary, TH, CB and ACs stand aside to permit Ministers to
pass; Ministers genuflect, bow to the choir, and go to the sedilia.

\rubric TH puts the Candle in the holder and turns it so that the cross faces
the people; CB and ACs go to the credence.

\rubric C removes cope, D \& SD remove folded chasubles.

\section{Litany of the Saints}

\rubric At a signal from MC, C, D \& SD go the the altar, bow to the choir,
kneel and prostrate themselves.

\rubric MC kneels behind and to the right of D.

\rubric At \textit{Per sanctam resurrectiónem tuam} ACs and TH go to the center
of the sanctuary; MC gets birettas.

\rubric At \textit{Peccatóres} Ministers rise. C bows and others genuflect, and
all go to the sacristy, ministers wearing birettas.

\section{First Mass of Easter}

\rubric Ministers put on white vestments.

\rubric Altar is prepared for Mass.

\rubric At \textit{Christe audi nos} MC gives a signal and all bow to the Cross
in the sacristy. Procession in as usual\footcite[If the baptismal font was
blessed, the AC's candles are already on the credence table; if the font was
not blessed, the ACs carry them.][footnote, p. 233]{hweekls}.

\rubric At the foot of the altar, the Ministers bow before genuflecting.

\rubric Differences in the Mass:

\rubric No Introit; D \& SD line up in Introit position for \textit{Kyrie} and
then go to the center for the \textit{Glória}.

\rubric Bells are rung throughout\footnote{Check if the tower bells only should
continue.} the sung \textit{Glória}; Violet coverings are removed.

\rubric After C blesses SD after the Epistle, he sings the \textit{Allelúja}
thgree times, each in a higher tone, each time the choir answers in the same
tone. D \& SD stand in Introit positions. Afterwards C recites Versicle and
Tract in a subdued voice. SD changes the book after the Tract.

\rubric ACs do not carry candles in the Gospel procession.

\rubric No Creed

\rubric No Offertory Prayer, but \textit{Orémus} is said.

\rubric No \textit{Agnus Dei} and the \textit{Pax} is not given. After C sings
\textit{Pax Dómini sit semper vobíscum} D goes to the Gospel side after placing
the pall on the chalice; SD goes to the right of C.

\rubric TH get the thurible after distribution.

\section{Vespers}

\rubric Vespers begin immediately after the ablutions.

\rubric D \& SD stand in the Introit positions and recite the \textit{Allelúja}
antiphon and Psalm \textit{Laudéte} alternately with C.

\rubric After the choir is finished singing, C intones \textit{Véspere autem
sábbati} and recites it with D \& SD.

\rubric When the choir begins the \textit{Magnificat} Ministers sign
themselves, go to the center, impose incense, and the altar is incensed as
usual. C recites \textit{Magnificat} with D \& SD, repeating the antiphon
either during the incensation or afterwards at the book.

\rubric Incensation of C, clergy, choir and people as usual. Afterwards TH
returns the thurible to the sacristy.

\rubric After the antiphon has been repeated by the choir Ministers to to
center. C sings \textit{Dóminus vobíscum}, Ministers return to the Missal and C
reads the Oration \textit{Spíritum nobis.}

\rubric Ministers return to the center, C sings \textit{Dóminus vobíscum} and D
sings \textit{Ite, Missa est, allelúja, allelúja.}

\rubric Mass concludes in the usual manner.

}

