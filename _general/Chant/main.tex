% !TEX program = lualatex

\documentclass[letterpaper]{article}

\author{Gregory R. Barnes}
\title{Chant}

\usepackage{geometry}
% BOOKLET:
%\geometry{paperheight=8.5in, paperwidth=5.5in, top=0.5in, bottom=0.5in, inner=0.5in, outer=0.5in,}
% LETTERPAPER:
%\geometry{paperwidth=8.5in, paperheight=11in, left=1.0in, right=1.0in, top=1.0in, bottom=1.0in,}

% GREGORIO
%\usepackage[autocompile]{gregoriotex}
%\gresetlinecolor{gregoriocolor}
%\grechangestaffsize{17}
%\gresetinitiallines{1}

\thispagestyle{empty}

\usepackage[T1]{fontenc}
\usepackage{tgpagella}
\usepackage{pifont}

\usepackage{pifont}
\usepackage{xcolor}

\begin{document}
	\maketitle
	\section*{Parts of the Mass Sung}
	The ministers at the altar, the \textit{schola cantorum,} and the faithful have parts to sing in the Mass. The Ministers may never use any other music than plainsong. The \textit{schola} amy, withing reasonable limits, use polyphonic music approved by the Church. \\
	The parts of the Mass which must be sing, if the Mass is High or Solemn, are the Introit, \textit{Kyrie, Gloria,} Gradual, \textit{Alleluia,} Tract, Sequence, Creed\footnote{While the Creed is being sung, the celebrant may not go on with the Mass. (SRC 1936.)}, Offertory, \textit{Sanctus, Benedictus, Agnus Dei,} and \textit{Communio.}\footnote{Cf. SRC 3994, 7.} These the \textit{schola} must never truncate or omit.\footnote{SRC 2424, 2; 3385, 7.}
    % \gregorioscore{}
	\section*{References}
		\begin{itemize}
			\item O'Connell, Rev. Laurence J. \textit{The Book of Ceremonies} Milwaukee: The Bruce Publishing Company, 1944
			%\item First, Last. \textit{Title, vol/part} Edition. City: Publisher, year
		\end{itemize}
\end{document}