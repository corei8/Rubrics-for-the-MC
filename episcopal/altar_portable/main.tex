\documentclass[twocolumn]{report}
\title{Consecration of a Portable Altar}
\author{Gregory R. Barnes}

\usepackage{pifont}
\usepackage{xcolor}
\usepackage{tabularx}

\begin{document}
	\maketitle
\section{Preparation}
\subsection{Overview}
\begin{enumerate}
	\item A portable altar may be consecrated at any time.
	\item After the consecration, Mass is celebrated over the altar just consecrated. This Mass may be said by the consecrating bishop or be a priest delegated by him.\footnote{Menghini, p. 315, \P{} 4. Therefore it is not necessary that the Bishop himself fulfill the rubric \textit{jejuno tamen stomacho,} as is in the Pontificale.}
\end{enumerate}
\subsection{Setup}
\begin{enumerate}
	\item The stone to be consecrated is placed on a table which is covered by a white cloth.\footnote{The purpose of this cloth is to absorb any oil that may flow from the stone.}
	\item On the same table, or on another, covered by a white cloth:
	\begin{enumerate}
		\item One small vessel of holy water with aspergillum.
		\item One small vessel with water to be consecrated.
		\item One small vessel with the Oil of the Catechumens.
		\item One small vessel with Chrism.
		\item Small quantity of ash in a metal dish.
		\item Small quantity of salt in a metal dish.
		\item Small quantity of wine in an ampulla.
	\end{enumerate}
\end{enumerate}
\subsection{Preparation of the Bishop}

If a public ceremony, the Bishop is vested in amice, alb, cinture, stole, white cope and linen miter. If a private ceremony, the Bishop is bested in a white stole over his rochet and the simple miter.

The Bishop washes his hands\footnote{Menghini, p. 317, \P{} 9} before vesting.

\section{Before the altar}
\begin{enumerate}
	\item B before the altar says: \textit{Deus onmipoténtem...}
	\item With bared head (detecto capite\footnote{\textit{Pontificale Romanum} (1862), p. 174}), B genuflects before the altar to be consecrated, saying: \textit{Dues, in adjutórium meum inténde.} While B rises, the choir says: \textit{Dómine, ad adjuvándum me festína.} Wtihout miter, B says: \textit{Glória Patri...,} choir responding.
	\item This responsory is said three times, each time in a higher voice.
	\item B reads: \textit{Omnípotens, sempitérne Deus...}
\end{enumerate}
\section{Blessing the Water}
\subsection{Exorcism of the Salt and the Water}
B with miter and standing in the same place says \textit{Exorcizo te, creatura salis...}. B then without miter reads the versicles and the oration. B receives the miter for the exorcism of the water, then removes it for the versicles and the oration, as before.
\subsection{Blessing of the Ashes}
\begin{enumerate}
	\item Without miter, B, standing in the same place, blesses the ashes.
	\item B takes the salt and mixes it with the ashes in the form of a cross saying: \textit{Commíxtio salis, et cíneris páriter fiat. Pa \ding{64} tris, et Fí \ding{64} lii, et Spíritus \ding{64} Sancti. R. Amen.}
	\item Taking a handful of the salt and ashes, B sprinkles the mixture three times into the water saying: \textit{Commíxtio salis, cíneris, et aquae páriter fiat. Pa \ding{64} tris, et Fí \ding{64} lii, et Spíritus \ding{64} Sancti. R. Amen.}
\end{enumerate}
\subsection{Blessing of the Wine}
\begin{enumerate}
	\item Without miter, B, standing in the same place, blesses the wine.
	\item B takes the wine and pours it into the water in the form of a cross saying: \textit{Commíxtio vini, salis, cíneris et aquae páriter fiat. Pa \ding{64} tris, et Fí \ding{64} lii, et Spíritus \ding{64} Sancti. R. Amen.}
	\item Without miter B reads the oration: \textit{Omnípotens sempitérne Deus...}
\end{enumerate}
\section{Signing the Stone}

\begin{enumerate}
	\item With miter, B takes the water with the thumb of his right hand and makes crosses in the middle of the stone saying: \textit{Sancti \ding{64} ficétur, et conse \ding{64} crétur haec tábula,} making signs of the cross in the usual manner at \textit{In nómine Pa \ding{64} tris, et Fí \ding{64} lii, et Spíritus \ding{64} Sancti. Pax tibi.}
	\item This signing is repeated in each corner of the altar, first on the Gospel side, back, then on the Epistle side, front, then on the Gospel side, front, then the Epistle side, back.
\begin{center}
	\begin{tabular}{ | l c r | }
		\hline
		\ding{64} \textbf{1} &                      & \textbf{4} \ding{64} \\
							 & \ding{64} \textbf{0} &                      \\
		\ding{64} \textbf{3} &                      & \textbf{2} \ding{64} \\
		\hline
	\end{tabular}
\end{center}
\item The ministers recite Psalm 50 with the antiphon \textit{Aspérges me.} The antiphon is doubled.
\end{enumerate}
\end{document}