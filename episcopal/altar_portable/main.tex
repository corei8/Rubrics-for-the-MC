\documentclass[twocolumn]{report}
\title{Consecration of a Portable Altar}
\author{Gregory R. Barnes}

\usepackage[T1]{fontenc}
\usepackage{tgpagella}
\usepackage{pifont}
\usepackage{xcolor}
\usepackage{tabularx}
%\usepackage{libertine}
%\usepackage[nottoc]{tocbibind}

\begin{document}
	\maketitle
\section*{Preparation}
\subsection*{Overview}
\begin{enumerate}
	\item A portable altar may be consecrated at any time.
	\item For a private consecration, the crozier is not used at any time\footnote{Moretti, p. 477, \P{} 3053.}.
	\item After the consecration, Mass is celebrated over the  consecrated altar. This Mass may be said by the consecrating bishop or be a priest delegated by him.\footnote{Menghini, p. 315, \P{} 4. Therefore it is not necessary that the Bishop himself fulfill the rubric \textit{jejuno tamen stomacho,} as is in the Pontificale.}
\end{enumerate}
\subsection*{Setup} 
%! fix the setup
\begin{enumerate}
	\item On the altar to be consecrated\footnote{If the stone to be consecrated is small, then it is placed on a table which is covered by a white cloth. The sole purpose of this cloth is to absorb any oil that may flow from the stone.}:
	\begin{enumerate}
		\item A silver tray with the relic and three grains of incense; the tray should be covered with a silk, red veil\footnote{Moretti, p. 477 \P{} 3053.}
		\item 
	\end{enumerate}
	\item On the same table, or on another table, which is covered by a white cloth:
	\begin{enumerate}
		\item One small vessel of holy water with aspergillum.
		\item One small vessel with water to be consecrated.
		\item One small vessel with the Oil of the Catechumens.
		\item One small vessel with Chrism.
		\item Small quantity of ash in a metal dish.
		\item Small quantity of salt in a metal dish.
		\item Small quantity of wine in an ampulla.
		\item One finger towel.
		\item Aspergillum.
	\end{enumerate}
	\item On the credence:
	\begin{enumerate}
		\item White stole.
		\item Cloth-of-gold miter.
	\end{enumerate}
	\item The servers:
	\begin{enumerate}
		\item Master of Ceremonies (MC)
		\item Thurifer (TH)
		\item Miter Bearer (MB)
		\item Two Chaplains (Chs)
	\end{enumerate}
\end{enumerate}
\subsection*{Preparation of the Bishop}

B vested in a white stole over his rochet and the simple miter. 

%! is this true?
The Bishop washes his hands\footnote{Menghini, p. 317, \P{} 9} before vesting.

\section*{Before the altar}
\begin{enumerate}
	\item B before the altar says: \textit{Deus omnipoténtem...}
	\item Without miter, which is removed by MC or Ch, B genuflects before the altar to be consecrated, saying: \textit{Deus, in adjutórium meum inténde.} Chs respond: \textit{Dómine, ad adjuvándum me festína.} then B rises. Wtihout miter, B says: \textit{Glória Patri...,} choir responding.
	\item This responsory is said three times, each time in a higher voice.
\end{enumerate}
\section*{Blessing the Water}
\subsection*{Exorcism of the Salt and the Water}
\begin{enumerate}
	\item B receives miter from MC or Ch.
	\item B with miter and standing in the same place says \textit{Exorcízo te, creatúra salis...}. If the church is not being consecrated, then in the place of \textit{hujus ecclésiæ et altáris} B reads only \textit{hujus altáris\footnote{ibid. \P{} 10}.}
	\item B then without miter reads the versicles and the oration.
	\item B receives the miter for the exorcism of the water, then removes it for the versicles and the oration, as before.
\end{enumerate}
\subsection*{Blessing of the Ashes}
\begin{enumerate}
	\item Without miter, B, standing in the same place, blesses the ashes.
	\item B takes the salt and mixes it with the ashes in the form of a cross saying: \textit{Commíxtio salis, et cíneris páriter fiat. Pa \ding{64} tris, et Fí \ding{64} lii, et Spíritus \ding{64} Sancti. R. Amen.}
	\item Taking a handful of the salt and ashes, B sprinkles the mixture three times into the water saying: \textit{Commíxtio salis, cíneris, et aquae páriter fiat. Pa \ding{64} tris, et Fí \ding{64} lii, et Spíritus \ding{64} Sancti. R. Amen.}
\end{enumerate}
\subsection*{Blessing of the Wine}
\begin{enumerate}
	\item Without miter, B, standing in the same place, blesses the wine.
	\item B takes the wine and pours it into the water in the form of a cross saying: \textit{Commíxtio vini, salis, cíneris et aquae páriter fiat. Pa \ding{64} tris, et Fí \ding{64} lii, et Spíritus \ding{64} Sancti. R. Amen.}
	\item Without miter B reads the oration: \textit{Omnípotens sempitérne Deus...}
\end{enumerate}
\section*{Signing the Stone}
\subsection*{Signing with the water}
\begin{enumerate}
	\item With miter, B takes the water with the thumb of his right hand and makes a single cross in the middle of the stone saying: \textit{Sancti \ding{64} ficétur, et conse \ding{64} crétur haec tábula,} blessing with his right hand in the usual manner at \textit{In nómine Pa \ding{64} tris, et Fí \ding{64} lii, et Spíritus \ding{64} Sancti. Pax tibi.}
	\item This signing is repeated in each corner of the altar, first on the Gospel side, back, then on the Epistle side, front, then on the Gospel side, front, then the Epistle side, back.
\begin{center}
	\begin{tabular}{ | l c r | }
		\hline
		\ding{64} \textbf{2} &                      & \textbf{5} \ding{64} \\
							 & \ding{64} \textbf{1} &                      \\
		\ding{64} \textbf{4} &                      & \textbf{3} \ding{64} \\
		\hline
	\end{tabular}
\end{center}
\item B wipes his thumb in a towel provided by MC, then is handed the aspergillum.
%\item The ministers recite Psalm 50 with the antiphon \textit{Aspérges me.} The antiphon is doubled.
%todo double check this procedure.
\item B intones the antiphon \textit{Aspérges me} which he recites with Chs. Psalm 50 is then recited, followed by the antiphon; \textit{Glória Patri} is omitted.
\item While the Psalm and Antiphon are being recited, \textit{Pontifex cum mitra aspergit cum aspersorio de herba hyssopi facto de ipsa aqua tabulam, per ejus circuitum tribus vicibus.}
\end{enumerate}
\subsection*{Incensation}
\begin{enumerate}
	\item The stone is wiped clean with a linen cloth.
	\item B, without miter says the versicles and then the oration \textit{Deus, qui es visibilium.}
	\item The ministers say the antiphon: \textit{Dirigatur oratio mea sicut incensum in conspectu tuo, Domine.}
	\item After the antiphon is begun, B accepts miter and incenses the stone three times, circling it.
\end{enumerate}
\section*{Anointing with Holy Oil}
\subsection*{First anointing}
\begin{enumerate}
	\item After the incesation, B begins, with the ministers, the antiphon \textit{Erexit Jacob.} The ministers continue the antiphon and recite Psalm 83, the \textit{Glória Patri} begin omitted.
	\item While the antiphon and the psalm are being recited, B stands with miter, takes some Oil of the Catechumens on his right thumb, and signs the stone five times with clear crosses, in the same manner as before with the water. 
	\item The same formula is pronounced for each cross as before: \textit{Sancti \ding{64} ficétur, et conse \ding{64} crétur haec tábula, In nómine Pa \ding{64} tris, et Fí \ding{64} lii, et Spíritus \ding{64} Sancti. Pax tibi.}
\end{enumerate}
\subsection*{First incensation}
\begin{enumerate}
	\item B with his ministers begins the antiphon \textit{Dirigatur oratio mea sicut incensum in conspectu tuo, Domine.}
	\item While the antiphon is being recited, B incenses the stone, circling it.
	\item After the conclusion of the antiphon, B, without miter, says \textit{Orémus,} the ministers answering \textit{Flectámus génua, Leváte.} B recites the oration \textit{Adsit, Dómine.}
\end{enumerate}
\subsection*{Second anointing}
\begin{enumerate}
	\item B begins with his ministers the antiphon \textit{Mane surgens Jacob.} The ministers continue the antiphon and recite Psalm 91, the \textit{Glória Patri} begin omitted.
	\item While the antiphon and the psalm are being recited, B stands with miter, takes some Oil of the Catechumens on his right thumb and again anoints the stone, as before.
\end{enumerate}
\subsection*{Second incensation}
\begin{enumerate}
	\item B with his ministers begins the antiphon \textit{Dirigatur oratio mea sicut incensum in conspectu tuo, Domine.}
	\item While the antiphon is being recited, B incenses the stone, circling it.
	\item After the conclusion of the antiphon, B, without miter, says \textit{Orémus,} the ministers answering \textit{Flectámus génua, Leváte.} B recites the oration \textit{Adésto, Dómine.}
\end{enumerate}
\subsection*{Third anointing}
\begin{enumerate}
	\item B begins with his ministers the antiphon \textit{Unxit te, Deus.} The ministers continue the antiphon and recite Psalm 45, the \textit{Glória Patri} begin omitted.
	\item While the antiphon and the psalm are being recited, B stands with miter, takes some Chrism on his right thumb and again anoints the stone, as before.
\end{enumerate}
\subsection*{Third incensation}
\begin{enumerate}
	\item B imposes and blesses incense.
	\item B with his ministers begins the antiphon \textit{Dirigatur oratio mea sicut incensum in conspectu tuo, Domine.}
	\item While the antiphon is being recited, B incenses the stone, circling it.
	\item After the conclusion of the antiphon, B, without miter, says \textit{Orémus,} the ministers answering \textit{Flectámus génua, Leváte.} B recites the oration \textit{Exáudi nos, Deus noster.}
\end{enumerate}
\section*{Anointing the Confession}
\begin{enumerate}
	\item Having concluded the oration, B with miter takes Chrism with his right thumb and anoints the confession in the middle saying \textit{Conse \ding{64} crétur, et sancti \ding{64} ficétur hoc sepélcrum. In nómine Pa \ding{64} tris, et Fí \ding{64} lii, et Spíritus \ding{64} Sancti. Pax huic domui.}
	\item B, without miter, places the relics with three grains of frankincense on the stone, or on the altar, and he closes the confession.
	\item B stands without miter and says: \textit{Orémus. Deus, qui ex ómnium cohabitatióne...}
\end{enumerate}
\section*{Cleaning the Stone}
\begin{enumerate}
	\item B begins with the ministers the antiphon \textit{Ecce odor.} The ministers continue the antiphon and recite Psalm 86, the \textit{Glória Patri} begin omitted.
	\item While the antiphon and the psalm are being recited, B with miter cleanses the stone of all the oil.
	\item After the conclusion of the psalm, B standing with miter says: \textit{Lápidem hunc.}
	\item B begins the \textit{Aedificavit Moyses} with his ministers.
	\item Afterwards, B standing with miter says: \textit{Dei Patris omnipoténtis.}
\end{enumerate}
\section*{Burning of Incense on the Altar}
\begin{enumerate}
	\item B stands without miter before the altar, with the incense to be burned on the altar, says the versicles and the oration \textit{Dómine, Deus omnípotens...}
	\item B sprinkles the incense with holy water and takes up the miter.
	\item With his own hand the B makes five crosses of five grains; each cross is placed over the one of the five spots anointed previously.
	\item Over each cross of incense he places a cross of thin candle (taper), to measure the cross of the grains of incense, and the top of each cross is lit, and with the incense it is burnt.
	\item All of the crosses having been lit, B without miter kneels before the altar and begins with the ministers the verse: \textit{Allelúia\footnote{The \textit{Allelúia} being omitted if the ceremony take place between Septuagesima and Easter.}. V. Veni, Sancte Spíritus,...}
	\item After the versicle is concluded, B stands before the altar without miter and the schola or the ministers sing the antiphons \textit{Ascéndit fumus} and \textit{Stetit Angelus.}
	\item After the conclusion of the antiphons, B says \textit{Orémus,} the ministers answering \textit{Flectámus génua, Leváte.} B recites the oration \textit{Súpplices tibi.}
\end{enumerate}
\section*{Preface}
\begin{enumerate}
	\item After the oration, B, in the same place and with hands extended, reads the preface in the middle voice.
	\item He reads the conclusion \textit{Per Dominum nostrum...} in a lower voice, so it can be heard only by the bystanders.
\end{enumerate}
\section*{Find a Title}
\begin{enumerate}
	\item B begins and the schola or the ministers continue the antiphon \textit{Confírma hoc,} which is repeated.
	\item B reads the oration: \textit{Quaesumus omnípotens Deus.}
	\item B intones the antiphon \textit{Omnis terra adoret.}
	\item B accepts the miter and incenses the altar in the form of a cross.
	\item B without miter says oration \textit{Descéndat, quaesumus.}
\end{enumerate}
\section*{Conclusion}
\begin{enumerate}
	\item After the oration the stone is wiped by the ministers.
	\item B then prepares himself for celebrating Mass, or the priest who is to celebrate the Mass prepares himself.
	\item The Mass is of the day of dedication of an altar.
\end{enumerate}
\section*{References}
\begin{itemize}
	\item Menghini, I. B. M. \textit{Manuale Sacrarum Cæremoniarum, vol. II.} Editio terita. Ratisbonæ - Romæ - Neo Eboraci: Fredericus Pustet, 1915
	\item Benedicto XIV. \textit{Pontificale Romanum, pars prima.} Mechliniæ: H. Dessain, 1862
	\item Moretti, Aloisius. \textit{De Sacris Functionibus, vol. IV.} Turin, Marii E. Marietti, 1938
\end{itemize}
\end{document}